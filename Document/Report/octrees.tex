\chapter{Octrees}

An \octree is a spatial tree \idxs{data}{structure} in which each \idxs{internal}{node} has exactly eight \idxse{child}{node}{children}. Octrees are often used to represent \idxs{three-dimensional}{data}, for which each node corresponds to a \cube in \idxs{three-dimensional}{space}, and where each of those cubes that corresponds to a \idxs{parent}{node} is subdivided into eight half-sized cubes, corresponding to the children of the node. Octrees are especially usefull when the data that needs to be repersented has different requirements for the \LOD in different parts of space. Octrees are a variant of \quadtrees in which each internal node has four children and often are used to represent \idxs{two-dimensional}{data}.

\begin{figure}
    \centering
    \subcaptionbox{\label{fig:quadtree}}[.495\textwidth]{
        \begin{tikzpicture}[x=.4\textwidth,y=.4\textwidth]
            \newrobustcmd{\drawplus}[3]{;\draw (#1+#3/2,#2) -- (#1+#3/2,#2+#3); \draw (#1,#2+#3/2) -- (#1+#3,#2+#3/2);}
            \newrobustcmd{\rectanglepath}[1]{-- ++(#1,0) -- ++(0,#1) -- ++(-#1,0) -- cycle}

            \draw (0,0) \rectanglepath{1};
            \drawplus{0}{0}{1}
            \drawplus{0}{0}{1/2}
        \end{tikzpicture}
    }
    \subcaptionbox{\label{fig:octree}}[.495\textwidth]{
        \begin{tikzpicture}[x=.4\textwidth,y=.4\textwidth]
            \newrobustcmd{\drawplus}[3]{;\draw (#1+#3/2,#2) -- (#1+#3/2,#2+#3); \draw (#1,#2+#3/2) -- (#1+#3,#2+#3/2);}
            \newrobustcmd{\rectanglepath}[1]{-- ++(#1,0) -- ++(0,#1) -- ++(-#1,0) -- cycle}
            
            \draw (0,0) \rectanglepath{1};
            \drawplus{0}{0}{1}
        \end{tikzpicture}
    }
    \caption{\subrefp{fig:quadtree} A \quadtree used to partition twodimensional space. \subrefp{fig:octree} An \octree used to partition three-dimensional space. Note that an octree is essentially the same as a quadtree but extended to work in three dimensions instead of two.}
    \label{fig:quadtree_and_octree}
\end{figure}

%\pgfmathsetseed{1}
%\foreach \col in {black,red,green,blue}
%{
%    \begin{tikzpicture}[x=10pt,y=10pt,ultra thick,baseline,line cap=round]
%    \coordinate (current point) at (0,0);
%    \coordinate (old velocity) at (0,0);
%    \coordinate (new velocity) at (rand,rand);
%    \foreach \i in {0,1,...,100}
%    {
%        \draw[\col!\i] (current point)
%        .. controls ++([scale=-1]old velocity) and
%        ++(new velocity) .. ++(rand,rand)
%        coordinate (current point);
%        \coordinate (old velocity) at (new velocity);
%        \coordinate (new velocity) at (rand,rand);
%    }
%    \end{tikzpicture}
%}

\section{Varying level of detail}

In \thisprojectwork, an octree has been used to partition the \idxs{computational}{domain} (the space in which the simulation will be performed) into the \cells required by the \FVM. Since only the \surface of the water is visible in the \simulation, the \idxsp{surface}{cell}{s} have been given a high \LOD; the \LOD then decreases at the water depth increases, staying a few \idxse{layer of}{cells}{layers of cells} a time on each \LOD, forming a \idxs{LOD}{layer} (a layer consisting of cells which all have the same \LOD, surrounded by cells with other \LODs).

Ideally, although not implemented in \thisprojectwork because of \itslimitedtime, the surface will have a higher \LOD where the \idxsp{surface}{detail}{s} are more important to the simulation, such as close to the \camera where they are more easily seen, and a lower \LOD far away from the camera where the cells take up little \idxse{screen}{space}{space on the screen}, or where they are out of the \FOV.

The total number of cells $N_{\text{t}}$ used in the simulationcan can be \approximated by an expression containing the thicknesses of the different LOD layers and the number of cells visible at the surface that belong to each \LOD accoring to

\begin{equation} \label{eq:number_of_cells_total_sum}
N_{\text{t}} \,=\, \sum_{i\,=\,0}^\infty N_{\text{s},i}\sum_{j\,=\,i}^\infty d_j\cdot 4^{-(j-i)},
\end{equation}

where $N_{\text{s},i}$ is the number of cells visible on the surface belonging to LOD layer $i$, and $d_j$ is the thickness in number of cells of LOD layer $j$, where LOD layer $0$ corresponds to the highest \LOD and an increasing LOD layer index corresponds to a lower \LOD. Different LOD layers can have different thickness; if this is the case, it will also be reflected in the simulation and waves with different wavelengths will be simulated with different \accuracy. In \thisprojectwork, all LOD layers have been given the same thickness, $d$. \eqref{eq:number_of_cells_total_sum} therefore turns into

\begin{equation} \label{eq:number_of_cells_total}
N_{\text{t}} \,=\, d\sum_{i\,=\,0}^\infty N_{\text{s},i}\sum_{k\,=\,0}^\infty 4^{-k} \,=\, d\,N_{\text{s}}\,\frac{1}{1-4^{-1}} \,=\, \frac{4}{3}\,d\,N_{\text{s}},
\end{equation}

where $k = j-i$ and $N_{\text{s}}$ is the total number of cells visible on the surface. We can therefore in this case conclude that

\begin{equation} \label{eq:number_of_cells_total_ordo}
N_{\text{t}} \,=\, O(N_{\text{s}}).
\end{equation}

Hence, the \idx{time step} for updating the \idxs{fluid}{flow} is roughly porportional to the number of \idxsp{surface}{cell}{s} $N_{\text{s}}$, but the simulation still catches all motion under the surface, with decaying \accuracy at increasing \idxsp{water}{depth}{s}.

\section{The differentiating problem}

\subsection{Perturbed cell interfaces method}

\subsection{Distributed velocities method}

\section{Multilevel acceleration}

\section{Spurious wave reflections at level transitions}