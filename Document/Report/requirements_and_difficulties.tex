\chapter{Requirements and difficulties}
\label{chap:requirementsanddifficulties}

\section{Wave dispersion and non-linearity}

One of the greatest challenges when simulating an ocean is the fact that ocean waves are subject to \idxs{wave}{dispersion}, also known as \indexify{frequency dispersion}\index{frequency dispersion!see{dispersion}}, which means that waves with different \wavelengths travel with different speeds. This means that the wave equation,

\begin{equation} \label{eq:wave_equation}
\frac{\partial^2 \eta}{\partial t^2} \,=\, c^2\nabla^2\eta,
\end{equation}

--- a \PDE\xspace --- which otherwise both has a simple definition and is simple to solve numerically, cannot be used, since it assumes that waves of all wavelengths travel with a single speed, $c$. Here, $\eta$ is the free surface elevation as a function of the the horizontal location, $\vec{r}$, and the time, $t$, and $\nabla$ is the \idxs{del}{operator}\index{$\nabla$|see{del operator}} which is commonly used in \idxs{vector}{algebra}.

In Airy wave theory, which threats the propagation of surface waves, the dispersion relation

\begin{equation} \label{eq:dispersion}
\omega^2(k) \,=\, \left(g\,+\,\frac{\gamma}{\rho}\,k^2\right)\,k\,\tanh(k\,h),
\end{equation}

is derived. Here, $\omega$ is the \idxs{angular}{frequency} of one \idxs{wave}{component}, $k$ is the \idx{wavenumber} of the component, $g$ is the \idxs{gravitational}{acceleration}, $\gamma$ is the \idxs{surface}{tension}, $\rho$ is the water density and $h$ is the \idxs{water}{depth}.

While this equation describes the propagation of a wave composed of a single wavelength very well, it cannot tell how a free surface elevation, $\eta$, consisting of multiple wavelengths will evolve. Only if the wave amplitude is very small (typically such that $|\nabla\eta| \ll 1$ and $|\eta| \ll h$, where $h$ is the water depth) can the surface be approximated as linear, and waves with different wavelengths can be individually described by \eqref{eq:dispersion} and superposed on top of each other to form the free surface elevation, without introducing too much error. If the \idxs{wave}{amplitude} on the other hand is not that small, strong non-linear phenomena are likely to take place, including for example \idxs{wave}{breaking}, which won't be caught in the simulation is the surface is linearized.

What is maybe even worse is that there is no simple way of turning this equation into a \PDE represented in the spatial domain (like \eqref{eq:wave_equation}). This increases the difficulty to describe the evolution of the free surface elevation significantly, even for a surface that has already been linearized. This is typically solved by transforming the free surface elevation in some way before processing it.

\section{Fluid--Structure Interaction}

As noted previously, ships have to be affected by waves, and ships also have to give rise to waves. Hence, there has to be a \idxs{two-way}{interaction} between water and ships. For most of the two-dimensional wave models, which just treats the surface as a \idxs{height}{map}, there is no natural way to make water and ships interact with each other.

The ships can quite easily be made to roll in a realistic way by just approximating the \idxs{pressure}{field} felt by the \idxs{ship}{hull} by looking at the free surface elevation, even though this method is slightly incorrect since it doesn't take into account the deviations the ship itself causes the pressure field. But to make ships give rise to waves as they are traveling on the water is more challenging.

One possibility is to use a separate, static height map for the wake, which moves after the ship as it is traveling. This wake will always look the same no matter where the ship is traveling and will not be affected by obstacles in the water. However, if the ship suddenly changes speed or course, so does the wake, which is a highly unnatural behavior for a wake.

A better approach may be to use a \idxs{response}{map} that tells the water how to respond when a ship is traveling on it. In that case, the wake will not be stored as a separate height map, but be merged into the same height map that is used to simulate the waves that affect the ship. The response would off course also depend on the speed of the ship so that the faster the ship goes, the higher the generated waves will be, and for a non-moving ship, there will be no waves generated.