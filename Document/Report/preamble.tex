% % % % % % % % % % % % % % % % % % % % % % % % % % % % % % % % % % % % % % % % % % % % % % % %
% Warnings and their (possible) solutions
% ---------------------------------------
%
% "LaTeX Font Warning: Some font shapes were not available, defaults substituted. " -- \usepackage[T1]{fontenc} (\usepackage{t1enc} ?)
% "No file OMScmtt.fd. on input line 10." -- \usepackage[T1]{fontenc} OR \usepackage{t1enc}
% NOTE: \usepackage[T1]{fontenc} and \usepackage{t1enc} loads an ugly bitmap font by default. Install package cm-super (have not tries) using the MikTeX package manager (if you use Windows) to get around this problem, or \usepackage{lmodern} which replaces the bitmap font with teh Latin Modern font.
% % % % % % % % % % % % % % % % % % % % % % % % % % % % % % % % % % % % % % % % % % % % % % % %


% % % % % % % % % % % % % % % % % % % % % % % % % % % % % % % % % % % % % % % % % % % % % % % %
% Tips and trix
% -------------
%
% Tips for shrinking the size of the paper: http://gurmeet.net/computer-science/latex-tips-n-tricks-for-conference-papers/ (Note: I want to do the opposite)
% % % % % % % % % % % % % % % % % % % % % % % % % % % % % % % % % % % % % % % % % % % % % % % %


% % % % % % %
% PACKAGES  %
% % % % % % %

% Formatting
\usepackage[utf8]{inputenc} % Get å, ä, ö, Å, Ä, Ö
%\usepackage{t1enc} % Gets rid of the warning "No file OMScmtt.fd. on input line 10.", but loads an ugly bitmap font.
\usepackage[resetfonts]{cmap} % Help reader with interpreting ligatures (if you use cmap, you must use it before fontenc)
\usepackage[T1]{fontenc} % Gets rid of warning "No file OMScmtt.fd. on input line 10." but loads an ugly bitmap font
\usepackage{lmodern} % Replaces the default bitmap font used by t1enc and fontenc with the Latin Modern font
%\usepackage{multicol} % An environment that enables a different number of columns, locally.
\usepackage{longtable,tabu} % Handles long tables and text wrapping, see http://ctan.uib.no/macros/latex/required/tools/longtable.pdf and  http://mirror.ctan.org/macros/latex/contrib/tabu/tabu.pdf

% Miscellaneous tools
\usepackage{etoolbox} % Includes the macro \ifstrequal
%\usepackage{xkeyval} % For enabling labeled macro arguments (see http://www.tex.ac.uk/tex-archive/macros/latex/contrib/xkeyval/doc/xkeyval.pdf)

% Abbreviations
\usepackage{abbrevs_package_with_bug_fixed} % For defining abbreviations and other stuff
\usepackage{xspace} % To prevent incorrect spacing after abbreviations

% Paper size
\iffalse
\usepackage{./Packages/s5paper/s5paper}
\fi

% Referencing
\iffalse
\usepackage[round, authoryear, comma]{natbib} % Harward style referencing
%\bibliographystyle{unsrtnat}
%\bibliographystyle{plain}
\bibliographystyle{plainnat}
\else
\iftrue
\usepackage[square, authoryear, comma]{natbib} % Harward style referencing
%\bibliographystyle{unsrtnat}
%\bibliographystyle{plain}
\bibliographystyle{plainnat}
\else
\usepackage[square, numbers, comma]{natbib} %
\bibliographystyle{ieeetr}
%\bibliographystyle{plain}
%\bibliographystyle{plainnat}
\fi
\fi

% Indexing
\usepackage{makeidx}
\makeindex

% Mathematics
\usepackage{mathtools}
\usepackage{amssymb} % To get \therefore (see documentation: There is no documentation :( )
\usepackage{esint} % Eddie Saudrais integrals (see documentation: http://ctan.uib.no/macros/latex/contrib/esint/esint.pdf).
% For more math symbols, see The Comprehensive LaTeX Symbol List (http://ctan.uib.no/info/symbols/comprehensive/symbols-a4.pdf)
\usepackage{cancel} % To be able to cancel out parts of an equation

% Title page
\usepackage{ifmmaketitle}

% Graphics
\usepackage{graphicx} % Necessary in order to include images
\DeclareGraphicsExtensions{.pdf,.png,.jpg} % Specifies the priority of image formats if there exists more than one image with the same name
%\usepackage{wrapfig} % For allowing figures to be wrapped within the text
\usepackage[footnotesize]{caption} % For captions
\usepackage{subcaption} % For subcaptions
\usepackage{tikz} % For advanced graphics functions
\usepackage{pgfplots}
\usepgfplotslibrary{external} 
\tikzexternalize
\usetikzlibrary{patterns,shapes.multipart,arrows,calc} % For enabling filling with patterns
%\usepackage{pgfmath} % For parsing mathematical data. Is included by the tikz package, but can be used independently for stuff unrelated to graphics, even when the tikz package is not used.
%\usepackage{epstopdf} % For including eps files in a file converted with pdflatex (?)

% Linking
\usepackage{hyperref} % Links to pages. The hyperref should generally be loaded last, with some exceptions
\usepackage{bookmark} % Ease bookmark management

% % % % % %
% MACROS  %
% % % % % %

% Naming conventions
% ------------------
%
% LaTeX does try to encourage a naming scheme:
%
% * Document level commands (\section) lowercase.
% * Package interface commands (\DeclareTextCommandDefault) CamelCase.
% * Package or kernel internal commands (\@text@composite@) lower@case@with@.
% * TeX primitives (\expandafter) lowercase.

% General macros
\newrobustcmd{\comment}[1]{}

% Flags
\newrobustcmd{\newflag}[2]{\newtoggle{#1}\settoggle{#1}{#2}}

% Document macros
\newrobustcmd{\levelname}{%
\ifnumequal{\value{subparagraph}}{0}{%
\ifnumequal{\value{paragraph}}{0}{%
\ifnumequal{\value{subsection}}{0}{%
\ifnumequal{\value{section}}{0}{%
\ifnumequal{\value{chapter}}{0}{%
\ifnumequal{\value{part}}{0}{%
}{part\xspace}%
}{chapter\xspace}%
}{section\xspace}%
}{subsection\xspace}%
}{paragraph\xspace}%
}{subparagraph\xspace}%
}

% Appearance
\newrobustcmd{\HRule}{\rule{\linewidth}{0.5mm}}
\newrobustcmd{\nspace}{\!\!}
\newrobustcmd{\blankpage}{\clearpage\thispagestyle{fancy}\renewcommand{\headrulewidth}{0pt}\chead{}\cfoot{}\IFDEBUGELSE{\begin{center}\large\red{\textit{This page is intentionally left blank}}\end{center}}{\ }\clearpage}
% % % Text color
\definecolor{orange}{rgb}{1,0.5,0}
\definecolor{purple}{rgb}{0.5,0,0.5}
\definecolor{darkpurple}{rgb}{0.31,0,0.31}
\definecolor{darkblue}{rgb}{0,0,0.5}
\definecolor{grey}{rgb}{0.5,0.5,0.5}
\definecolor{darkgrey}{rgb}{0.25,0.25,0.25}
\definecolor{lightgrey}{rgb}{0.75,0.75,0.75}
\newrobustcmd{\red}[1]{\textcolor{red}{#1}}
\newrobustcmd{\yellow}[1]{\textcolor{yellow}{#1}}
\newrobustcmd{\orange}[1]{\textcolor{orange}{#1}}
\newrobustcmd{\green}[1]{\textcolor{green}{#1}}
\newrobustcmd{\cyan}[1]{\textcolor{cyan}{#1}}
\newrobustcmd{\blue}[1]{\textcolor{blue}{#1}}
\newrobustcmd{\darkblue}[1]{\textcolor{darkblue}{#1}}
\newrobustcmd{\magenta}[1]{\textcolor{magenta}{#1}}
\newrobustcmd{\purple}[1]{\textcolor{purple}{#1}}
\newrobustcmd{\darkpurple}[1]{\textcolor{darkpurple}{#1}}
\newrobustcmd{\white}[1]{\textcolor{white}{#1}}
\newrobustcmd{\lightgrey}[1]{\textcolor{lightgrey}{#1}}
\newrobustcmd{\grey}[1]{\textcolor{grey}{#1}}
\newrobustcmd{\darkgrey}[1]{\textcolor{darkgrey}{#1}}
\newrobustcmd{\black}[1]{\textcolor{black}{#1}}
%\newrobustcmd{\transparent}[1]{\textcolor{transparent}{#1}}

% Table of contents
\newrobustcmd{\addtotoc}[2]{\phantomsection\addcontentsline{toc}{#1}{#2}}
\newrobustcmd{\clearaddtotoc}[2]{\cleardoublepage\addtotoc{#1}{#2}}
% Do not add to table of contents and do not number
\newrobustcmd{\notocpart         }[3][]{\pdfbookmark[#1]{#2}{#3}\part*         {#2}}
\newrobustcmd{\notocchapter      }[3][]{\pdfbookmark[#1]{#2}{#3}\chapter*      {#2}}
\newrobustcmd{\notocsection      }[3][]{\pdfbookmark[#1]{#2}{#3}\section*      {#2}}
\newrobustcmd{\notocsubsection   }[3][]{\pdfbookmark[#1]{#2}{#3}\subsection*   {#2}}
\newrobustcmd{\notocsubsubsection}[3][]{\pdfbookmark[#1]{#2}{#3}\subsubsection*{#2}}
\newrobustcmd{\notocparagraph    }[3][]{\pdfbookmark[#1]{#2}{#3}\paragraph*    {#2}}
\newrobustcmd{\notocsubparagraph }[3][]{\pdfbookmark[#1]{#2}{#3}\subparagraph* {#2}}
% Add to table of contents but do not number
\iftrue
\newrobustcmd{\nonumpart         }[1]{\addtotoc{part}         {#1}\part*         {#1}}
\newrobustcmd{\nonumchapter      }[1]{\addtotoc{chapter}      {#1}\chapter*      {#1}}
\newrobustcmd{\nonumsection      }[1]{\addtotoc{section}      {#1}\section*      {#1}}
\newrobustcmd{\nonumsubsection   }[1]{\addtotoc{subsection}   {#1}\subsection*   {#1}}
\newrobustcmd{\nonumsubsubsection}[1]{\addtotoc{subsubsection}{#1}\subsubsection*{#1}}
\newrobustcmd{\nonumparagraph    }[1]{\addtotoc{paragraph}    {#1}\paragraph*    {#1}}
\newrobustcmd{\nonumsubparagraph }[1]{\addtotoc{subparagraph} {#1}\subparagraph* {#1}}
\else
\newrobustcmd{\nonumpart         }[1]{\part*         {#1}}
\newrobustcmd{\nonumchapter      }[1]{\chapter*      {#1}}
\newrobustcmd{\nonumsection      }[1]{\section*      {#1}}
\newrobustcmd{\nonumsubsection   }[1]{\subsection*   {#1}}
\newrobustcmd{\nonumsubsubsection}[1]{\subsubsection*{#1}}
\newrobustcmd{\nonumparagraph    }[1]{\paragraph*    {#1}}
\newrobustcmd{\nonumsubparagraph }[1]{\subparagraph* {#1}}
\fi

% Indexing
\newrobustcmd{\indexify}[1]{\iftoggle{ITALICIZEINDEXEDTEXT}{\textit{\iftoggle{INDEXEDTEXTPURPLE}{\darkpurple{#1}}{#1}}}{\iftoggle{INDEXEDTEXTPURPLE}{\darkpurple{#1}}{#1}}}
\newrobustcmd{\qindexify}[1]{#1} % For indexifying text quietly (don't do anything)

% The idx macro series
% % % Base versions
\newrobustcmd{\Bidx   }[2]{\index{#2}#1{#2}}           % The basic version of the idx macro
\newrobustcmd{\Bidxe  }[3]{\index{#2}#1{#3}}           % The extended version of the idx macro
\newrobustcmd{\Bidxp  }[3]{\index{#2}#1{#2#3}}         % The plural version of the idx macro
\newrobustcmd{\Bidxs  }[3]{\indexs{#2}{#3}#1{#2 #3}}   % The split version of the idx macro
\newrobustcmd{\Bidxse }[4]{\indexs{#2}{#3}#1{#4}}      % The split extended version of the idx macro
\newrobustcmd{\Bidxsp }[4]{\indexs{#2}{#3}#1{#2 #3#4}} % The split plural version of the idx macro
% % % Normal versions
\newrobustcmd{\idx   }[1]{\Bidx{\indexify}{#1}}           % The basic version of the idx macro
\newrobustcmd{\idxe  }[2]{\Bidxe{\indexify}{#1}{#2}}      % The extended version
\newrobustcmd{\idxp  }[2]{\Bidxp{\indexify}{#1}{#2}}      % The plural version
\newrobustcmd{\idxs  }[2]{\Bidxs{\indexify}{#1}{#2}}      % The split version
\newrobustcmd{\idxse }[3]{\Bidxse{\indexify}{#1}{#2}{#3}} % The split extended version
\newrobustcmd{\idxsp }[3]{\Bidxsp{\indexify}{#1}{#2}{#3}} % The split plural version
% % % Quite versions
\newrobustcmd{\qidx   }[1]{\Bidx{\qindexify}{#1}}           % The quite basic version of the idx macro
\newrobustcmd{\qidxe  }[2]{\Bidxe{\qindexify}{#1}{#2}}      % The quite extended version
\newrobustcmd{\qidxp  }[2]{\Bidxp{\qindexify}{#1}{#2}}      % The quite plural version
\newrobustcmd{\qidxs  }[2]{\Bidxs{\qindexify}{#1}{#2}}      % The quite split version
\newrobustcmd{\qidxse }[3]{\Bidxse{\qindexify}{#1}{#2}{#3}} % The quite split extended version
\newrobustcmd{\qidxsp }[3]{\Bidxsp{\qindexify}{#1}{#2}{#3}} % The quite split plural version macro

\newrobustcmd{\indexs}[2]{\iftoggle{REVERTINDEXORDEROFSPLITKEYS}%
{\index{#2!#1}\index{#1 #2|see{#2}}}% Index key has the reeversed order of the parts
{\index{#1 #2}\index{#2!#1|see{#1 #2}}}% Index key has the natural order of the parts
}

% Table of technical abbreviations and acronyms

%\newrobustcmd{\addtototaa}[2]{}
\newrobustcmd{\addtototaa}[2]{\gappto\TableoftaaBody{#1 & #2 \\}}
\let\TableoftaaBody\relax % Or it could just be defined as an empty command
\newrobustcmd{\tableoftaa}[0]{\begin{longtabu}{l X}%
        \textbf{Abbreviation} & \textbf{Full text} \\%
        \hline \\ \endhead%
        \TableoftaaBody%
        \end{longtabu}}

% Easier-to-read indexing
\newrobustcmd{\declareindexkey}     [1]{\expandafter\newabbrev\csname #1\endcsname{\idx{#1}}}
\newrobustcmd{\declareindexkeyq}    [1]{\expandafter\newabbrev\csname #1\endcsname{\qidx{#1}}}
\newrobustcmd{\declareindexkeypair} [2]{\expandafter\newabbrev\csname #2\endcsname{\idxe{#1}{#2}}}
\newrobustcmd{\declareindexkeypairq}[2]{\expandafter\newabbrev\csname #2\endcsname{\qidxe{#1}{#2}}}
%\newrobustcmd{\declareindexkeyq}[1]{\expandafter\newabbrev\csname #1\endcsname{\idxq{#1}}}
% Abbreviations
\newrobustcmd{\declareacronym} [2]{\addtototaa{#1}{#2}\iftoggle{INDEXACRONYMS}%
{\expandafter\newabbrev\csname #1\endcsname{\idxe{#1}{#2}\iftoggle{USEACRONYMS}{ (\qidxe{#2|see{#1}}{#1})}{}}[\iftoggle{USEACRONYMS}{\qidx{#1}}{\qidxe{#1}{#2}}]}%
{\expandafter\newabbrev\csname #1\endcsname{\idx{#2}\iftoggle{USEACRONYMS}{ (\qidxe{#1|see{#2}}{#1})}{}}     [\iftoggle{USEACRONYMS}{\qidxe{#2}{#1}}{\qidx{#2}}]}
}
\newrobustcmd{\declareacronyms}[3]{\addtototaa{#1}{#2 #3}\iftoggle{INDEXACRONYMS}%
{\expandafter\newabbrev\csname #1\endcsname{\idxe{#1}{#2 #3}\index{#3!#2|see{#1}}\index{#2 #3|see{#1}}\iftoggle{USEACRONYMS}{ (#1)}{}}[\iftoggle{USEACRONYMS}{\qidx{#1}}{\qidxe{#1}{#2 #3}}]}%
{\expandafter\newabbrev\csname #1\endcsname{\idxs{#2}{#3}\iftoggle{USEACRONYMS}{ (\qidxe{#1|see{\iftoggle{REVERTINDEXORDEROFSPLITKEYS}{#3}{#2 #3}}}{#1})}{}}[\iftoggle{USEACRONYMS}{\qidxse{#2}{#3}{#1}}{\qidxs{#2}{#3}}]}
}

\newrobustcmd{\declareabbreviation} [3][]{\expandafter\newabbrev\csname #2\endcsname{#1#3}}
\newrobustcmd{\declareabbreviations}[4][]{\expandafter\newabbrev\csname #2\endcsname{#1#3 #4}}
\newrobustcmd{\declareabbreviationi} [3][]{\expandafter\newabbrev\csname #2\endcsname{#1\idx{#3}}}
\newrobustcmd{\declareabbreviationis}[4][]{\expandafter\newabbrev\csname #2\endcsname{#1\idxs{#3}{#4}}}
\newrobustcmd{\declareabbreviationqi} [3][]{\expandafter\newabbrev\csname #2\endcsname{#1\qidx{#3}}}
\newrobustcmd{\declareabbreviationqis}[4][]{\expandafter\newabbrev\csname #2\endcsname{#1\qidxs{#3}{#4}}}
\def\simpleabbrev{\declareabbreviation}
%TODO: Find some better way to denote plural form (one macro would be better since acronyms are indexed verbosely only ocne)
% Allow pluralform of abbreviations
%\newrobustcmd{\s}  {\nspace s\xspace} % For abbreviations that are not indexed
%\newrobustcmd{\qsi}{\nspace\qindexify{s}\xspace} % For abbreviations that are indexed quietly
%\newrobustcmd{\si} {\nspace\indexify{s}\xspace} % For abbreviations that are indexed verbosely

% Mathematics
\newrobustcmd{\opd}[0]{\operatorname{d}\!} % Used e.g. in integrals and derivatives
\newrobustcmd{\normal}[0]{\normvec{n}} % Used e.g. in integrals and derivatives
\newrobustcmd{\range}[2]{#1{--}#2}
% Sets
\newrobustcmd{\naturals}[0]{\mathbb{N}}
\newrobustcmd{\reals}[0]{\mathbb{R}}
\newrobustcmd{\complexes}[0]{\mathbb{C}}

% TikZ
% Two-dimensional versions
\newrobustcmd{\drawplus}[3]{;\draw (#1+#3/2,#2) -- (#1+#3/2,#2+#3); \draw (#1,#2+#3/2) -- (#1+#3,#2+#3/2);}
\newrobustcmd{\squarepath}[1]{-- ++(#1,0) -- ++(0,#1) -- ++(-#1,0) -- cycle}

% Three-dimensional versions
\newrobustcmd{\drawthreedimplus}[4]{\def\cx{#1} \def\cy{#2} \def\cz{#3} \def\s{#4} \drawthreedimplushelper}
\newrobustcmd{\drawthreedimplushelper}[6]{;\draw (\cx+\s*#1/2,\cy+\s*#2/2,\cz+\s*#3/2) -- +(\s*#4,\s*#5,\s*#6); \draw (\cx+\s*#4/2,\cy+\s*#5/2,\cz+\s*#6/2) -- +(\s*#1,\s*#2,\s*#3);}
\newrobustcmd{\threedimsquarepath}[7]{-- ++(#1*#2,#1*#3,#1*#4) -- ++(#1*#5,#1*#6,#1*#7) -- ++(-#1*#2,-#1*#3,-#1*#4) -- cycle}

% DEBUG is defined in options.tex
\newrobustcmd{\IFDEBUG        }[1]{\iftoggle {DEBUG}{#1}{  }}
\newrobustcmd{\IFDEBUGELSE    }[2]{\iftoggle {DEBUG}{#1}{#2}}
\newrobustcmd{\IFNOTDEBUG     }[1]{\nottoggle{DEBUG}{#1}{  }}
\newrobustcmd{\IFNOTDEBUGHELSE}[2]{\nottoggle{DEBUG}{#1}{#2}}

% PAPERPRINT is defined in options.tex
\newrobustcmd{\IFPAPERPRINT        }[1]{\iftoggle {PAPERPRINT}{#1}{  }}
\newrobustcmd{\IFPAPERPRINTELSE    }[2]{\iftoggle {PAPERPRINT}{#1}{#2}}
\newrobustcmd{\IFNOTPAPERPRINT     }[1]{\nottoggle{PAPERPRINT}{#1}{  }}
\newrobustcmd{\IFNOTPAPERPRINTHELSE}[2]{\nottoggle{PAPERPRINT}{#1}{#2}}

% % % % % % % % % % % % % % % % % % % %
% OPTIONS FOR COMPILING THE DOCUMENT  %
% % % % % % % % % % % % % % % % % % % %
% % % % % % % % % % % % % % % % % % % % % % % % % % % % % % % % % % % % % % % % % % % % % % % %
% Options.tex
% -----------
%
% This file contains all options that can be tuned throughout the document. This is the only file that should contain any parameters that can be changed or adjusted in order to tune the document.
% % % % % % % % % % % % % % % % % % % % % % % % % % % % % % % % % % % % % % % % % % % % % % % %


% % % % % % % % % % % % % % % % % % % % % % % % % % % % % % % % % % % % % % % % % % % % % % % %
% Dimensions
% ----------
%
% (see http://nwalsh.com/tex/texhelp/Plain.html#dimensions, http://en.wikipedia.org/wiki/Point_%28typography%29)
%
% pt: Point
% pc: pica (12 pt)
% in: inch (72.27 pt)
% bp: Big point (72 bp = 1 in)
% cm: Centimeter
% mm: Millimeter
% dd: Didot point
% cc: cicero (12 dd)
% sp: Scaled point (65,536 sp = 1 pt), the smallest TeX unit
% ex: Nomimal x-height
% em: Nominal m-width (M-width?)
%
%   Available in math mode:
%
% mu: math unit, 1 em = 18 mu, where em is taken from the math symbols family, various lengths are derived from it (thinspace, thickspace, etc.)
%
%   Additionally available in pdfTeX and LuaTeX:
%
% px: "pixel", the dimension given to the \pdfpxdimen primitive; default value is 1 bp, corresponding to a pixel density of 72 dpi
% % % % % % % % % % % % % % % % % % % % % % % % % % % % % % % % % % % % % % % % % % % % % % % %


% % % % % % % % % % % % % % % % % % % % % % % % % % % % % % % % % % % % % % % % % % % % % % % %
% Spacings in math mode
% ---------------------
%
% \,       thin space (normally 1/6 of a quad)
% \> or \: medium space (normally 2/9 of a quad)
% \;       thick space (normally 5/18 of a quad)
% \!       negative thin space (normally 1/6 of a quad)
%
% TexXBook definition: \def\,{\mskip\thinmuskip} \def\!{\mskip-\thinmuskip}
%
% \thinmuskip would normally be .16667em (= 3 mu), though it might be redefined.
%
% \quad    usually 1em (derived from identities above)
% % % % % % % % % % % % % % % % % % % % % % % % % % % % % % % % % % % % % % % % % % % % % % % %



% % % % % %
% FLAGS   %
% % % % % %

% DEBUG (should be false when publishing or simulating publishing, true otherwise)
% Affects: Extra information written into the document
\newflag{DEBUG}{true} % For debugging the document
%\newflag{DEBUG}{false} % For publishing the document

% PAPERPRINT (should be true if the compiled document is intended to be printed to paper, false otherwise)
% Affects: Link colors
%\newflag{PAPERPRINT}{true} % For debugging the document
\newflag{PAPERPRINT}{false} % For publishing the document

% Miscellaneous flags
\newflag{SKPIPLINEBETWEENPARAGRAPHS} {true}  % Controls whether a line is skipped between paragraphs
\newflag{ITALICIZEINDEXEDTEXT}       {false}  % Controls whether indexed text becomes italicized in the body text
\newflag{USEACRONYMS}                {true}  % Controls whether acronyms should be used or not
\newflag{INDEXACRONYMS}              {true}  % Controls whether the acronyms or the full forms get indexed
\newflag{REVERTINDEXORDEROFSPLITKEYS}{false}  % Controls which is the main index key for split keys
\newflag{SCRIPTSIZEINDEX}            {false}  % Controls the size of the index
\newflag{SCRIPTSIZEBIBLIOGRAPHY}     {false}  % Controls the size of the bibliography

% Debug options
\newflag{INDEXEDTEXTPURPLE}          {\iftoggle{DEBUG}{true}{false}}  % Controls whether indexed text becomes purple in the body text
%\newflag{INDEXEDTEXTPURPLE}          {false}  % Controls whether indexed text becomes purple in the body text
%\newflag{PRINTLABLES}                {\iftoggle{DEBUG}{true}{false}}  % Controls whether indexed text becomes purple in the body text
\newflag{PRINTLABLES}                {false}  % Controls whether indexed text becomes purple in the body text

% % % % % % % %
% APPEARANCE  %
% % % % % % % %

% Names
\renewcommand{\contentsname}{Table of Contents} % Rename Contents to Table of contents

% Names possible to change
%
% \abstractname   Abstract
% \alsoname       see also (makeidx package)
% \appendixname   Appendix
% \bibname        Bibliography (report,book)
% \ccname         cc (letter)
% \chaptername    Chapter (report,book)
% \contentsname   Contents
% \enclname       encl (letter)
% \figurename     Figure (for captions)
% \headtoname     To (letter)
% \indexname      Index
% \listfigurename List of Figures
% \listtablename  List of Tables
% \pagename       Page (letter)
% \partname       Part
% \refname        References (article)
% \seename        see (makeidx package)
% \tablename      Table (for caption)

%\figurename             *\figurename*
%\tablename              *\tablename*
%\partname               *\partname*
%\appendixname           *\appendixname*
%\equationname           *\equationname*
%\Itemname               *\Itemname*
%\chaptername            *\chaptername*
%\sectionname            *\sectionname*
%\subsectionname         *\subsesctionname*
%\subsubsectionname      *\subsubsectionname*
%\paragraphname          *\paragraphname*
%\Hfootnotename          *\Hfootnotename*
%\AMSname                *\AMSname*
%\theoremname            *\theoremname*

% Links
\hypersetup{
    pdfborder = {0 0 0}, % Remove the frame around links
    colorlinks=\iftoggle{PAPERPRINT}{false}{true}, % Don't color links on paper prints
    citecolor=blue, %Used for links to the bibliography
    linkcolor=black, %Used for internal links to labels
    urlcolor=blue, %Used for external links
}

% Header
\setlength{\headheight}{14pt} % To prevent warning " \headheight is too small (12.0pt): Make it at least 14.0pt."

% Maths

% % % Vectors
\robustify{\vec}
%\renewrobustcmd{\vec}[1]{\bar{#1}} % For bars over vectors
%\renewrobustcmd{\vec}[1]{\mathbf{#1}} % For bold font vectors (deosn't work for all characters, for example \pi\pixi)
% % % Normalized vectors
\newrobustcmd{\normvec}[1]{\hat{#1}} % Hats over normalized vectors
%\newrobustcmd{\normvec}[1]{\widehat{#1}} % Wide hats over normalized vectors
% % % Operators
\newrobustcmd{\sop}[1]{\widehat{#1}} % For scalar operators
\newrobustcmd{\vop}[1]{\widehat{\vec{#1}}} % For vector operators
% % % Fourier transform
\newrobustcmd{\fdfunc}[1]{\widetilde{#1}} % For a function in the frequency domain

\iffalse
%% % % Integrals
%%\newrobustcmd{\HalfBetweenIntegralSigns}{\!\!}
%\newrobustcmd{\HalfBetweenIntegralSigns}{\nspace} % \nspace = \!\!
%\newrobustcmd{\BetweenIntegralSigns}{\HalfBetweenIntegralSigns\HalfBetweenIntegralSigns}
%% Double integral
%\robustify{\iint}
%\renewrobustcmd{\iint}{\int\BetweenIntegralSigns\int}
%% Triple integral
%\robustify{\iiint}
%\renewrobustcmd{\iiint}{\int\BetweenIntegralSigns\int\BetweenIntegralSigns\int}
%% Closed double integral
%\newrobustcmd{\oiint}{\begingroup
%    \displaystyle \unitlength 1pt
%    %\let\CharactaristicSize 3pt
%    %\int\mkern-7.2mu
%    \int\HalfBetweenIntegralSigns\mkern-1.2mu
%    \begin{picture}(0,3)
%    %\put(0,3){\oval(10,8)} %\put uses units of \unitlength
%    \put(0,3){\oval(10,8)} %\put uses units of \unitlength
%    \end{picture}
%    %\mkern-7mu\int
%    \HalfBetweenIntegralSigns\mkern-1mu\int
%\endgroup}
\fi

% Referencing
\robustify{\eqref}
\renewrobustcmd{\eqref}    [1]{Equation \ref{#1}}
\newrobustcmd  {\subeqref} [1]{Equation \subref{#1}}
\newrobustcmd  {\eqrefs}   [0]{Equations\xspace}
\newrobustcmd  {\figref}   [1]{Figure \ref{#1}}
\newrobustcmd  {\subfigref}[1]{Figure \subref{#1}}
\newrobustcmd  {\figrefs}  [0]{Figures\xspace}
\newrobustcmd  {\subrefp}  [1]{(\subref{#1})}


% % % % % % % % % % % % % % % % % % % % % % % % % % % % % % % % % % % % % % % % % % % % % % % %
% ABBREVIATIONS AND ACRONYMS
% % % % % % % % % % % % % % % % % % % % % % % % % % % % % % % % % % % % % % % % % % % % % % % %

% % % % % % % % %
% ABBREVIATIONS %
% % % % % % % % %

% The abbreviations are sorted by the abbreviated forms
%\declareabbreviationqi{threedim}      {three-dimensional}
%\declareabbreviationqi{twodim}        {two-dimensional}
%\declareabbreviation  {NS}            {Navier--Stokes}
\declareabbreviation  {itslimitedtime}      {its limited time}
\declareabbreviation  {masterthesisworktime}{about five months}
\declareabbreviationqi{NULL}                {NULL}
\declareabbreviation  {numchildren}         {$2^d$}
\declareabbreviationqi{Saab}                {Saab} % Saab seems to use the Gill Sans font
\declareabbreviation  {thismasterthesiswork}{the master thesis work behind this report}
\declareabbreviation  {thisprojectwork}     {the project work described in this report}

% % % % % % %
% ACRONYMS  %
% % % % % % %

% The acronyms are sorted by the abbreviated forms
\declareacronym {BFECC} {back and forth error compensation and correction}
\declareacronyms{BEM}   {boundary element}{method}
\declareacronyms{CBC}   {convection boundedness}{criterion}
\declareacronyms{CD}    {central}{difference}
\declareacronyms{CFD}   {computational}{fluid dynamics}
\declareacronym {CFL}   {Courant--Friedrichs--Lewy} %[\index{condition!Courant--Friedrichs--Lewy|see{Courant--Friedrichs--Lewy}}]
\declareacronyms{CFMM}  {continuous fast multipole}{method}
\declareacronym {CICSAM}{compressive interface capturing scheme for arbitrary meshes}
\declareacronym {CIP}   {constrained interpolation profile}
\declareacronym {FCSCF} {fast compressive surface capturing formulation}
\declareacronym {FCT}   {flux-corrected transport} % Used in the MULES scheme
\declareacronyms{FMM}   {fast multipole}{method}
\declareacronyms{FOV}   {field of}{view}
\declareacronyms{FVM}   {finite volume}{method}
\declareacronyms{HRIC}  {high resolution}{interface capturing}
\declareacronyms{LOD}   {level of}{detail}
\def            \LODs   {\mbox{\LOD\nspace s}\xspace}
\declareacronym {LS}    {level set}
\declareacronym {LUDS}  {linear upwind difference scheme}
\def            \LUDSs  {\mbox{\LUDS\nspace s}\xspace}
\declareacronym {MAC}   {marker-and-cell}%[\index{method!marker-and-cell|see{marker-and-cell}}]
\declareacronym {MULES} {multidimensional universal limiter with explicit solution}%[\index{method!marker-and-cell|see{marker-and-cell}}]
\declareacronyms{NVD}   {normalised variable}{diagram}
\declareacronym {PCG}   {preconditioned conjugate gradient}
\declareacronyms{PDE}   {partial differential}{equation}
\def            \PDEs   {\mbox{\PDE\nspace s}\xspace}
\declareacronym {SPH}   {smoothed-particle hydrydynamics}
\declareacronym {UQ}    {Quadratic upwind difference scheme}
\declareacronyms{VOF}   {volume of}{fluid}%[\index{method!volume of fluid|see{volume of fluid}}]
\declareacronyms{VOS}   {volume of}{solid}%[\index{method!volume of solid|see{volume of solid}}]
\declareacronym {QUICK} {quadratic upwind interpolation for convective kinematics}

% % % % % % % % % % % % %
% INDEX SINGLE KEYWORDS %
% % % % % % % % % % % % %

\declareindexkey      {accuracy}
\declareindexkey      {advection}
\declareindexkey      {air}
%\declareindexkey      {algorithm}
%\declareindexkeypair      {algorithm}{algorithms}
\declareindexkey      {approximation}%[\index{approximation|seealso{neglection}}]
\declareindexkeypair      {approximation}{approximate}
\declareindexkeypair      {approximation}{approximated}
\declareindexkeypair      {approximation}{approximately}
\declareindexkey      {area}
\declareindexkeypair      {area}{areas}
%\declareindexkey      {array} % Command \array already defined.
\declareindexkey      {average}
\declareindexkeypair      {average}{averaged}
\declareindexkey      {backwash}
\declareindexkey      {camera}
\declareindexkey      {cell}
\declareindexkeypair      {cell}{cells}
\declareindexkey      {compressibility}
\declareindexkeypair      {compressibility}{compressible}
\declareindexkey      {cube}
\declareindexkeypair      {cube}{cubes}
\declareindexkey      {damping}
\declareindexkeypair      {damping}{damp}
%\declareindexkey      {data} % Use \idxs{two-dimensional}{data} or \idxs{three-dimensional}{data} instead
\declareindexkey      {density}
%\declareindexkey      {depth} % Use \idxs{water}{depth} instead
%\declareindexkeypair      {depth}{depths} %Use \idxsp{water}{depth}{s} instead
\declareindexkey      {derivative}
\declareindexkeypair      {derivative}{derivatives}
\declareindexkey      {diffusion}
%\declareindexkey      {dispersion} % Use \idxs{wave}{dispersion} instead
\declareindexkey      {dimension}
%\declareindexkeypair      {dimension}{dimensions} % This keyword is listed under dimensionality
\declareindexkey      {dimensionality}
\declareindexkeypair      {dimensionality}{dimensions}
\declareindexkey      {divergence}
\declareindexkeypair      {divergence}{divergences}
\declareindexkey      {discretization}
\declareindexkeypair      {discretization}{discretize}
\declareindexkeypair      {discretization}{discretized}
\declareindexkey      {estimate}
\declareindexkeypair      {estimate}{estimation}
\declareindexkey      {equilibrium}
\declareindexkey      {flow}
\declareindexkeypair      {flow}{flowing}
\declareindexkeypair      {flow}{flows}
\declareindexkey      {fluid}
\declareindexkeypair      {fluid}{fluids}
\declareindexkey      {frequency}
\declareindexkeypair      {frequency}{frequencies}
\declareindexkey      {gradient}
\declareindexkeypair      {gradient}{gradients}
\declareindexkey      {incompressibility}
\declareindexkeypair      {incompressibility}{incompressible}
\declareindexkey      {infinitesimal}
\declareindexkey      {instability}
\declareindexkeypair      {instability}{unstable}
\declareindexkey      {interpolation}
\declareindexkey      {isosurface}
\declareindexkeypair      {isosurface}{isosurfaces}
\declareindexkey      {jump}
\declareindexkeypair      {jump}{jumps}
\declareindexkey      {method}
\declareindexkeypair      {method}{methods}
\declareindexkey      {momentum}
\declareindexkey      {neglection}%[\index{neglection|seealso{approximation}}]
\declareindexkeypair      {neglection}{neglect}
\declareindexkeypair      {neglection}{neglected}
\declareindexkey      {neighbor}
\declareindexkeypair      {neighbor}{neighboring}
\declareindexkey      {node}
\declareindexkeypair      {node}{nodes}
\declareindexkey      {normalization}
\declareindexkeypair      {normalization}{normalize}
\declareindexkeypair      {normalization}{normalized}
\declareindexkey      {octree}
\declareindexkeypair      {octree}{octrees}
\declareindexkey      {orthogonal}
\declareindexkeypair      {orthogonal}{orthogonalized}
%\declareindexkey      {particle}
%\declareindexkeypair      {particle}{particles}
\declareindexkey      {preformance}
\declareindexkey      {phase}
\declareindexkeypair      {phase}{phases}
\declareindexkey      {pointer}
\declareindexkeypair      {pointer}{pointers}
\declareindexkey      {preconditioning}
\declareindexkeypair      {preconditioning}{preconditioner}
%\declareindexkey      {precision} % Use \idxs{numerical}{precision} or \accuracy instead
\declareindexkey      {pressure}
\declareindexkey      {property}
\declareindexkeypair      {property}{properties}
\declareindexkey      {quadtree}
\declareindexkeypair      {quadtree}{quadtrees}
\declareindexkey      {room}
\declareindexkey      {simulation}
\declareindexkeypair      {simulation}{simulate}
\declareindexkeypair      {simulation}{simulating}
\declareindexkeypair      {simulation}{simulations}
\declareindexkey      {spectrum}
\declareindexkey      {surface}
\declareindexkeypair      {surface}{surfaces}
\declareindexkey      {temperature}
\declareindexkey      {unboundedness}
\declareindexkeypair      {unboundedness}{unbounded}
\declareindexkey      {UPWIND}
\declareindexkey      {vacuum}
\declareindexkey      {velocity}
\declareindexkeypair      {velocity}{velocities}
\declareindexkey      {visualization}
\declareindexkeypair      {visualization}{visualize}
\declareindexkey      {wake}
\declareindexkey      {water}
\declareindexkey      {wavelength}
\declareindexkeypair      {wavelength}{wavelengths}


% % % % % % % % % % % % % %
% FLAG DEPENDENT BEHAVIOR %
% % % % % % % % % % % % % %

\iftoggle{PRINTLABLES}{
    \usepackage{showlabels} % Print the internal labels for various objects in the document
}{}

\iftoggle{SKPIPLINEBETWEENPARAGRAPHS}{
    \usepackage{parskip} % Skips a line between paragraphs instead of indenting the paragraphs
}{}


