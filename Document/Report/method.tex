\part{Method}

\chapter{Finite volume method}

%The \FVM is a way to realistically simulate the \idx{flow} of a \idx{fluid} by dividing the fluid into a large number of non-moving, adjactent \idxp{cell}{s} and letting the fluid flow between the cells, through the \idxp{cell face}{s}. The magnitude of the \idxs{fluid}{flux} between two cells is directly proportional to the area of the cell face between the cells and the component of the velocity of the fluid in the direction of the cell face \idx{normal}.

%The motion of the fluid is described by a \PDE, or a set of \PDE\s.

%When simulating fluids with the \FVM it is common to \idxe{approximation}{approximate} the fluid as being \idxse{incompressible}{fluid}{incompressible}. This is because many numerical methods for solving the discretized \PDE

%This is because it is difficult to simulate compressible fluids when the \idx{time step} becomes large, due to \idxe{instability}{instabilities} in the \idx{numerical method} which leads to \idxs{spurious}{oscillations}.

The \FVM is a way of solving a \PDE, or a set of \PDE\s, where the room is discretized into a large number of non-moving, adjactent volume elements which we will call cells, and different properties are discretized into certain points. Scalar fields are usually discretized to the cell centers, or sometimes to nodes of corners of the cells, as in \citep{Losasso2004}, which can be convenient since interpolations of fields discretized to the cell centers tend to be more difficult. In a collocated grid, all properties are stored at the same locations, so the vector properties are discretized to the same locations as the scalar properties. On a staggered grid on the other hand, the velocity (or the momentum, depending on implementation) is discretized to the cell faces. For the work described in this report, a staggered grid has been used, and we will mention the discretization locations as storage locations for the .

In the \PDE\s, in order to calculate the divergences of various vector fields, the divergence theorem is used and a volume integral of the divergence of the field is converted to a surface integral of the vector field itself. The divergence theorem states that

\begin{equation} \label{eq:divergence_theorem}
\iiint_V\nabla\cdot\vec{F}(\vec{r'})\,\infinitesimal V \,=\, \oiint_S(\vec{F}(\vec{r'})\cdot\vec{n})\,\infinitesimal S
\end{equation}

where $\vec{F}$ is a vector field, $V$ is a control volume, which in our case is the cell surrounding the point $\vec{r}$ in which the divergence is being calculated, $S$ is the surface of the control volume, with normal pointing outwards, $\infinitesimal V$ and $\infinitesimal S$ are infinitesimal elements in $V$ and $S$ respectively, $\vec{n}$ is the normal of $\infinitesimal S$ and $\vec{r'}$ is the position of $\infinitesimal V$ and $\infinitesimal S$ respectively. The divergence of $\vec{F}(\vec{r})$ is then approximated as the average divergence of $\vec{F}$ in $V$, and calculated as

\begin{equation} \label{eq:divergence_surface_integral}
\nabla\vec{F}(\vec{r}) \,=\, \frac{1}{V}\,\oiint_S(\vec{F}\cdot\vec{n})\,\infinitesimal S.
\end{equation}

In the \FVM, the surface of a cell consists of cell faces, $S_i$, between the cell itself and neighboring cells, so \eqref{eq:divergence_surface_integral} can be rewritten as

\begin{equation} \label{eq:divergence_cell_face_sum}
\nabla\vec{F}(\vec{r})\ =\ \frac{1}{V}\,\sum_{S_i} \oiint_{S_i}(\vec{F}\cdot\vec{n})\,\infinitesimal S\ =\ \frac{1}{V}\,\sum_{S_i} F_i\,S_i,
\end{equation}

where $S_i$ is the area of the cell face to the $i$:th neighbor cell, and $F_{\vec{n},\,i}$

and $\vec{F}_i$ is 

Furthermore, each cell face is given a value $\vec{F}_i$ which is used as an approximation for the vector field in the entire cell face.

For orthogonal grids, the gradients of various scalar fields are calculated in a similar way, but in this case the gradient theorem is used. The gradient theorem states that

\begin{equation} \label{eq:gradient_theorem}
\phi(\vec{r}_2)-\phi(\vec{r}_1) \,=\, \int_{\gamma[\vec{r}_1,\,\vec{r}_2]}\nabla\phi(\vec{r})\cdot \infinitesimal\vec{r},
\end{equation}

where $\phi$ is a scalar function and $\gamma[\vec{r}_1,\,\vec{r}_2]$ is a path within $\phi$'s domain, connecting the vectors $\vec{r}_1$ and $\vec{r}_2$.

%In Computational Fluid Dynamics, the finite element method is used to realistically simulate the flow of a fluid by dividing the fluid into a large number of non-moving, adjactent cells and letting the fluid flow between the cells, through the cell faces. The motion of the fluid is described by a set of Partial Differential Equations, usually the Euler equations or the Navier-Stokes equations. The finite volume method works by discretizing the partial differential equations 

%The magnitude of the fluid flux between two cells is directly proportional to the area of the cell face between the cells and the component of the velocity of the fluid in the direction of the cell face normal.

\section{Artificial compression}

\subsection{Instability}

\section{Incompressible Navier-Stokes equations}

\subsection{Iterative methods}

\subsubsection{Gauss-Seidel method}

\subsection{Acceleration of iterative methods}

\subsubsection{Preconditioned conjugate gradient method}

\begin{itemize}
    \item Extension of the gradient descent
\end{itemize}

See also \textit{Incomplete Cholesky Preconditioned Conjugate Gradients method}, described in \textit{\href{http://www.cs.ubc.ca/~rbridson/fluidbook/}{Fluid Simulation for Computer Graphics}}. This method uses the \textit{\href{http://en.wikipedia.org/wiki/Incomplete_Cholesky_factorization}{incomplete Cholesky factorization}} as preconditioner.

\subsubsection{Multigrid method}

See
\begin{itemize}
    \item \textit{\href{http://developer.download.nvidia.com/books/cuda-by-example/cuda-by-example-sample.pdf}{CUDA by Example: An Introduction to General-Purpose GPU Programming}}
    \item \textit{\href{http://people.freebsd.org/~snb/school/hp_mg.pdf}{High Performance Multigrid for Poisson's Equation in 3D}}
\end{itemize}


\section{Semicompressible water}

\section{Boundary conditions}

\chapter{Octrees}

\section{Determining level of detail}

\section{The differentiating problem}

\subsection{Velocity advection term}

\section{Multilevel acceleration}

\section{Wave reflection at level transitions}

\chapter{Free Surface Modelling (FSM)}

There are both surface trackibng methods and surface capturing methods.

See also \textit{\href{http://physbam.stanford.edu/~fedkiw/papers/cam1998-17.pdf}{A Non-Oscillatory Eulerian Approach to Interfaces in Multimaterial Flows (The Ghost Fluid Method)}}

\begin{itemize}
    \item Lecture: \textit{\href{http://www.ims.nus.edu.sg/Programs/fluiddynamic/files/Lecture1-basics.pdf}{Moving Interface Problems: Methods \& Applications Tutorial Lecture I}}
\end{itemize}

\section{Level set method}

\subsection{Marching cubes}

\section{Volume of fluid method}

\begin{itemize}
    \item Reference: \textit{\href{http://pages.csam.montclair.edu/~yecko/icodes/HirtNichols_Surfer_JCP1981.pdf}{Volume of Fluid (VOF) Method for the Dynamics of Free Boundaries}}
    \item Comparsion: \textit{\href{http://capfluidicslit.mme.pdx.edu/reference/Numerics/Gopala_ChemEngJ2008_VOFMethodsFreeSurfaceFlow.pdf}{Volume of fluid methods for immiscible-fluid and free-surface flows}}
\end{itemize}

\subsection{VOF vs. Pseudo VOF}

\begin{itemize}
    \item Explanation: \textit{\href{http://www.flow3d.com/cfd-101/cfd-101-VOF.html}{VOF (Volume of Fluid) - What's in a Name?}}
\end{itemize}

\subsection{Interface reconstruction}
%\section{Internal alpha distribution}

\subsection{Smearing during advection}

\subsection{Geometric advection schemes}

\begin{itemize}
    \item A simple (at least so it seems) scheme: \textit{\href{http://www.lmm.jussieu.fr/~zaleski/nota02.pdf}{A geometrical area-preserving Volume-of-Fluid advection method}}
\end{itemize}

\subsection{Algebraic advection schemes}

\subsubsection{Convection Boundedness Criterion (CBC)}

\sloppy
\begin{itemize}
    \item Reference: \textit{Curvature-compensated Convective Transport: SMART a New Boundedness- Preserving Transport Algorithm}
    \item Extended Convective Boundedness Criterion (ECBC): \textit{Discussion on Numerical Stability and Boundedness of Convective Discretized Scheme}
    \item General Convective Boundedness Criterion (GCBC): \textit{\href{http://gr.xjtu.edu.cn:8080/upload/PUB.1673.4/Wei_NHT.pdf}{A New General Convective Boundedness Criterion}}
    \item Convection Boundedness Criterion for arbitrarily unstructured meshes: \textit{\href{http://powerlab.fsb.hr/ped/kturbo/openfoam/papers/GammaPaper.pdf}{High resolution NVD differencing scheme for arbitrarily unstructured meshes}}
\end{itemize}
\fussy

More:
\begin{itemize}
    \item Normalised Variable Diagram (NVD)
    \item \textit{\href{http://warminski.pollub.plwww.ptmts.org.pl/Waclaw-Koron-2-08.pdf}{Comparison of CICSAM and HRIC High-resolution Schemes for Interface Capturing}}
    \item \textit{\href{http://proceedings.fyper.com/eccomascfd2006/documents/85.pdf}{MODELING OF THE WAVE BREAKING WITH CICSAM AND HRIC HIGH-RESOLUTION SCHEMES}}
\end{itemize}

\subsubsection{Multidimensional Universal Limiter with Explicit Solution (MULES)}

See \textit{OpenFOAM-1.5.x/src/finiteVolume/fvMatrices/solvers/MULES/MULES.H} for details

% Escape characters
%\& \% \$ \# \_ \{ \}
%\textasciitilde  = ~
%\textasciicircum = ^
%\textbackslash   = \

\begin{itemize}
    \item Described here: \textit{\href{http://link.libris.kb.se/sfxliub?sid=?url_ver=Z39.88-2004&rfr_id=info:sid/bibl.liu.se\%3Axerxes+\%28+PubMed+LiU\%29&rft.genre=article&rft_val_fmt=info\%3Aofi\%2Ffmt\%3Akev\%3Amtx\%3Ajournal&rft.issn=15393755&rft.date=2009&rft.jtitle=Phys+Rev+E+Stat+Nonlin+Soft+Matter+Phys&rft.volume=79&rft.issue=3+Pt+2&rft.spage=036306&rft.atitle=Drop+impact+onto+a+liquid+layer+of+finite+thickness+\%3A+dynamics+of+the+cavity+evolution+&rft.aulast=Berberovi\%C4\%87&rft.aufirst=Edin}{Drop impact onto a liquid layer of finite thickness: Dynamics of the cavity evolution}}
    \item An improvement for more than two phases: \textit{\href{http://www.mathematik.uni-ulm.de/numerik/staff/urban/reports/ECCOMASCFD2010paperfinal.pdf}{A Coupled Pressure Based Solution Algorithm Based on the Volume-Of-Fluid Approach for Two or More Immiscible Fluids}}
\end{itemize}

\subsubsection{SOLA-VOF}

\begin{itemize}
    \item Reference: \textit{\href{http://www.ewp.rpi.edu/hartford/~ernesto/Su2012/CFD/Readings/SOLA-VOF-1980-P1.pdf}{SOLA-VOF: A Solution Algorithm for Transient Fluid Flow with Multiple Free Boundaries}}
\end{itemize}

\subsubsection{Hyper-C flux limiter}

\begin{itemize}
    \item Reference: \textit{\href{http://www.water.tkk.fi/wr/kurssit/Yhd-12.112/TVD1.pdf}{The Ultimate Conservative Difference Scheme Applied to Unsteady One-Dimensional Advection}}
\end{itemize}

\paragraph{Floating mixed cells}

\begin{itemize}
    \item Remedy: \textit{\href{https://e-reports-ext.llnl.gov/pdf/245038.pdf}{A Simple Advection Scheme for Material Interface}}
\end{itemize}

\subsubsection{Compressive Interface Capturing Scheme for Arbitrary Meshes (CICSAM)}

\begin{itemize}
    \item Reference: \textit{\href{http://ac.els-cdn.com/S0021999199962769/1-s2.0-S0021999199962769-main.pdf?_tid=85161b57da5f4401e55c9d07495e24ea&acdnat=1336167249_a59e4f578adbacf3bff69936c48cdd57}{A Method for Capturing Sharp Fluid Interfaces on Arbitrary Meshes}}
    \item Also described in (by the same author): \textit{\href{http://powerlab.fsb.hr/ped/kturbo/OpenFOAM/docs/OnnoUbbinkPhD.pdf}{Numerical prediction of two fluid systems with sharp interfaces}}
    \item Test with different Courant numbers: \textit{\href{http://www.marin.nl/upload_mm/8/2/c/1807524470_1999999096_2007-ECCOMAS_HoekstraVazAbeilBunnik.pdf}{Free Surface Flow Modelling with Interface Capturing Techniques}}
    \item Improvement 1: \textit{\href{http://powerlab.fsb.hr/ped/kturbo/openfoam/docs/HenrikRuschePhD2002.pdf}{Computational Fluid Dynamics of Dispersed Two-Phase Flows at High Phase Fractions}}
\end{itemize}

\subsubsection{High Resolution Interface Capturing (HRIC) scheme}

\begin{itemize}
    \item Described here: \textit{\href{http://warminski.pollub.plwww.ptmts.org.pl/Waclaw-Koron-2-08.pdf}{Comparison of CICSAM and HRIC High-resolution Sche\-mes for Interface Capturing}}
\end{itemize}

\subsubsection{Switching Technique for Advection and Capturing of Surfaces scheme (STACS)}

\begin{itemize}
    \item Reference: \textit{\href{http://webfea-lb.fea.aub.edu.lb/cfd/pdfs/publications2/STACS-Complete.pdf}{Convective Schemes for Capturing Interfaces of Free-Surface Flows on Unstructured Grids}}
\end{itemize}

\subsubsection{Inter-Gamma Scheme}

\begin{itemize}
    \item Reference: \textit{\href{http://powerlab.fsb.hr/ped/kturbo/openfoam/docs/InterTrack.pdf}{Interface Tracking Capabilities of the Inter-Gamma Differencing Scheme}}
\end{itemize}

\subsubsection{Constrained Interpolation Profile (CIP) method}

%TODO: Used for advecting fluid interfaces?? At least apparently very good for simple advection.

\begin{itemize}
    \item Reference: \textit{\href{http://www.mech.titech.ac.jp/~ryuutai/paper/JCP2001CIPReviewYabe.pdf}{The Constrained Interpolation Profile Method for Multiphase Analysis}}
\end{itemize}

\subsection{Advection schemes for compressible water}

\begin{itemize}
    \item Remedy: Advect both water volume and total volume and then define alpha as the ration between them
\end{itemize}

\subsubsection{Fast Compressive Surface Capturing Formulation (FCSCF)}

\begin{itemize}
    \item Reference: \textit{\href{http://researchspace.csir.co.za/dspace/bitstream/10204/5282/1/Heyns_2011.pdf}{Free-Surface Modelling Technology for Compressible and Violent Flows}}
\end{itemize}

\section{Coupled Level Set/Volume of Fluid  method}

\begin{itemize}
    \item Reference: \textit{\href{http://pages.csam.montclair.edu/~yecko/icodes/SussmanPuckett_LevelSetVOF.pdf}{A Coupled Level Set and Volume-of-Fluid Method for Computing 3D and Axisymmetric Incompressible Two-Phase Flows}}
\end{itemize}

\chapter{Interaction with moving objects}

see e.g.:

\begin{itemize}
    \item \textit{\href{http://physbam.stanford.edu/~fedkiw/papers/stanford2010-04.pdf}{Numerically Stable Fluid-Structure Interactions Between Compressible Flow and Solid Structures}}
    \item \textit{\href{http://efdl.as.ntu.edu.tw/research/papers/JCP03GCIBM.pdf}{A ghost-cell immersed boundary method for flow in complex geometry}}
    \item \textit{\href{http://www.cs.columbia.edu/~batty/papers/Batty07.pdf}{A Fast Variational Framework for Accurate Solid-Fluid Coupling}} (solid fraction, non-stick to walls)
\end{itemize}

\section{Immersed Boundary Method}

\begin{itemize}
    \item Reference: \textit{\href{http://www4.ncsu.edu/~zhilin/TEACHING/MA798Z/Peskin1.pdf}{The immersed boundary method}}
    \item For compressible flow: \textit{\href{http://www.cecs.wright.edu/~haibo.dong/wp-content/themes/publications/IBM_JCP_2007.pdf}{A sharp interface immersed boundary method for compressible viscous flows}}
\end{itemize}

\section{Volume of Solid Method (VOS)}

\begin{itemize}
    \item Reference: \textit{The simulation of fluid-rigid body interaction}
    \item Described in \textit{Numerical Modeling of Deforming Bubble Transport Related to Cavitating Hydraulic Turbines}
\end{itemize}

\section{Rotation of rigid bodies}

See \textit{\href{http://en.wikipedia.org/wiki/Euler\%27s_equations_\%28rigid_body_dynamics\%29}{Euler's equations (rigid body dynamics)}}

\chapter{Visualization}

\chapter{Conservation laws}

\begin{itemize}
    \item Mass
    \item Energy
    \item Momentum
    \item Angular momentum
    \item (d/dt)(Center of mass) - momentum = 0
\end{itemize}