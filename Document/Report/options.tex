% % % % % % % % % % % % % % % % % % % % % % % % % % % % % % % % % % % % % % % % % % % % % % % % 
% Options.tex
% -----------
%
% This file contains all options that can be tuned throughout the document. This is the only file that should contain any parameters that can be changed or adjusted in order to tune the document.
% % % % % % % % % % % % % % % % % % % % % % % % % % % % % % % % % % % % % % % % % % % % % % % % 


% Dimensions
% ----------
%
% (see http://nwalsh.com/tex/texhelp/Plain.html#dimensions, http://en.wikipedia.org/wiki/Point_%28typography%29)
%
% pt: Point
% pc: pica (12 pt)
% in: inch (72.27 pt)
% bp: Big point (72 bp = 1 in)
% cm: Centimeter
% mm: Millimeter
% dd: Didot point
% cc: cicero (12 dd)
% sp: Scaled point (65,536 sp = 1 pt), the smallest TeX unit
% ex: Nomimal x-height
% em: Nominal m-width (M-width?)
%
%   Available in math mode:
%
% mu: math unit, 1 em = 18 mu, where em is taken from the math symbols family, various lengths are derived from it (thinspace, thickspace, etc.)
%
%   Additionally available in pdfTeX and LuaTeX:
%
% px: "pixel", the dimension given to the \pdfpxdimen primitive; default value is 1 bp, corresponding to a pixel density of 72 dpi



% % % % % %
% FLAGS   %
% % % % % %

% DEBUG
\newflag{DEBUG}{true} % For debugging the document
%\newflag{DEBUG}{false} % For publishing the document

% PAPERPRINT
%\newflag{PAPERPRINT}{true} % For debugging the document
\newflag{PAPERPRINT}{false} % For publishing the document

% DEBUG OPTIONS
% -------------

% % % % % % % %
% APPEARANCE  %
% % % % % % % %

% Links
\hypersetup{
    pdfborder = {0 0 0}, % Remove the frame around links
    colorlinks=\iftoggle{PAPERPRINT}{false}{true}, % Don't color links on paper prints
    citecolor=black, %Used for links to the bibliography
    linkcolor=black, %Used for internal links to labels
    urlcolor=blue, %Used for external links
}

% Header
\setlength{\headheight}{14pt} % To prevent warning " \headheight is too small (12.0pt): Make it at least 14.0pt."

% Maths
% % % Vectors
\robustify{\vec}
%\renewrobustcmd{\vec}[1]{\bar{#1}} % For bars over vectors
%\renewrobustcmd{\vec}[1]{\mathbf{#1}} % For bold font vectors (deosn't work for all characters, for example \pi\pixi)
% % % Operators
\newrobustcmd{\sop}[1]{\widehat{#1}} % For scalar operators
\newrobustcmd{\vop}[1]{\widehat{\vec{#1}}} % For vector operators
% % % Fourier transform
\newrobustcmd{\fdfunc}[1]{\widetilde{#1}} % For a function in the frequency domain

% Indexing
%\newrobustcmd{\indexify}[1]{#1} % For leaving indexed text in the body text as it is
\newrobustcmd{\indexify}[1]{\textit{#1}} % For italicizing the indexed text in the body text

% Referencing
\robustify{\eqref}
\renewrobustcmd{\eqref}[1]{equation \ref{#1}}
