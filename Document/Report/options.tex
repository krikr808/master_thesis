% % % % % % % % % % % % % % % % % % % % % % % % % % % % % % % % % % % % % % % % % % % % % % % %
% Options.tex
% -----------
%
% This file contains all options that can be tuned throughout the document. This is the only file that should contain any parameters that can be changed or adjusted in order to tune the document.
% % % % % % % % % % % % % % % % % % % % % % % % % % % % % % % % % % % % % % % % % % % % % % % %


% % % % % % % % % % % % % % % % % % % % % % % % % % % % % % % % % % % % % % % % % % % % % % % %
% Dimensions
% ----------
%
% (see http://nwalsh.com/tex/texhelp/Plain.html#dimensions, http://en.wikipedia.org/wiki/Point_%28typography%29)
%
% pt: Point
% pc: pica (12 pt)
% in: inch (72.27 pt)
% bp: Big point (72 bp = 1 in)
% cm: Centimeter
% mm: Millimeter
% dd: Didot point
% cc: cicero (12 dd)
% sp: Scaled point (65,536 sp = 1 pt), the smallest TeX unit
% ex: Nomimal x-height
% em: Nominal m-width (M-width?)
%
%   Available in math mode:
%
% mu: math unit, 1 em = 18 mu, where em is taken from the math symbols family, various lengths are derived from it (thinspace, thickspace, etc.)
%
%   Additionally available in pdfTeX and LuaTeX:
%
% px: "pixel", the dimension given to the \pdfpxdimen primitive; default value is 1 bp, corresponding to a pixel density of 72 dpi
% % % % % % % % % % % % % % % % % % % % % % % % % % % % % % % % % % % % % % % % % % % % % % % %


% % % % % % % % % % % % % % % % % % % % % % % % % % % % % % % % % % % % % % % % % % % % % % % %
% Spacings in math mode
% ---------------------
%
% \,       thin space (normally 1/6 of a quad)
% \> or \: medium space (normally 2/9 of a quad)
% \;       thick space (normally 5/18 of a quad)
% \!       negative thin space (normally 1/6 of a quad)
%
% TexXBook definition: \def\,{\mskip\thinmuskip} \def\!{\mskip-\thinmuskip}
%
% \thinmuskip would normally be .16667em (= 3 mu), though it might be redefined.
%
% \quad    usually 1em (derived from identities above)
% % % % % % % % % % % % % % % % % % % % % % % % % % % % % % % % % % % % % % % % % % % % % % % %



% % % % % %
% FLAGS   %
% % % % % %

% DEBUG (should be false when publishing or simulating publishing, true otherwise)
% Affects: Extra information written into the document
\newflag{DEBUG}{true} % For debugging the document
%\newflag{DEBUG}{false} % For publishing the document

% PAPERPRINT (should be true if the compiled document is intended to be printed to paper, false otherwise)
% Affects: Link colors
%\newflag{PAPERPRINT}{true} % For debugging the document
\newflag{PAPERPRINT}{false} % For publishing the document

% Miscellaneous flags
\newflag{SKPIPLINEBETWEENPARAGRAPHS} {true}  % Controls whether a line is skipped between paragraphs
\newflag{ITALICIZEINDEXEDTEXT}       {false}  % Controls whether indexed text becomes italicized in the body text
\newflag{USEACRONYMS}                {true}  % Controls whether acronyms should be used or not
\newflag{INDEXACRONYMS}              {true}  % Controls whether the acronyms or the full forms get indexed
\newflag{REVERTINDEXORDEROFSPLITKEYS}{false}  % Controls which is the main index key for split keys
\newflag{SCRIPTSIZEINDEX}            {false}  % Controls the size of the index
\newflag{SCRIPTSIZEBIBLIOGRAPHY}     {false}  % Controls the size of the bibliography

% Debug options
\newflag{INDEXEDTEXTPURPLE}          {\iftoggle{DEBUG}{true}{false}}  % Controls whether indexed text becomes purple in the body text
%\newflag{INDEXEDTEXTPURPLE}          {false}  % Controls whether indexed text becomes purple in the body text
%\newflag{PRINTLABLES}                {\iftoggle{DEBUG}{true}{false}}  % Controls whether indexed text becomes purple in the body text
%\newflag{PRINTLABLES}                {true}  % Controls whether indexed text becomes purple in the body text
\newflag{PRINTLABLES}                {false}  % Controls whether indexed text becomes purple in the body text

% % % % % % % %
% APPEARANCE  %
% % % % % % % %

% Formatting
\setlength{\extrarowsep}{0.5ex} % Specifies the vertical separation between rows in tabu tables

% Names
\renewcommand{\contentsname}{Table of Contents} % Rename Contents to Table of contents

% Names possible to change
%
% \abstractname   Abstract
% \alsoname       see also (makeidx package)
% \appendixname   Appendix
% \bibname        Bibliography (report,book)
% \ccname         cc (letter)
% \chaptername    Chapter (report,book)
% \contentsname   Contents
% \enclname       encl (letter)
% \figurename     Figure (for captions)
% \headtoname     To (letter)
% \indexname      Index
% \listfigurename List of Figures
% \listtablename  List of Tables
% \pagename       Page (letter)
% \partname       Part
% \refname        References (article)
% \seename        see (makeidx package)
% \tablename      Table (for caption)

%\figurename             *\figurename*
%\tablename              *\tablename*
%\partname               *\partname*
%\appendixname           *\appendixname*
%\equationname           *\equationname*
%\Itemname               *\Itemname*
%\chaptername            *\chaptername*
%\sectionname            *\sectionname*
%\subsectionname         *\subsesctionname*
%\subsubsectionname      *\subsubsectionname*
%\paragraphname          *\paragraphname*
%\Hfootnotename          *\Hfootnotename*
%\AMSname                *\AMSname*
%\theoremname            *\theoremname*

% Links
\hypersetup{
    pdfborder = {0 0 0}, % Remove the frame around links
    colorlinks=\iftoggle{PAPERPRINT}{false}{true}, % Don't color links on paper prints
    citecolor=blue, %Used for links to the bibliography
    linkcolor=black, %Used for internal links to labels
    urlcolor=blue, %Used for external links
}

% Header
\setlength{\headheight}{14pt} % To prevent warning " \headheight is too small (12.0pt): Make it at least 14.0pt."

% Maths

% % % Vectors
\robustify{\vec}
%\renewrobustcmd{\vec}[1]{\bar{#1}} % For bars over vectors
%\renewrobustcmd{\vec}[1]{\mathbf{#1}} % For bold font vectors (deosn't work for all characters, for example \pi\pixi)
% % % Normalized vectors
\newrobustcmd{\normvec}[1]{\hat{#1}} % Hats over normalized vectors
%\newrobustcmd{\normvec}[1]{\widehat{#1}} % Wide hats over normalized vectors
% % % Operators
\newrobustcmd{\sop}[1]{\widehat{#1}} % For scalar operators
\newrobustcmd{\vop}[1]{\widehat{\vec{#1}}} % For vector operators
% % % Fourier transform
\newrobustcmd{\fdfunc}[1]{\widetilde{#1}} % For a function in the frequency domain

\iffalse
%% % % Integrals
%%\newrobustcmd{\HalfBetweenIntegralSigns}{\!\!}
%\newrobustcmd{\HalfBetweenIntegralSigns}{\nspace} % \nspace = \!\!
%\newrobustcmd{\BetweenIntegralSigns}{\HalfBetweenIntegralSigns\HalfBetweenIntegralSigns}
%% Double integral
%\robustify{\iint}
%\renewrobustcmd{\iint}{\int\BetweenIntegralSigns\int}
%% Triple integral
%\robustify{\iiint}
%\renewrobustcmd{\iiint}{\int\BetweenIntegralSigns\int\BetweenIntegralSigns\int}
%% Closed double integral
%\newrobustcmd{\oiint}{\begingroup
%    \displaystyle \unitlength 1pt
%    %\let\CharactaristicSize 3pt
%    %\int\mkern-7.2mu
%    \int\HalfBetweenIntegralSigns\mkern-1.2mu
%    \begin{picture}(0,3)
%    %\put(0,3){\oval(10,8)} %\put uses units of \unitlength
%    \put(0,3){\oval(10,8)} %\put uses units of \unitlength
%    \end{picture}
%    %\mkern-7mu\int
%    \HalfBetweenIntegralSigns\mkern-1mu\int
%\endgroup}
\fi

% Referencing
\robustify{\eqref}
\renewrobustcmd{\eqref}    [1]{Equation \ref{#1}}
\newrobustcmd  {\subeqref} [1]{Equation \subref{#1}}
\newrobustcmd  {\eqrefs}   [0]{Equations\xspace}
\newrobustcmd  {\figref}   [1]{Figure \ref{#1}}
\newrobustcmd  {\subfigref}[1]{Figure \subref{#1}}
\newrobustcmd  {\figrefs}  [0]{Figures\xspace}
\newrobustcmd  {\subrefp}  [1]{(\subref{#1})}


% % % % % % % % % % % % % % % % % % % % % % % % % % % % % % % % % % % % % % % % % % % % % % % %
% ABBREVIATIONS AND ACRONYMS
% % % % % % % % % % % % % % % % % % % % % % % % % % % % % % % % % % % % % % % % % % % % % % % %

% % % % % % % % %
% ABBREVIATIONS %
% % % % % % % % %

% The abbreviations are sorted by the abbreviated forms
%\declareabbreviationqi{threedim}      {three-dimensional}
%\declareabbreviationqi{twodim}        {two-dimensional}
%\declareabbreviation  {NS}            {Navier--Stokes}
%\declareabbreviation  {gammapath}           {{\gamma[\vec{r}_1,\,\vec{r}_2]}}
\newrobustcmd         {\devstressten}          {\boldsymbol{\mathsf{T}}}
\newrobustcmd         {\gammapath}          {{\gamma[\vec{r}_1,\,\vec{r}_2]}}
\declareabbreviation  {itslimitedtime}      {its limited time}
\declareabbreviation  {masterthesisworktime}{about five months}
\declareabbreviationqi{NULL}                {NULL}
\declareabbreviation  {numchildren}         {$2^d$}
\declareabbreviationqi{Saab}                {Saab} % Saab seems to use the Gill Sans font
%\declareabbreviation  {textgammapath}       {\mbox{$\gammapath$}}
\newabbrev            {\textgammapath}      {\mbox{$\gammapath$}}
\declareabbreviationqi{thismasterthesiswork}{the master thesis work behind this report}
\declareabbreviationqi{thisprojectwork}     {the project work described in this report}

% % % % % % %
% ACRONYMS  %
% % % % % % %

% The acronyms are sorted by the abbreviated forms
%\declareacronyms{BEM}   {boundary element}{method}
\declareacronyms{BFECC} {back and forth}{error compensation and correction}
%\declareacronyms{CBC}   {convection boundedness}{criterion}
\declareacronyms{CD}    {central}{difference}
\declareacronyms{CFD}   {computational}{fluid dynamics}
\declareacronym {CFL}   {Courant--Friedrichs--Lewy} %[\index{condition!Courant--Friedrichs--Lewy|see{Courant--Friedrichs--Lewy}}]
\declareacronyms{CFMM}  {continuous fast multipole}{method}
%\declareacronyms{CICSAM}{compressive}{interface capturing scheme for arbitrary meshes}
%\declareacronyms{CIP}   {constrained interpolation}{profile}\addtototaa{}{\textit{or} cubic-interpolated propagation}\index{cubic-interpolated propagation|see{CIP}}\index{propagation!cubic-interpolated|see{CIP}}
\declareacronym {CLSVOF}{coupled level set/volume of fluid}
%\declareacronyms{FCSCF} {fast compressive surface capturing}{formulation}
%\declareacronyms{FCT}   {flux-corrected}{transport} % Used in the MULES scheme
\declareacronyms{FMM}   {fast multipole}{method}
\declareacronyms{FOV}   {field of}{view}
%\declareacronyms{FSM}   {free surface}{modeling}
\declareacronyms{FVM}   {finite volume}{method}
%\declareacronym {hoctree}{hyperoctree}\index{octree!hyper|see(hoctree)}
%\declareacronyms{HRIC}  {high resolution}{interface capturing}
\declareacronyms{LOD}   {level of}{detail}
\def            \LODs   {\mbox{\LOD\nspace s}\xspace}
\declareacronyms{LS}    {level}{set}
\declareacronyms{LUDS}  {linear upwind difference}{scheme}
\def            \LUDSs  {\mbox{\LUDS\nspace s}\xspace}
%\declareacronym {MAC}   {marker-and-cell}%[\index{method!marker-and-cell|see{marker-and-cell}}]
\declareacronyms{MULES} {multidimensional universal}{limiter with explicit solution}
%\declareacronyms{NVD}   {normalised variable}{diagram}
\declareacronyms{PCG}   {preconditioned}{conjugate gradient}
\declareacronyms{PDE}   {partial differential}{equation}
\def            \PDEs   {\mbox{\PDE\nspace s}\xspace}
\declareacronyms{QUICK} {quadratic upwind}{interpolation for convective kinematics}
%\declareacronyms{SIMPLE}{semi-implicit}{method for pressure-linked equations}
\declareacronyms{SPH}   {smoothed-particle}{hydrodynamics}
%                UPWIND
%\declareacronyms{UQ}    {quadratic upwind difference}{scheme}
\declareacronyms{VOF}   {volume of}{fluid}%[\index{method!volume of fluid|see{volume of fluid}}]
%\declareacronyms{VOS}   {volume of}{solid}%[\index{method!volume of solid|see{volume of solid}}]
% If capitalizing certain letters would be relevant in the future, take a look at http://www.tex.ac.uk/cgi-bin/texfaq2html?label=casechange


% % % % % % % % % % % % %
% INDEX SINGLE KEYWORDS %
% % % % % % % % % % % % %

\declareindexkey      {accuracy}
\declareindexkey      {advection}
\declareindexkey      {air}
%\declareindexkey      {algorithm}
%\declareindexkeypair      {algorithm}{algorithms}
\declareindexkey      {approximation}%[\index{approximation|seealso{neglection}}]
\declareindexkeypair      {approximation}{approximate}
\declareindexkeypair      {approximation}{approximated}
\declareindexkeypair      {approximation}{approximately}
\declareindexkey      {area}
\declareindexkeypair      {area}{areas}
%\declareindexkey      {array} % Command \array already defined.
\declareindexkey      {average}
\declareindexkeypair      {average}{averaged}
\declareindexkey      {backwash}
\declareindexkey      {brain}
\declareindexkeypair      {brain}{brains}
\declareindexkey      {camera}
\declareindexkey      {cell}
\declareindexkeypair      {cell}{cells}
\declareindexkey      {compressibility}
\declareindexkeypair      {compressibility}{compressible}
\declareindexkey      {cube}
\declareindexkeypair      {cube}{cubes}
\declareindexkey      {cuboid}
\declareindexkeypair      {cuboid}{cuboids}
\declareindexkey      {damping}
\declareindexkeypair      {damping}{damp}
\declareindexkeypair      {damping}{damped}
%\declareindexkey      {data} % Use \idxs{two-dimensional}{data} or \idxs{three-dimensional}{data} instead
\declareindexkey      {density}
%\declareindexkey      {depth} % Use \idxs{water}{depth} instead
%\declareindexkeypair      {depth}{depths} %Use \idxsp{water}{depth}{s} instead
\declareindexkey      {derivative}
\declareindexkeypair      {derivative}{derivatives}
\declareindexkey      {diffusion}
%\declareindexkey      {dispersion} % Use \idxs{wave}{dispersion} instead
\declareindexkey      {dimension}
%\declareindexkeypair      {dimension}{dimensions} % This keyword is listed under dimensionality
\declareindexkey      {dimensionality}
\declareindexkeypair      {dimensionality}{dimensions}
\declareindexkey      {divergence}
\declareindexkeypair      {divergence}{divergences}
\declareindexkey      {discretization}
\declareindexkeypair      {discretization}{discretize}
\declareindexkeypair      {discretization}{discretized}
\declareindexkey      {estimate}
\declareindexkeypair      {estimate}{estimation}
\declareindexkey      {equilibrium}
\declareindexkey      {flow}
\declareindexkeypair      {flow}{flowing}
\declareindexkeypair      {flow}{flows}
\declareindexkey      {fluid}
\declareindexkeypair      {fluid}{fluids}
\declareindexkey      {frequency}
\declareindexkeypair      {frequency}{frequencies}
\declareindexkey      {gradient}
\declareindexkeypair      {gradient}{gradients}
\declareindexkey      {incompressibility}
\declareindexkeypair      {incompressibility}{incompressible}
\declareindexkey      {infinitesimal}
\declareindexkey      {instability}
\declareindexkeypair      {instability}{unstable}
\declareindexkey      {interpolation}
\declareindexkey      {isosurface}
\declareindexkeypair      {isosurface}{isosurfaces}
\declareindexkey      {jump}
\declareindexkeypair      {jump}{jumps}
\declareindexkey      {method}
\declareindexkeypair      {method}{methods}
\declareindexkey      {momentum}
\declareindexkey      {neglection}%[\index{neglection|seealso{approximation}}]
\declareindexkeypair      {neglection}{neglect}
\declareindexkeypair      {neglection}{neglected}
\declareindexkey      {neighbor}
\declareindexkeypair      {neighbor}{neighboring}
\declareindexkey      {node}
\declareindexkeypair      {node}{nodes}
\declareindexkey      {normalization}
\declareindexkeypair      {normalization}{normalize}
\declareindexkeypair      {normalization}{normalized}
\declareindexkey      {octree}
\declareindexkeypair      {octree}{octrees}
\declareindexkey      {orthogonal}
\declareindexkeypair      {orthogonal}{orthogonalized}
%\declareindexkey      {particle}
%\declareindexkeypair      {particle}{particles}
\declareindexkey      {preformance}
\declareindexkey      {phase}
\declareindexkeypair      {phase}{phases}
\declareindexkey      {pointer}
\declareindexkeypair      {pointer}{pointers}
\declareindexkey      {preconditioning}
\declareindexkeypair      {preconditioning}{preconditioner}
%\declareindexkey      {precision} % Use \idxs{numerical}{precision} or \accuracy instead
\declareindexkey      {pressure}
\declareindexkey      {property}
\declareindexkeypair      {property}{properties}
\declareindexkey      {quadtree}
\declareindexkeypair      {quadtree}{quadtrees}
\declareindexkey      {rectangle}
\declareindexkeypair      {rectangle}{rectangles}
\declareindexkey      {remeshing}
\declareindexkeypair      {remeshing}{remesh}
\declareindexkey      {rendering}
\declareindexkeypair      {rendering}{rendered}
\declareindexkey      {room}
\declareindexkey      {simulation}
\declareindexkeypair      {simulation}{simulate}
\declareindexkeypair      {simulation}{simulating}
\declareindexkeypair      {simulation}{simulations}
\declareindexkey      {spectrum}
\declareindexkey      {surface}
\declareindexkeypair      {surface}{surfaces}
%\declareindexkey      {square} % Command \square already defined.
\declareindexkey      {temperature}
\declareindexkey      {unboundedness}
\declareindexkeypair      {unboundedness}{unbounded}
\declareindexkey      {UPWIND}
\declareindexkey      {vacuum}
\declareindexkey      {velocity}
\declareindexkeypair      {velocity}{velocities}
\declareindexkey      {viscosity}
\declareindexkeypair      {viscosity}{viscosities}
\declareindexkey      {visualization}
\declareindexkeypair      {visualization}{visualize}
\declareindexkeypair      {visualization}{visualizing}
\declareindexkey      {wake}
\declareindexkey      {water}
\declareindexkey      {wavelength}
\declareindexkeypair      {wavelength}{wavelengths}
