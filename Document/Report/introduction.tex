\part{Introduction}

\chapter{Motivation}

\chapter{Difficulties}

\begin{itemize}
    \item Wave dispersion
    \item Different depths gives different speeds
    \item The water should interact with moving objects
\end{itemize}

\chapter{Earlier approaches}

\section{Twodimensional methods}

\subsection{Twodimensional PDEs for shallow water}

\subsection{Spectral methods}

\begin{itemize}
    \item Reference 1: \textit{\href{http://web1.see.asso.fr/ocoss2010/Session_4/20100531111216_Monnier_OCOSS2010-Paper_MERCUDA_item_2.pdf}{Real time modelling of multispectral ocean scenes}}
    \item Reference 2: \textit{GPU-based simulation of Radar sea clutter}
\end{itemize}

\section{Threedimensional methods}

\subsection{Particles}

\subsubsection{Screen Space Meshes}

\begin{itemize}
    \item Reference: \textit{\href{http://www.matthiasmueller.info/publications/screenSpaceMeshes.pdf}{Screen Space Meshes}}
\end{itemize}

\subsection{Marker-and-Cell method (MAC)}

\begin{itemize}
    \item Reference: \textit{Numerical calculation of time-dependent viscous incompressible flow of fluid with a free surface}
    \item Described in: \textit{\href{http://people.sc.fsu.edu/~jburkardt/pdf/fluid_flow_for_the_rest_of_us.pdf}{Fluid Flow for the Rest of Us: Tutorial of the Marker and Cell Method in Computer Graphics}}
\end{itemize}

\subsection{Boundary element method}

\subsection{Finite element method}

\subsection{Finite volume method with tall cells}

\subsection{Finite volume method with octrees}

%This is the method I have choosen to use for my work, with the main reference \cite{Harlow65b}
This is the method I have choosen to use for my work, with the main reference \citealp{LosassoGF04} and \citep{popinet2003}.

\begin{itemize}
    \item Reference 1: \textit{\href{http://gfs.sourceforge.net/gerris.pdf}{Gerris: a tree-based adaptive solver for the incompressible Euler equations in complex geometries}}
    \item Reference 2: \textit{\href{http://physbam.stanford.edu/~fedkiw/papers/stanford2004-02.pdf}{Simulating Water and Smoke with an Octree Data Structure}}
\end{itemize}

\section{Twodimensional and threedimensional hybrid methods}