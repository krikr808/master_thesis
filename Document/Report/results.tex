\chapter{Results}

% Pros:

The program that was developed could successfully simulate water and air and keep the two phases separated, with a region of cells containing a mix of both water and air in between. Pressure waves could be propagates in the bulks of the fluids, and gravity waves propagating on the surface in the interface between the water and the air were successfully be simulated.

% Cons:

The method chosen to simulate water waves in realtime, which was the \FVM on an \octree datastructure together with the \VOF, proved to be quite advanced and difficult to implemented properly within the time frame for \thismasterthesiswork, which was \masterthesisworktime. The simulations suffered from numerous problems for which there wasn't enough time to implement any remedy. Here is a list of issues that the simulation implementation from:

% although there would most likely have been several ways to fix all of them in.

%The limitation of the Courant number is a big problem, as it prevents the simulation from taking as large time steps as it would have to; this could maybe be remedied by switching from explicit to implicit methods. On the other hand, it can now manage to propagate surface waves.

\begin{itemize}
    \item Vacuum could suddenly occur in the air regions, causing a numerical instability that would spread through the entire simulation domain and "eat up" the contents of all other cells.
    
    %\item Vacuum could suddenly occur in the air regions, causing velocities to go to infinity. With constant time step, this lead to numerical instabilities, while with an adaptive time step, this lead to that the simulation went slower and slower the closer the content of a cell came to vacuum and the higher the velocity went, and it would eventually freeze completely. In any case, one of the cells would eventually lose all its content, which would start a numerical instability that would spread like wildfire and "eat up" all the rest of the air and water in the entire simulation domain. This issue did't occur until the simulation had been running for a while and was not a bug, but a result of the way the equations were solved.
    
    \item The simulation speed was extremely low. % Even though the tests carried out during the development of the program were only held in two dimensions, and thus heavily reduced the number of cells to be processed in each time step, the simulation still went many times slower than realt-time speed. This was partly due to the fact that no optimization or paralellization of the code was made, and partly due to the fact that the time step was restricted by the \CFL condition, which forced the time step to be as low as 0.3~ms which corresponds to a \idx{frame rate} of 3333~\FPS, while the frame rate only needed to be at \flightsimulatorfps~\FPS.
    
    \item There were problems with keeping the interface intact. % Unlike linear advection schemes, like \UPWIND, the \idx{Hyper-C} advection scheme, of which a variant was implemented in \thisprojectwork, is supposed to keep the interface compressed. However, the interface was not kept compressed and water was churned up in the air and formed something that closest resembled a kind of mist, which is very non-ideal for several reasons. First, visualization becomes more difficult, since the thicker the interface is the more difficult it is to know where the surface is located. Second, all cells that contain at least some water have to be processed, which means that a thick interface will decrease the simulation speed significantly. Third, a thick interface is a bad \approximation of a real surface interface, whose thickness is almost zero, and numerical errors will be introduced as a result of that.
    
    %It is unknown whether the the fact that the interface grew so large is a property of the Hyper-C advection scheme, if it was because the Hyper-C advection scheme was implemented incorrectly, or just because it is difficult to gereralize these kinds of interface compressing advection schemes to support complessible flows. It is also unknown whether the thickness of the interface would grow arbitrarily large if the simulation would continue to go on without being subject to numerical instability, or if the thickness would eventually stabilize at a certain level.
\end{itemize}

\HRule

\begin{itemize}
    \item Not just the water was simulated, but also the air. % This is a fact that also meant that the simulation domain needed also to have a boundary in the air. This introduced the additional problem of choosing where this boundary should be located, why it should be located at that place, and what boundary conditions it should have. The fact that the air was also simulated meant that a lot of more cells had to be processed each time step than if only the water would have been simulated, which had a significant impact on the simulation speed.
    
    \item The simulation used explicit numerical methods, which meant that it was limited by the \CFL condition. % This meant that the Courant number couldn't exceed one, which limited the time step to a very small value which made it practically impossible to come even near real time speed.
    
    \item Fluid--structure interaction was not implemnted. % This is necessary in order to be able to simulate wakes after ships or in order for waves to make ships rock forth and back.
\end{itemize}