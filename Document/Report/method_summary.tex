\chapter{Method summary}

To make things clear, this is a short summary of all methods that have been used in \thisprojectwork.

The \FVM on a \idxs{stagered}{grid} has been used. The \PDEs that have been solved are the \idxs{Euler}{equations} and the flow has been considered to be \idxse{compressible}{flow}{compressible}.

The grid has been modeled with an \octree. The \LOD of the \idxsp{water surface}{cell}{s}\indexs{surface}{cell} depends on how visially important they are, and the \LOD of the \idxsp{bulk}{cell}{s} decreases the further down under the surface the cells are located. This ensures that the total number $N_t$ of cells used in the simulation is roughly proportional to the number $N_s$ of cells visible on the surface according to \eqref{eq:number_of_cells_total_ordo}.

The interface has been modeled using the \VOF method and the \idxs{advection}{scheme} used for transporting the \idxs{$\alpha$}{field} is a variant of the \idxs{Hyper-C}{flux limiter}\index{limiter!Hyper-C flux limiter|see{Hyper-C flux limiter}} for compressible flow.

The region over the \idxs{water}{surface} is assumed to be \air. Initially, only cells that are at least partly filled with water are added to the octree, in order to save \idxs{computational}{power} (and since there would basically be an infinite number of air cells if they would have been represented). In order for the advection scheme to work properly, all cells with at least some water in them, i.e. $\alpha > 0$, also have to have neighbor cells in all directions the water can be advected, so whenever $\alpha$ for one cell goes from $\alpha = 0$ to $\alpha > 0$, all surfaces of the cell that don't border to another cell or to a \idxs{solid}{boundary} are found and cells with $\alpha = 0$ are created adjacent to those surfaces. This also happens automatically for all cells in the beginning of the simulations when cells initially get filled with water.
