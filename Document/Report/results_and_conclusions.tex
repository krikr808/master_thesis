\part{Results and discussion}

\chapter{Results}

In order for helicopter pilots to get the most out of training landings on ships in a \idxs{flight}{simulator}, the ships have to be affected by the state the water they are in and start to rock forth and back if large surface waves hit them, or the ship will lie still in the water and the landing will be unrealistically easy.

Besides, as an aircraft carrier travels travels with full speed against the wind in order to give the pilot who is supposed to land on it the minimal speed relative to the carrier during the landing, it leaves a wake behind of itself which serves as an aid for the pilot when trying to find the right angle to approach the ship with. Therefore, ships also have to give rise to waves as they travel on the water, which should propagate in a realistic way in order to form the characteristic \idxs{Kelvin}{wake pattern}. Ships should also draw air down into the water in order to give rise to the also so characteristic \backwash that follows a large ship like a tail.

It is therefore necessary to in a realistic way simulate waves on the surface of the water, with a two-way interaction between water and ships. This should happen in realtime in order to make it possible to implement it in a \idxs{flight}{simulator}. Simulation of water waves faces a number of challenges, like \idxs{wave}{dispersion}

the ships have to be able to become affected by surface waves and rock forth and back 

The method chosen to simulate water waves in realtime, which was the \FVM on an \octree datastructure together with the \VOF, proved to be quite advanced and difficult to implemented properly within the time assigned for \thismasterthesiswork, which was \masterthesisworktime. As for the speed, it is not possible to 

For realtime simulation of surface waves in large bodies of water, where non-linear phenomena are not of any significant importance, the best method to use, considering both implementation time and simulation quality, is probably spectral methods.

\section{Speed}

\section{Accuracy}

\chapter{Discussion}

\section{Conclusions}

\subsection{Difficulties and drawbacks with the method}

Difficulties:
\begin{itemize}
    \item Dynamical creation/termination of surface cells and determination of properties in new cells
    \item High Courant numbers
    \item High speeds of sound (remedied in \textit{\href{http://physbam.stanford.edu/~kwatra/papers/compressible_semi_implicit/compressible_semi_implicit.pdf}{A method for avoiding the acoustic time step restriction in compressible flow}})
    \item Keeping a sharp interface
    \item Making it work in realtime
\end{itemize}

\section{Already existing software}

\begin{itemize}
    \item OpenFOAM (\red{FOAM, see} \textit{\href{http://powerlab.fsb.hr/ped/kturbo/openfoam/docs/foam.pdf}{A tensorial approach to computational continuum mechanics using object-oriented techniques}})
\end{itemize}

\chapter{Improvements}

\section{Parallellization}

See e.g. \textit{\href{http://gfs.sourceforge.net/papers/agbaglah2011.pdf}{Parallel simulation of multiphase flows using octree adaptivity and the volume-of-fluid method}}

\subsection{Space filling curves}

See e.g. \textit{\href{http://j.teresco.org/research/publications/octpart02/octpart02.pdf}{Dynamic Octree Load Balancing Using Space-Filling Curves}}

and \textit{\href{http://downloads.isrn.com/journals/appmath/2012/246491.pdf}{Parallel Adaptive Mesh Refinement Combined with Additive Multigrid for the Efficient Solution of the Poisson Equation}}

\section{Local-time stepping}

\section{Remedy for regions with high Courant number}

\section{Perfectly matched layers}

See e.g. \textit{\href{http://liu.diva-portal.org/smash/get/diva2:359805/FULLTEXT01}{Memory Efficient Methods for Eulerian Free Surface Fluid Animation}}

\section{Wind waves}

\subsection{Spectral methods}

\subsection{Air-water interaction}

\section{Visual effects}

\subsection{Splash and foam}

See \textit{\href{http://en.wikipedia.org/wiki/Sea_foam}{Wikipedia -- Sea foam}} or search for \textit{protein skimming} or \textit{foam fractionation}

Implemented in \textit{\href{http://nguyendangbinh.org/Proceedings/Eurographics/2003/cgf/volume22/issue3/paper127/paper127.pdf}{Realistic Animation of Fluid with Splash and Foam}}

\section{Sharpening of various advected fields}

\subsection{Backward Error Compensation and Forward Error Correction}

\begin{itemize}
    \item Reference: \textit{\href{http://smartech.gatech.edu/xmlui/bitstream/handle/1853/29473/2002-389.pdf}{Back and forth error compensation and correction methods for removing errors induced by uneven gradients of the level set function}}
    \item Applied to the velocity field and images: \textit{\href{http://www.gvu.gatech.edu/~jarek/papers/FlowFixer.pdf}{FlowFixer: Using BFECC for Fluid Simulation}}
\end{itemize}

\section{Code optimization}