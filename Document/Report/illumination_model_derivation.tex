%\chapter{Derivation of an illumination model for sea surfaces}
\chapter{An illumination model for sea surfaces}
\label{chap:illumination_model_derivation}

\begin{itemize}
\item Reflection coefficient
\item Transmission term
\item Slope of the surface
\item Chance of shading/hiding
\end{itemize}

\HRule

%To be written \comment
{
If we simplify a water surface by cutting off a part of the \idxs{wave}{spectrum}, such as the part containing all frequencies over the \idxs{Nyquist}{frequency}, we will loose surface details which we --- for \visualization purposes --- need to compensate for by making \reflections in the water surface \idxse{diffuse}{reflection}{diffuse}.

For that reason, an \idxs{illumination}{model} --- which was developed outside of the scope of \thismasterthesiswork\ \,---\ \,is derived in this appendix, and it models this diffuseness.

\section{Microfacets}

Although some of the frequencies has been removed from the rendering, and hence become impossible to treat directly, it is still possible to treat them statistically.

To model the water surface, we will use something known as \microfacets, which are basically infinitesimal surface elements, tiled one after another, for which the \idxs{surface}{normal} is \assumed to be \stochastically distributed with a known \idxs{probability}{distribution} --- we will cover this later. These microfacets are on a much smaller \scale than what is actually represented in the \rendering.

The idea is that if we know the \idxs{probability}{distribution} for all removed frequencies, we can calculate the distribution for the surface normal. This makes it possible to calculate the distribution for the \idxs{reflection}{direction} of a \ray that is \reflected on the water surface, for a certain \idxs{incident}{direction}, which in turn makes it possible to calculate the likelihood that the object of which the reflection can be seen in a specific point on the surface is --- for example --- the \sun.

This means that we can calculate the average brightness of a point on the surface as perceived by the observer. Besides, if we \assume that the removed frequencies correspond to waves that are so short that the wave tips become indistinguishable from one another (it helps if the observer has a bad sight), the average brightness will probably also be a good approximation of the actual brightness perceived by the observer. But before we can start to calculate this value, we need to make a few \assumptions.

First, let's \assume that all frequencies of the wave spectrum are \superposed linearly on top of each other. This implies that there is no \idxs{horizontal}{displacement} --- that is, either the waves are not \idxsp{Gerstner}{wave}{s}, or the \idxs{wave}{amplitude} is so much smaller than the \wavelength that whether the waves are Gerstner waves or not is considered negligible.

Second, to make things simpler, let's \assume that both the fraction of the water surface that is hidden behind other waves, and hence is not visible to the observer, and the fraction of the surface that is \shadowed, i.e. is hidden from the sun behind other waves, are almost zero. In other words, our \microfacets can be assumed not to cover any of the other microfacets, either from the sun or from the observer.

Third, let's also assume that no light will ever be reflected twice on the water surface before reaching the observer.

Now, let's denote the \index{elevation!free surface|see{free surface elevation}}\idxs{free}{surface elevation} of the actual water surface as $\eta^*$, and the free surface elevation of the simplified surface (from which a part of the wave spectrum has been cut off) as $\eta$, and make them functions of $\vec{r}$, which is a \twodimensional vector in the horizontal plane. We can start by noting that $\eta$ is less detailed and contains less information about the actual water surface than $\eta^*$, and that $\eta$ is known, whereas $\eta^*$ is not.

We can therefore only calculate the normal of the simplified surface --- let's call this normal $\normvec{n}$. Let's also define the \idx{anti-normalization} $\anormvec{\xi}$ of a vector $\vec{\xi}$ as

\begin{equation} \label{eq:anti_normal_definition}
\anormvec{\xi} \,=\, \frac{\vec{\xi}}{\normvec{z}\cdot\vec{\xi}}\,,
\end{equation}

where $\normvec{z}$ is the \idxs{unit}{vector} in the positive vertical direction (which is the negation of the direction of the \idxs{gravitational acceleration}{vector}, or the normal of a flat water surface). This implies that the $\normvec{z}$-component of an anti-normalized vector always is 1. Note that the anti-normalization of a vector is only defined if the original vector has a non-zero $\normvec{z}$-component, and that when it is defined, it is parallel to the original vector. The \idx{anti-normal} $\anormvec{n}$ of the surface $\eta$ becomes

\begin{equation} \label{eq:anti_normal_from_gradient}
\anormvec{n} \,=\, \left(\!\!\!\begin{array}{c}-\nabla\eta(\vec{r}) \\ 1\end{array}\!\!\!\right),
\end{equation}

where the last component is the $\normvec{z}$-component. The \gradient in turn can be rewritten as

\begin{equation}
\nabla\eta(\vec{r}) \,=\, \nabla\mathcal{F}^{-1}\{\fdfunc{\eta}(\vec{k})\}(\vec{r}) \,=\, \mathcal{F}^{-1}\{\vec{k}\,\fdfunc{\eta}(\vec{k})\}(\vec{r}),
\end{equation}

where $\mathcal{F}^{-1}$ is the \index{operator!inverse non-unitary Fourier transform|see{inverse non-unitary Fourier transform operator}}\index{non-unitary Fourier transform operator!inverse|see{inverse non-unitary Fourier transform operator}}\idxs{inverse non-unitary}{Fourier transform operator}, $\fdfunc{\eta}(\vec{k})$ is the \idxs{Fourier}{transform} of $\eta(\vec{r})$ and $\vec{k}$ is the \idxs{wave}{vector}. Since $\eta^*$ is unknown, the anti-normal of the real surface, $\anormvec{n}^*$, cannot be calculated, although it can still be written as

\begin{equation} \label{eq:eta_anti_normal}
\anormvec{n}^* \,=\, \left(\!\!\!\begin{array}{c}-\nabla\eta^*(\vec{r}) \\ 1\end{array}\!\!\!\right),
\end{equation}

where the gradient in turn can be written as

\begin{equation}
\nabla\eta^*(\vec{r}) \,=\, \mathcal{F}^{-1}\{\vec{k}\,\fdfunc{\eta}^*(\vec{k})\}(\vec{r}),
\end{equation}

where $\fdfunc{\eta}^*(\vec{k})$ is the \idxs{Fourier}{transform} of $\eta^*$. However, only the part of $\fdfunc{\eta}^*(\vec{k})$ where the wave spectrum has not been cut off is known; in fact, this part is equal to $\fdfunc{\eta}(\vec{k})$, while the reminding part of $\fdfunc{\eta}^*(\vec{k})$ is unknown. On the other hand, if we know what the \idxs{wave}{spectrum} looks like, we can calculate the \idxs{probability}{distribution} of $\eta^*$.

If $\vec{r}$ is assumed to be \idxse{uniform}{distribution}{uniformly distributed} over a large surface area, i.e. there is no \bias in the distribution of $\vec{r}$ regarding to the surface slope $\nabla\eta^*$, or any other local property of the surface for that matter, the distribution of $\fdfunc{\eta}^*$ is easily obtained by looking at the wave spectrum. And from that distribution we will be able to get something that is more valuable to us, namely the distribution of $\nabla\eta^*$.

\subsection{Wave spectra}

When using a \idxs{wave}{spectrum}, the free surface elevation is considered indeterministic, that is, \stochastic. Let's consider $\psi$ to be a completely unknown free surface elevation (as opposed to $\eta^*$, for which the \idxs{frequency}{domain} representation $\fdfunc{\eta}^*$ known in a part of the spectrum but unknown everywhere else). Almost every \idxs{wave}{spectrum}, $S$, used in \idxs{computer}{graphics}, is a function of the wave vector $\vec{k}$ and gives the \variance of the free surface elevation per unit \idxs{reciprocal}{length} squared (the elements of $\vec{k}$ are reciprocal lengths), and thus has the unit $\text{m}^4$. 

If $S$ assumes that the wave components corresponding to the wave vectors are on the exponential form $e^{i(\vec{k}\cdot\vec{r}-\omega t)}$, which is what $\fdfunc{\psi}$ also assumes, rather than on the \sinusoidal form $\sin(\vec{k}\cdot\vec{r}-\omega t)$, which otherwise is a pretty common case in computer graphics, this means that

\begin{equation} \label{eq:wave_spectrum_definition}
S(\vec{k}) \,=\, \frac{\opd \Var(\psi)}{\opd\vec{k}},
\end{equation}

where $\Var$ is the \idxs{variance}{operator} with respect to $\vec{r}$ ($\vec{r}$ is considered to be stochastically distributed), for a fixed time $t$, and $\opd\vec{k}$ is an \infinitesimal \idxs{reciprocal}{area} element, centered around $\vec{k}$. Here, $\opd\Var(\psi)$ denotes the additional variance within $\psi$ caused solely by the part of the wave spectrum specified by $\opd\vec{k}$.

\eqref{eq:wave_spectrum_definition} implies that the additional variance in $\psi$ --- $\Var_{\Delta\vec{k}}(\psi)$ --- caused by a part of the wave spectrum specified by a reciprocal area element $\Delta\vec{k}$ is

\begin{equation} \label{eq:variance_k_eta}
\Var_{\Delta\vec{k}}(\psi) \,=\, \int_{\Delta\vec{k}}\opd \Var(\psi) \,=\, \int_{\Delta\vec{k}}\frac{\opd \Var(\psi)}{\opd\vec{k}}\opd\vec{k} \,=\, \int_{\Delta\vec{k}}S(\vec{k})\opd\vec{k}.
\end{equation}

Since the variance doesn't \correlate wave components of different \wavelengths, we have

\begin{equation} \label{eq:variance_k_eta_to_variance_eta_k}
\Var(\psi_{\Delta\vec{k}}) \,=\, \Var_{\Delta\vec{k}}(\psi),
\end{equation}

where $\psi_{\Delta\vec{k}}$ is defined as

\begin{equation} \label{eq:eta_k_of_fd_eta_k}
\psi_{\Delta\vec{k}} \,=\, \mathcal{F}^{-1}\{\fdfunc{\psi}_{\Delta\vec{k}}(\vec{k})\},
\end{equation}

where in turn

\begin{equation} \label{eq:fd_eta_delta_k_definition}
\fdfunc{\psi}_{\Delta\vec{k}}(\vec{k}) \,=\, \begin{cases}
\fdfunc{\psi}(\vec{k}), & \vec{k} \in \Delta\vec{k} \\
0, & \text{otherwise}
\end{cases}.
\end{equation}

Here, $\fdfunc{\psi}$ is the \idxs{non-unitary}{Fourier transform} of $\psi$, i.e.\ $\fdfunc{\psi}(\vec{k}) = \mathcal{F}\{\psi(\vec{r})\}(\vec{k})$. Besides, the time average of $\fdfunc{\psi}(\vec{k})$ is assumed to be 0 for any given wave vector $\vec{k}$, that is

\begin{equation} \label{eq:zero_expectation_value_reciprocal_space}
\E_{\vec{k}}[\fdfunc{\psi}] \,=\, 0,\ \forall\ \vec{k},
\end{equation}

where $\E_{\vec{k}}$ is the \idxs{expectation value}{operator} with respect to time, for a fixed value of $\vec{k}$, and can be defined as

\begin{equation} \label{eq:expectation_value_reciprocal_space_with_regard_to_t_definition}
\E_{\vec{k}}[\fdfunc{f}] \,=\, \lim_{T\to\infty}\frac{1}{T}\int_0^T \fdfunc{f}(\vec{k},\,t)\opd t,
\end{equation}

where $\fdfunc{f}$ is an arbitrary function of $\vec{k}$ and $t$. After \idxsp{inverse}{Fourier transform}{ing} \eqref{eq:zero_expectation_value_reciprocal_space}, this also results in 

\begin{equation} \label{eq:zero_expectation_value_real_space_with_regard_to_t}
\E_{\vec{r}}[\psi] \,=\, 0,\ \forall\ \vec{r},
\end{equation}

where $\E_{\vec{r}}$ is the \idxs{expectation value}{operator} with respect to time, for a fixed value of $\vec{r}$, and can be defined as

\begin{equation} \label{eq:expectation_value_real_space_with_regard_to_t_definition}
\E_{\vec{r}}[f] \,=\, \lim_{T\to\infty}\frac{1}{T}\int_0^T f(\vec{r},\,t)\opd t,
\end{equation}

where $f$ is an arbitrary function of $\vec{r}$ and $t$.

Let's assume that $\psi$ is periodic in both horizontal dimensions with the period $P$ (this is a commit trick in \idxs{computer}{graphics}) with a large, square surface element $A$ (with \mbox{$|A| = P^2$}) that keeps repeating itself, such that

\begin{equation} \label{eq:water_surface_periodicity}
\psi(\vec{r}+P\normvec{x}) \,=\, \psi(\vec{r}+P\normvec{y}) \,=\, \psi(\vec{r}),\ \forall\ \vec{r},
\end{equation}

where $\normvec{x}$ and $\normvec{y}$ are the \index{vector!orthonormal basis|see{orthonormal basis vector}}\idxsp{orthonormal}{basis vector}{s} generating the \idxs{horizontal}{plane}. This implies that $\fdfunc{\psi}$ is discrete, with \idxsp{Dirac}{impulse}{s} arranged in the intersection points of a square grid with the size $d = 2\pi/P$, such that

\begin{equation} \label{eq:fd_eta_series}
\renewcommand*{\arraystretch}{2}
\begin{array}{c}
\displaystyle \fdfunc{\psi}(\vec{k}) \,=\, 4\pi^2\sum_{m=-\infty}^{+\infty}\sum_{n=-\infty}^{+\infty} \fdfunc{\psi}(m,\,n)\,\delta(\vec{k} - md\normvec{x} - nd\normvec{y}) \\
\displaystyle =\, 4\pi^2\sum_{m=-\infty}^{+\infty}\sum_{n=-\infty}^{+\infty} \fdfunc{\psi}(m,\,n)\,\delta(\vec{k} - \vec{k}_{m,n}),
\end{array}
\end{equation}

where $\fdfunc{\psi}(m,\,n)$ is a \twodimensional \sequence of coefficients associated with $\fdfunc{\psi}$, $m$ and $n$ are two integers used to index the terms in the \series, and $\vec{k}_{m,n} = md\normvec{x} + nd\normvec{y}$.

\comment
{
By \assuming that $\vec{r}$ is \idxse{uniform}{distribution}{uniformly distributed} within $A$, the \idxs{variance}{operator} with respect to $\vec{r}$ can be defined as

\begin{equation} \label{eq:variance_periodic_function_definition}
\Var(f) \,=\, \E_{A,t}[\,|f - \E_{A,t}[f]\,|^2\,],
\end{equation}

where $\E_{A,t}$ denotes the \idxs{expectation value}{operator} with regard to  $\vec{r}$ (which is uniformly distributed within $A$), for a fixed time $t$, and can be defined as

\begin{equation} \label{eq:expectation_value_with_regard_to_r_definition}
\E_{A,t}[f] \,=\, \frac{1}{|A|}\int_A f(\vec{r},\,t)\opd\vec{r}.
\end{equation}

By realizing that this \idxs{expectation}{value} will be time independent for $f(\vec{r}) = \varphi(\vec{r})$, where $\varphi$ is any periodic (obeying \eqref{eq:water_surface_periodicity}), free surface elevation that conserves mass, meaning that the integral over $A$ is time independent, we can use \eqref{eq:zero_expectation_value_real_space_with_regard_to_t} and \eqref{eq:expectation_value_with_regard_to_t_definition} to get

\begin{equation} \label{eq:zero_expectation_value_real_space_with_regard_to_r}
\renewcommand*{\arraystretch}{2}
\begin{array}{c}
\displaystyle \E_{A,t}[\varphi] \,=\, \frac{1}{|A|}\int_A \varphi(\vec{r},\,t)\opd\vec{r} \,=\, \lim_{T\to\infty}\frac{1}{T}\int_0^T\frac{1}{|A|}\int_A \varphi(\vec{r},\,t)\opd\vec{r}\opd t \\
\displaystyle =\, \frac{1}{|A|}\int_A\lim_{T\to\infty}\frac{1}{T}\int_0^T \varphi(\vec{r},\,t)\opd t\opd\vec{r} \,=\, \frac{1}{|A|}\int_A \E_{\vec{r}}[\varphi]\opd\vec{r} \,=\, 0.
\end{array}
\end{equation}

By using using \eqref{eq:expectation_value_with_regard_to_r_definition} and \eqref{eq:zero_expectation_value_real_space_with_regard_to_r}, \eqref{eq:variance_periodic_function_definition} becomes

\begin{equation} \label{eq:variance_integral}
\Var(\varphi) \,=\, \frac{1}{|A|}\int_A|\varphi(\vec{r})|^2\opd\vec{r}.
\end{equation}
}

\eqref{eq:fd_eta_series} can be reduced to

\begin{equation} \label{eq:fd_eta_series_reduced}
\fdfunc{\psi}(\vec{k}) \,=\, \sum_{m=-\infty}^{+\infty}\sum_{n=-\infty}^{+\infty} \fdfunc{\psi}_{m,n}(\vec{k}),
\end{equation}

where

\begin{equation} \label{eq:fd_eta_mn_definition}
\fdfunc{\psi}_{m,n}(\vec{k}) \,=\, 4\pi^2 \fdfunc{\psi}(m,\,n)\,\delta(\vec{k} - \vec{k}_{m,n}),
\end{equation}

and if we define $\Delta\vec{k}_{m,n}$ as a square that has the side $d$, is aligned with $\normvec{x}$ and $\normvec{y}$ and is centered in $\vec{k}_{m,n}$, we see that

\begin{equation} \label{eq:fd_eta_dk_mn_definition}
\fdfunc{\psi}_{\Delta\vec{k}_{m,n}}(\vec{k}) \,=\, \fdfunc{\psi}_{m,n}(\vec{k}).
\end{equation}

It follows naturally to do the same thing with the real space representation, $\psi(\vec{r})$, and write it as

\begin{equation} \label{eq:eta_series_reduced}
\psi(\vec{r}) \,=\, \sum_{m=-\infty}^{+\infty}\sum_{n=-\infty}^{+\infty} \psi_{m,n}(\vec{r}),
\end{equation}

where

\begin{equation} \label{eq:eta_mn_definition}
\renewcommand*{\arraystretch}{1.5}
\begin{array}{c}
\displaystyle \psi_{m,n}(\vec{r}) \,=\, \mathcal{F}^{-1}\{\fdfunc{\psi}_{m,n}(\vec{k})\}(\vec{r}) \,=\, \frac{1}{(2\pi)^2} \int_{\mathbb{R}^2} \fdfunc{\psi}_{m,n}(\vec{k}) e^{i\vec{k}\cdot\vec{r}}\opd \vec{k} \\
\displaystyle =\, \frac{1}{4\pi^2} \int_{\mathbb{R}^2} 4\pi^2\,\fdfunc{\psi}(m,\,n)\,\delta(\vec{k} - md\normvec{x} - nd\normvec{y}) e^{i\vec{k}\cdot\vec{r}}\opd \vec{k} \,=\, \fdfunc{\psi}(m,\,n)\,e^{i\vec{k}_{m,n}\cdot\vec{r}}.
\end{array}
\end{equation}

Taking the gradient of this yields

\begin{equation} \label{eq:grad_eta_one_dirac}
\renewcommand*{\arraystretch}{1.5}
\begin{array}{c}
\displaystyle \nabla\psi_{m,n}(\vec{r}) \,=\, \nabla(\fdfunc{\psi}(m,\,n)\,e^{i\vec{k}_{m,n}\cdot\vec{r}}) \\
\displaystyle =\, \vec{k}_{m,n}\,\fdfunc{\psi}(m,\,n)\,e^{i\vec{k}_{m,n}\cdot\vec{r}} \,=\, \vec{k}_{m,n}\,\psi_{m,n}(\vec{r}).
\end{array}
\end{equation}

Let's \assume that the components of the gradient of the free surface elevation are \idxse{normal}{distribution}{normally distributed}. For vectors, the corresponding distribution is the \textit{\href{http://en.wikipedia.org/wiki/Multivariate\_normal\_distribution\#Non-degenerate\_case}{multivariate normal distribution}} which is a generalization of the normal distribution, so let's assume that $\nabla\psi$ is multivariate normal distributed.

Just like the normal distribution, the multivariate normal distribution requires that the \mean is known, but instead of relying on the variance, it relies on the \idxs{covariance}{matrix} of the vector, which is a generalization of the variance with the only difference that the ordinary \idxs{product}{operator} has been replaced by the \idxs{outer product}{operator} $\otimes$, which is defined as

\begin{equation} \label{eq:outer_product_definition}
\vec{x}\otimes\vec{y} \,=\, \vec{x}\,\vec{y}^{\,\T} \,=\,
\left(\begin{array}{cccc}
x_1    \,y_1 & x_1    \,y_2 & \dots & x_1    \,y_{d_y} \\
x_2    \,y_1 & x_2    \,y_2 & \dots & x_2    \,y_{d_y} \\
\vdots       & \vdots       &       & \vdots           \\
x_{d_x}\,y_1 & x_{d_x}\,y_2 & \dots & x_{d_x}\,y_{d_y}
\end{array}\right),
\end{equation}

where a superscripted $\T$ denotes the \transpose of a matrix (note that vectors can be treated as column matrices), $x_i$ and $y_j$ is the $i$th and $j$th elements of $\vec{x}$ and $\vec{y}$ respectively, and $d_x$ and $d_y$ is the dimensions of $\vec{x}$ and $\vec{y}$ respectively. So, while the variance of a stochastic variable $x$ is defined as

\begin{equation} \label{eq:variance_definition}
Var(x) \,=\, \E[(x - \E[x])^2],
\end{equation}

the covariance matrix, $\Sigma$, of a stochastic vector $\vec{x}$ is defined as

\begin{equation} \label{eq:covariance_matrix_definition}
\Sigma(\vec{x}) \,=\, \E[(\vec{x} - \E[\vec{x}])^{\otimes 2}] \,=\, \E[(\vec{x} - \E[\vec{x}])\otimes(\vec{x} - \E[\vec{x}])].
\end{equation}

Here, a superscripted $\otimes n$ denotes the $n$th \idxs{tensor}{power}, which is simply ordinary $n$th power exponentiation where the ordinary product operator has been replaced by the \idxs{tensor}{product} which for vectors is the same as the outer product.

Isolating the element at row $i$ and in column $j$ in $\Sigma$ --- $\Sigma_{i,j}$ --- yields

\begin{equation} \label{eq:covariance_matrix_element}
\Sigma_{i,j}(\vec{x}) \,=\, \Cov(x_i,\,x_j) \,=\, \E[(x_i - \E[x_i])(x_j - \E[x_j])],
\end{equation}

and combining this with \eqref{eq:grad_eta_one_dirac} gives

\begin{equation} \label{eq:covariance_matrix_gradient_eta_mn}
\renewcommand*{\arraystretch}{2}
\begin{array}{c}
\Sigma(\nabla\psi_{m,n}(\vec{r})) \,=\, \Sigma(\vec{k}_{m,n}\,\psi_{m,n}(\vec{r})) \\
=\,
\left[\renewcommand*{\arraystretch}{1}
\begin{array}{cc}
(md)^2 & mnd^{\,2} \\
mnd^{\,2} & (nd)^2
\end{array}
\right]
\Var(\psi_{m,n}(\vec{r})) \,=\, \vec{k}_{m,n}\otimes\vec{k}_{m,n} \Var(\psi_{m,n}(\vec{r})).
\end{array}
\end{equation}

The covariance has the property

\begin{equation}
\renewcommand*{\arraystretch}{1.5}
\begin{array}{rl}
\Cov(a,\,b)
& =\, \E[(a-\E[a])(b-\E[b])] \\
& =\, \E[ab - a\E[b] - \E[a]b + \E[a]\E[b]] \\
& =\, \E[ab] - \E[a]\E[b] - \E[a]\E[b] + \E[a]\E[b] \\
& =\, \E[ab] - \E[a]\E[b],
\end{array}
\end{equation}

so for two stochastically independent vectors $\vec{x}$ and $\vec{y}$, we have

\begin{equation}
\renewcommand*{\arraystretch}{1.5}
\begin{array}{cl}
  & \Cov(x_i+y_i,\,x_j+y_j) \\
= & \E[(x_i+y_i)(x_j+y_j)] - \E[x_i+y_i]\E[x_j+y_j] \\
= & \E[x_i x_j] + \E[x_i y_j] + \E[y_i x_j] + \E[y_i y_j] \\
  & - \E[x_i]\E[x_j] - \E[x_i]\E[y_j] - \E[y_i]\E[x_j] - \E[y_i]\E[y_j] \\
= & \left/ \vec{x},\, \vec{y}\ \text{stochastically independent}\right/ \\
= & \E[x_i x_j] + \E[x_i]\E[y_j] + \E[y_i]\E[x_j] + \E[y_i y_j] \\
  & - \E[x_i]\E[x_j] - \E[x_i]\E[y_j] - \E[y_i]\E[x_j] - \E[y_i]\E[y_j] \\
= & (\E[x_i x_j] - \E[x_i]\E[x_j]) + (\E[y_i y_j] - \E[y_i]\E[y_j]) \\
= & \Cov(x_i,\,x_j) + \Cov(y_i,\,y_j),
\end{array}
\end{equation}

which implies that

\begin{equation} \label{eq:covariance_matrix_sum}
\Sigma(\vec{x} + \vec{y}) \,=\, \Sigma(\vec{x}) + \Sigma(\vec{y}).
\end{equation}

Combining this with \eqref{eq:eta_series_reduced} and \eqref{eq:covariance_matrix_gradient_eta_mn} gives

\begin{equation} \label{eq:covariance_matrix_gradient_eta}
\renewcommand*{\arraystretch}{2}
\begin{array}{c}
\displaystyle \Sigma(\nabla\psi(\vec{r})) \,=\, \Sigma\left(\nabla\left(\sum_{m=-\infty}^{+\infty}\sum_{n=-\infty}^{+\infty} \psi_{m,n}(\vec{r})\right)\right) \\
\displaystyle =\, \sum_{m=-\infty}^{+\infty}\sum_{n=-\infty}^{+\infty}\Sigma\left(\nabla\psi_{m,n}(\vec{r})\right) \,=\, \sum_{m=-\infty}^{+\infty}\sum_{n=-\infty}^{+\infty}\vec{k}_{m,n}\otimes\vec{k}_{m,n} \Var(\psi_{m,n}(\vec{r})).
\end{array}
\end{equation}

If we combine \eqref{eq:variance_k_eta}, \eqref{eq:variance_k_eta_to_variance_eta_k} and \eqref{eq:fd_eta_dk_mn_definition}, we see that

\begin{equation}
\Var(\psi_{m,n}(\vec{r})) \,=\, \Var(\psi_{\Delta\vec{k}_{m,n}}(\vec{r})) \,=\, \Var_{\Delta\vec{k}_{m,n}}(\psi(\vec{r})) \,=\, \int_{\Delta\vec{k}_{m,n}}S(\vec{k})\opd\vec{k},
\end{equation}

so \eqref{eq:covariance_matrix_gradient_eta} can be rewritten as

\begin{equation}
\Sigma(\nabla\psi(\vec{r})) \,=\, \sum_{m=-\infty}^{+\infty}\sum_{n=-\infty}^{+\infty}\vec{k}_{m,n}\otimes\vec{k}_{m,n} \int_{\Delta\vec{k}_{m,n}}S(\vec{k})\opd\vec{k}
\end{equation}

and since the unit of all reciprocal area elements $\Delta\vec{k}_{m,n}$ is $\mathbb{R}^2$ and since no two of them overlap, in the limit when $P\to\infty$ and $d\to 0$ we get

\begin{equation} \label{eq:covariance_matrix_gradient_eta_final}
\Sigma(\nabla\psi(\vec{r})) \,=\, \int_{\mathbb{R}^2}\vec{k}\otimes\vec{k}\,S(\vec{k})\opd\vec{k}.
\end{equation}

Additionally, by differentiating \eqref{eq:zero_expectation_value_real_space_with_regard_to_t}, we get

\begin{equation} \label{eq:psi_of_r_expectation_value}
\E_{\vec{r}}[\nabla\psi] \,=\, 0,\ \forall\ \vec{r},
\end{equation}

so we know the expectation value of $\nabla\psi$ as well, which means that we know the distribution of $\nabla\psi$ since we assumed it was multivariate normal distributed. The multivariate normal distribution density $f$ of a stochastic vector $\vec{x}$ is given by

\begin{equation} \label{eq:multivariate_normal_distribution_density}
f(\vec{x}) \,=\, \frac{1}{(2\pi)^{d/2}|\Sigma|^{1/2}}\exp\left(-\frac{1}{2}(\vec{x}-\vec{\mu})^{\T}\Sigma^{-1}(\vec{x}-\vec{\mu})\right),
\end{equation}

where $d$ is the \dimensionality of $\vec{x}$, $\vec{\mu} = \E[\vec{x}]$ is the expectation value of $\vec{x}$, $\Sigma = \Sigma(\vec{x})$ is the covariance matrix of $\vec{x}$, $|\Sigma|$ is the \determinant of $\Sigma$, and $\exp$ is the \idxs{exponential}{function}.

The distribution density $f$ of $\nabla\psi$ is therefore given by

\begin{equation} \label{eq:psi_grad_distribution}
f(\nabla\psi(\vec{r})) \,=\, \frac{1}{2\pi|\Sigma|^{1/2}}\exp\left(-\frac{1}{2}(\nabla\psi(\vec{r})-\vec{\mu})^{\T}\Sigma^{-1}(\nabla\psi(\vec{r})-\vec{\mu})\right),
\end{equation}

where $\vec{\mu} = \E[\nabla\psi(\vec{r})]$, and $\Sigma = \Sigma(\nabla\psi(\vec{r}))$ is given by \eqref{eq:covariance_matrix_gradient_eta_final}.

\comment
{
\HRule

Reality condition (for $\eta^*\in\mathbb{R}$):

\begin{equation}
\fdfunc{\eta}^*(-\vec{\xi}\,) \,=\, \overline{\fdfunc{\eta}^*(\vec{\xi}\,)}
\end{equation}

\HRule

If the \idxs{wave}{spectrum} is known, $\Var_{\vec{k}}(\Delta\fdfunc{\eta}^*)$ is also known. Knowing this, we can calculate the mean value $\vec{\mu}$ of $\nabla\eta^*$,

\begin{equation}
\renewcommand*{\arraystretch}{1.5}
\begin{array}{c}
\vec{\mu} \,=\, \E[\nabla\eta^*(\vec{r})] \,=\, E\left[\mathcal{F}^{-1}\left\{\vec{k}\,\fdfunc{\eta}^*(\vec{k})\right\}(\vec{r})\right] \\
=\, E\left[\mathcal{F}^{-1}\left\{\vec{k}\,\left(\fdfunc{\eta}(\vec{k})+\Delta\fdfunc{\eta}^*(\vec{k})\right)\right\}(\vec{r})\right] \\
=\, \mathcal{F}^{-1}\left\{\vec{k}\,\left(\E_{\vec{k}}[\fdfunc{\eta}]+\E_{\vec{k}}[\Delta\fdfunc{\eta}^*]\right)\right\}(\vec{r}) \,=\, \mathcal{F}^{-1}\{\vec{k}\,\fdfunc{\eta}(\vec{k})\}(\vec{r}) \,=\, \nabla\eta(\vec{r})\,.
\end{array}
\end{equation}

\HRule
}

\subsection{Surface normal distribution}

Since the \idxe{surface}{normal}{normal} of the water surface is the normalization of the \idx{anti-normal} $\anormvec{n}$ which is described by \eqref{eq:eta_anti_normal}, we need to know the distribution of $\nabla\eta^*$ in order to know the distribution of the surface normal. However, since we know $\eta$, which has been generated from a \idxs{wave}{spectrum} $S$ for simplified water surfaces, which is a reduced version of the full wave spectrum $S^*$ from which a part has been cut off, \eqref{eq:psi_grad_distribution} cannot be used, since it involves the free surface elevation $\psi$ which is completely unknown. Or more generically, if $\eta$ can be considered to be a \idxse{low-pass}{filter}{(low-pass) filter} version of $\eta^*$, with \idxs{transfer}{function} $H$, such that

\begin{equation}
\fdfunc{\eta}(\vec{k}) \,=\, H(\vec{k})\fdfunc{\eta}^*(\vec{k}),
\end{equation}

we can use \eqrefs \ref{eq:wave_spectrum_definition}--\ref{eq:fd_eta_delta_k_definition} and \eqref{eq:variance_definition} to conclude that

\begin{equation} \label{spectrum_reduced_definition}
\renewcommand*{\arraystretch}{2}
\begin{array}{c}
\displaystyle S(\vec{k}) \,=\, \frac{\opd \Var(\eta)}{\opd\vec{k}} \,=\, \frac{\Var(\eta_{\opd\vec{k}})}{\opd\vec{k}} \,=\, \frac{\E[(\eta_{\opd\vec{k}}-\E[\eta_{\opd\vec{k}}])^2]}{\opd\vec{k}} \\
\displaystyle =\, \frac{\E[(H(\vec{k})\eta^*_{\opd\vec{k}}-\E[H(\vec{k})\eta^*_{\opd\vec{k}}])^2]}{\opd\vec{k}} \,=\, H(\vec{k})^2\frac{\E[(\eta^*_{\opd\vec{k}}-\E[\eta^*_{\opd\vec{k}}])^2]}{\opd\vec{k}} \\
\displaystyle =\, H^2(\vec{k})\frac{\Var(\eta^*_{\opd\vec{k}})}{\opd\vec{k}} \,=\, H^2(\vec{k})S^*(\vec{k}),
\end{array}
\end{equation}

where $\fdfunc{\eta}_{\opd\vec{k}}$ and $\fdfunc{\eta}^*_{\opd\vec{k}}$ are defined by \eqref{eq:fd_eta_delta_k_definition}, where $\Delta\vec{k}$ has been replaced by $\opd\vec{k}$ and where $\fdfunc{\phi}$ has been replaced by $\eta$ and $\eta^*$ respectively.

Since $\Sigma(\vec{x}+\vec{y}) = \Sigma(\vec{x})+\Sigma(\vec{y})$ (according to \eqref{eq:covariance_matrix_sum}) for all stochastically independent vectors $\vec{x}$ and $\vec{y}$, we will \assume that the part of $\eta^*$ that is left out in $\eta$ can advantageously be simulated with a free surface elevation $\eta'$ that has a gradient with the \idxs{covariance}{matrix} $\Sigma(\nabla\eta'(\vec{r})) = \Sigma(\nabla\eta^*(\vec{r})) - \Sigma(\nabla\eta(\vec{r}))$. This matrix can according to  \eqref{eq:covariance_matrix_gradient_eta_final} be rewritten as

\begin{equation}
\renewcommand*{\arraystretch}{2}
\begin{array}{c}
\displaystyle \Sigma(\nabla\eta'(\vec{r})) \,=\, \Sigma(\nabla\eta^*(\vec{r})) - \Sigma(\nabla\eta(\vec{r})) \\
\displaystyle =\, \int_{\mathbb{R}^2}\vec{k}\otimes\vec{k}\,S^*(\vec{k})\opd\vec{k} \,- \int_{\mathbb{R}^2}\vec{k}\otimes\vec{k}\,S(\vec{k})\opd\vec{k} \\
\displaystyle =\, \int_{\mathbb{R}^2}\vec{k}\otimes\vec{k}\,(1-H^2(\vec{k}))S^*(\vec{k})\opd\vec{k} \,=\, \int_{\mathbb{R}^2}\vec{k}\otimes\vec{k}\,S'(\vec{k})\opd\vec{k},
\end{array}
\end{equation}

where

\begin{equation}
S'(\vec{k}) = S^*(\vec{k}) - S(\vec{k}) = (1-H^2(\vec{k}))S^*(\vec{k})
\end{equation}

is the \idxs{wave}{spectrum} of $\eta'$, which is the simulated part of $\eta^*$. The expectation value for $\eta'(\vec{r})$, is according to \eqref{eq:psi_of_r_expectation_value} \assumed to be

\begin{equation}
E[\eta'(\vec{r})] \,=\, 0.
\end{equation}

Since $\eta$ is perfectly known when a scene is rendered, we have $\E[\nabla\eta(\vec{r})] = \nabla\eta(\vec{r})$ and $\Sigma(\nabla\eta(\vec{r})) \,=\, 0_{2,2}$, where $0_{2,2}$ is the $2\times 2$ matrix in which all elements are 0. Knowing this, since $\E[\vec{x}+\vec{y}] = \E[\vec{x}]+\E[\vec{y}]$ for all vectors $\vec{x}$ and $\vec{y}$, we have

\begin{equation} \label{eq:nabla_eta_expectation_value}
\E[\nabla\eta^*(\vec{r})] \,=\, \E[\nabla\eta'(\vec{r})] + \E[\nabla\eta(\vec{r})] \,=\, \E[\nabla\eta(\vec{r})],
\end{equation}

and also

\begin{equation} \label{eq:nabla_eta_covariance_matrix}
\renewcommand*{\arraystretch}{2}
\begin{array}{c}
\displaystyle \Sigma(\nabla\eta^*(\vec{r})) \,=\, \Sigma(\nabla\eta'(\vec{r})) + \Sigma(\nabla\eta(\vec{r})) \,=\, \Sigma(\nabla\eta'(\vec{r})) \\
\displaystyle =\, \int_{\mathbb{R}^2}\vec{k}\otimes\vec{k}\,S'(\vec{k})\opd\vec{k}.
\end{array}
\end{equation}

The distribution of $\nabla\eta^*(\vec{r})$ is then given by replacing $\psi$ by $\eta^*$ \eqref{eq:psi_grad_distribution}, that is

\begin{equation} \label{eq:eta_grad_distribution_r_uniformly_distributed}
f(\nabla\eta^*(\vec{r})) \,=\, \frac{1}{2\pi|\Sigma|^{1/2}}\exp\left(-\frac{1}{2}(\nabla\eta^*(\vec{r})-\vec{\mu})^{\T}\Sigma^{-1}(\nabla\eta^*(\vec{r})-\vec{\mu})\right),
\end{equation}

where $\vec{\mu} = \E[\nabla\eta^*(\vec{r})]$ is given by \eqref{eq:nabla_eta_expectation_value} and $\Sigma = \Sigma(\nabla\eta^*(\vec{r}))$ is given by \eqref{eq:nabla_eta_covariance_matrix}.

\section{The rendering equation}

When doing \idxs{physically based}{rendering}, one often starts from an equation known as the \idxs{rendering}{equation}. This equation is what ties everything that has been covered in this appendix together. It was simultaneously introduced in \idxs{computer}{graphics} in \citep{temp} and \citep{temp}, and is given as

\begin{equation} \label{eq:rendering_equation_original}
\renewcommand*{\arraystretch}{1.5}
\begin{array}{c}
L_{\text{o}}(\vec{r},\,\normvec{\omega}_{\text{o}},\, \lambda,\, t) \,= \\
\displaystyle L_{\text{e}}(\vec{r},\, \normvec{\omega}_{\text{o}},\, \lambda,\, t) \ + \ \int_\Omega \rho'_{\text{r}}(\vec{r},\, \normvec{\omega}_{\text{i}},\, \normvec{\omega}_{\text{o}},\, \lambda,\, t)\, L_{\text{i}}(\vec{r},\, \normvec{\omega}_{\text{i}},\, \lambda,\, t)\, (\normvec{\omega}_{\text{i}}\,\cdot\,\normvec{n})\, \opd \normvec{\omega}_{\text{i}},
\end{array}
\end{equation}

where $L_{\text{o}}$ is the total \idxs{spectral}{radiance} (or power per unit solid-angle-in-the-direction-of-a-ray and unit projected-area-perpendicular-to-the-ray and unit wavelength) of \index{light!wavelength}\idxe{wavelength!light}{wavelength} $\lambda$ directed outward along direction $\normvec{\omega}_{\text{o}}$ at time $t$, from location $\vec{r}$ at the surface; $L_{\text{e}}$ is the spectral radiance emitted by the surface itself, $\rho'_{\text{r}}$ is the \BRDF which was first defined in \citep{temp} and describes how light from different directions are reflected on the surface, $L_{\text{i}}$ is the \idxs{spectral}{radiance} incoming from direction $\normvec{\omega}_{\text{i}}$, $\normvec{n}$ is the \idxs{surface}{normal}, and $\Omega$ is the \idxs{unit}{hemisphere} containing all possible directions for the incoming reflected light, and thus all possible values for $\normvec{\omega}_{\text{i}}$ (but also all possible values for $\normvec{\omega}_{\text{o}}$). Note that both $\normvec{\omega}_{\text{i}}$ and $\normvec{\omega}_{\text{o}}$ are directed outwards from the surface.

While this equation can lay the foundation for many very realistic renderings of \threedimensional scenes, and can make the reflective properties of a surface vary for different positions, for different points in time, for different wavelengths, which allows the surface to have a color and not just a gray-scale, and even for different rotations of the surface, there are several aspects of light it can't grasp. Some of these aspects, which are relevant to \surfacewaterrendering, include

\begin{itemize}
\item \textbf{\idxe{polarisation}{Polarization}:} Light polarized differently will sometimes have different reflection distributions, as in the case of light being reflected at a water surface.

\item \textbf{\idxe{transmission}{Transmission}:} Occurs when light is transmitted through the surface, as when it hits a glass object or a water surface.

\item \textbf{\idxse{subsurface}{scattering}{Subsurface scattering}:} Many materials exhibits the property that much of the incoming light is transmitted through the surface at one location, scattered, and transmitted back through the surface at a slightly different location. If such a material is rendered without taking subsurface scattering into account, it may appear plastic, and sometimes also unnaturally opaque.

However, it is not necessary to account specifically for this in the rendering equation if it includes transmission, since that will effectively also include light scattered under the surface, even if the rendering equation still doesn't provide a model for how the light is scattering under the surface.
\end{itemize}

Of these aspects, polarization and transmission are the two aspects that are the easiest to model. Subsurface scattering also plays a role, though, but it is not going to be modeled in this appendix.

Other aspects of light, which are not relevant to \surfacewaterrendering (but are still included in this report for leisure reading), include

\begin{itemize}
\item \textbf{\idxe{phosphorescence}{Phosphorescence}:} Light or other electromagnetic radiation is sometimes absorbed at one point in time and emitted at a later point in time, usually with a lower frequency (unless the absorbed electromagnetic radiation is very intense).

If the absorption and the emission occurs at the same point in time, but with different frequencies, this is called \idx{fluorescence}.
    
\item \textbf{\idxe{interference}{Interference}:} This can occur if the wave properties of light are exhibited, for example when light is passing though a \idxs{thin}{slit} or a \idxs{double}{slit}.
    
\item \textbf{\idxse{non-linear}{effect}{Non-linear effects}:} If the light is very intensive, two or more photons can sometimes hit the same electron in a material at the same time, increasing the energy of the electron with more than the energy of the individual photons. When the electron makes a transition back to a lower energy level, emission of a photon with a higher frequency is possible.
    
\item \textbf{\index{\idxse{effect!relativistic Doppler|see{relativistic Doppler effect}}}\idxse{relativistic}{Doppler effect}{Relativistic Doppler effect}:} Light that is reflected on an object that is moving with a very high speed relative to the reference frame (or to something that is observing the light) will get its wavelength changed. If the light is reflected on an object that is moving towards it, the impact will compress the photons, making the wavelength shorter which in turn makes the light blueshifted. The photons will also be packed more closely, so the photon flux will be increased. If the light instead is reflected on an object that is moving away from it, the opposite thing will happen.
\end{itemize}

To account for transmission and polarization, we will modify \eqref{eq:rendering_equation_original} into

\begin{equation} \label{eq:rendering_equation_improved}
\renewcommand*{\arraystretch}{1.5}
\begin{array}{cl}
L_{x,\text{o}}(\vec{r},\,\normvec{\omega}_{\text{o}},\, \lambda,\, t) \,=\, L_{x,\text{e}}(\vec{r},\, \normvec{\omega}_{\text{o}},\, \lambda,\, t) \ + \\
\displaystyle \int_{\Omega_{\text{r}}} \rho'_{x,\text{r}}(\vec{r},\, \normvec{\omega}_{\text{i}},\, \normvec{\omega}_{\text{o}},\, \lambda,\, t)\, L_{x,\text{i}}(\vec{r},\, \normvec{\omega}_{\text{i}},\, \lambda,\, t)\, (\normvec{\omega}_{\text{i}}\,\cdot\,\normvec{n})\, \opd \normvec{\omega}_{\text{i}} \ + \\
\displaystyle \int_{\Omega_{\text{t}}} \rho'_{x,\text{t}}(\vec{r},\, \normvec{\omega}_{\text{i}},\, \normvec{\omega}_{\text{o}},\, \lambda,\, t)\, L_{x,\text{i}}(\vec{r},\, \normvec{\omega}_{\text{i}},\, \lambda,\, t)\, (\normvec{\omega}_{\text{i}}\,\cdot\,(-\normvec{n}))\, \opd \normvec{\omega}_{\text{i}} & \!\!\!\! ,
\end{array}
\end{equation}

where $x$ is the light polarization, which is either s or p, depending on if the light is s or p polarized, $\Omega_{\text{r}}$ is the \idxs{unit}{hemisphere} containing all possible directions for the incoming reflected light, $\Omega_{\text{t}}$ is the \idxs{unit}{hemisphere} containing all possible directions for the incoming transmitted light (which is also the \idxs{relative}{complement} of $\Omega_{\text{r}}$ in the \idxs{unit}{sphere}, i.e. all points in the unit sphere that are not contained in $\Omega_{\text{r}}$), and $\rho'_{\text{t}}$ is the \BTDF which describes how light from different directions are transmitted through the surface.

The parameters $x$, $\vec{r}$, $t$ and $\lambda$ can be removed from the equation since they are of no importance to the derivation of the illumination model; one will just have to keep in mind that the functions depend on $x$, $\vec{r}$, $t$ and $\lambda$. Even the term $L_{x, \text{e}}$ can safely be removed since water surfaces is generally not considered to emit any electromagnetic radiation. We can therefore simplify \eqref{eq:rendering_equation_improved} to

\begin{equation} \label{eq:rendering_equation_reduced}
\renewcommand*{\arraystretch}{1.5}
\begin{array}{cl}
L_{\text{o}}(\normvec{\omega}_{\text{o}}) \,= \\
\displaystyle \int_{\Omega_{\text{r}}} \rho'_{\text{r}}(\normvec{\omega}_{\text{i}},\, \normvec{\omega}_{\text{o}})\, L_{\text{i}}(\normvec{\omega}_{\text{i}})\, (\normvec{\omega}_{\text{i}}\,\cdot\,\normvec{n})\, \opd \normvec{\omega}_{\text{i}} \ + \\[1ex]
\displaystyle \int_{\Omega_{\text{t}}} \rho'_{\text{t}}(\normvec{\omega}_{\text{i}},\, \normvec{\omega}_{\text{o}})\, L_{\text{i}}(\normvec{\omega}_{\text{i}})\, (\normvec{\omega}_{\text{i}}\,\cdot\,(-\normvec{n}))\, \opd \normvec{\omega}_{\text{i}} & \!\!\!\! .
\end{array}
\end{equation}

Here, $\rho'_{\text{r}}(\normvec{\omega}_{\text{i}},\, \normvec{\omega}_{\text{o}})$ is defined as

\begin{equation} \label{eq:brdf_definition}
\rho'_{\text{r}}(\normvec{\omega}_{\text{i}},\, \normvec{\omega}_{\text{o}}) \,=\, \frac{\opd L_{\text{o}}(\normvec{\omega}_{\text{o}})}{\opd E_{\text{i}}(\normvec{\omega}_{\text{i}})} \,=\, \frac{\opd L_{\text{o}}(\normvec{\omega}_{\text{o}})}{\opd \Phi_{\text{i}}(\normvec{\omega}_{\text{i}})A^{-1}},
\end{equation}

where $E_{\text{i}}$ is the total spectral irradiance (or \idxs{incident}{power} per unit surface area and unit wavelength) of \index{light!wavelength}\idxe{wavelength!light}{wavelength} $\lambda$, coming from direction $\normvec{\omega}_{\text{i}}$, and $\Phi_{\text{i}}$ is the total spectral radiant flux (or \idxs{incident}{power} per unit wavelength) of \index{light!wavelength}\idxe{wavelength!light}{wavelength} $\lambda$, coming from direction $\normvec{\omega}_{\text{i}}$, and $A$ is the area of the surface element that is being illuminated. Note that when determining $A$, the surface is seen as perfectly flat, ignoring the microfacets.

Here, $\opd L_{\text{o}}$ is the \infinitesimal increase in $L_{\text{o}}$, caused by $\opd E_{\text{i}}$ which is the \infinitesimal increase in $E_{\text{i}}$, or caused by $\opd \Phi_{\text{i}}$ which is the \infinitesimal increase in $\Phi_{\text{i}}$. \comment{, which in turn is caused by $\opd\normvec{\omega}_{\text{i}}$, which can be thought of as an infinitesimal increase in the solid angle of some light intake from which light can reach the surface}

Since \radiance is power per unit projected-area-perpendicular-to-a-ray and unit solid angle, we can rewrite $L_{\text{o}}$ as

\begin{equation} \label{eq:outgoing_radiance_definition}
L_{\text{o}} \,=\, \frac{\opd \Phi_{\text{o}}(\normvec{\omega}_{\text{o}})(A\cos\theta_{\text{o}})^{-1}}{\opd \normvec{\omega}_{\text{o}}},
\end{equation}

where $\Phi_{\text{o}}$ is the total reflected spectral power (or \idxs{reflected}{power} per unit wavelength), and $\theta_{\text{o}}$ is the angle between $\normvec{\omega}_{\text{o}}$ and the surface normal $\normvec{n}$ which can be written as

\begin{equation} \label{eq:normal_from_antinormal}
\normvec{n} \,=\, \frac{\anormvec{n}}{|\anormvec{n}|},
\end{equation}

where $\anormvec{n}$ is given by \eqref{eq:anti_normal_from_gradient}. In \eqref{eq:outgoing_radiance_definition}, $\opd \Phi_{\text{o}}$ is the infinitesimal increase in $\Phi_{\text{o}}$, caused by $\opd \normvec{\omega}_{\text{o}}$, which can be thought of as an infinitesimal increase in the solid angle of some light outlet through which the spectral power of the reflected light is measured. By using \eqref{eq:anti_normal_from_gradient}, we can rewrite \eqref{eq:brdf_definition} as

\begin{equation} \label{eq:brdf_rewritten}
\renewcommand*{\arraystretch}{2}
\begin{array}{c}
\displaystyle \rho'_{\text{r}}(\normvec{\omega}_{\text{i}},\, \normvec{\omega}_{\text{o}}) \,=\, \frac{\opd^2 \Phi_{\text{o}}(\normvec{\omega}_{\text{o}})(A\cos\theta_{\text{o}})^{-1}}{\opd \normvec{\omega}_{\text{o}}\opd \Phi_{\text{i}}(\normvec{\omega}_{\text{i}})A^{-1}} \,=\, \frac{\opd^2 \Phi_{\text{o}}(\normvec{\omega}_{\text{o}})}{\opd \normvec{\omega}_{\text{o}}\opd \Phi_{\text{i}}(\normvec{\omega}_{\text{i}})\cos\theta_{\text{o}}} \\
\displaystyle =\, (\cos\theta_{\text{o}})^{-1}\, \frac{\opd}{\opd \normvec{\omega}_{\text{o}}}\left(\frac{\opd \Phi_{\text{o}}(\normvec{\omega}_{\text{o}})}{\opd \Phi_{\text{i}}(\normvec{\omega}_{\text{i}})}\right),
\end{array}
\end{equation}

which is basically the distribution density of the reflection direction $\normvec{\omega}_{\text{o}}$ for the reflection of a ray that comes from the direction $\normvec{\omega}_{\text{i}}$, divided by $\cos\theta_{\text{o}}$.

But how will we know the distribution of the reflection direction for a ray that is being reflected? Well, we know that in order for a ray that comes from the direction $\normvec{\omega}_{\text{i}}$ to be reflected in the direction $\normvec{\omega}_{\text{o}}$, the surface has to be angeled in a certain way. Here, the surface we are talking about is the actual surface, or the surface of the \microfacet the ray is hitting, which is described by $\eta^*$, rather than the simplified surface which is described by $\eta$.

The equation that describes the reflection is given by the \index{reflection!Householder|see{Householder transformation}}\idxs{Householder}{transformation}, which states that the reflection $\vec{r}'$ of a point $\vec{r}$ in a plane with the surface normal $\normvec{v}$, is given by

\begin{equation}
\vec{r}' \,=\, \vec{r} - 2(\normvec{v}^{\Hermitian}\vec{r})\normvec{v},
\end{equation}

where a superscripted $\Hermitian$ denotes the \idxs{Hermitian}{transpose} operator, which performs the normal transpose and conjugates the elements. The right hand side of this equation is referred to as the Householder transformation of $\vec{r}$. For ordinary surface normals, which has real-valued elements, this becomes

\begin{equation}
\vec{r}' \,=\, \vec{r} - 2(\normvec{v}\cdot\vec{r})\normvec{v}.
\end{equation}

In our case, we know that the reflection direction $\normvec{\omega}_{\text{o}}$, is the reflection of the incident direction, or the reflection of the opposite of the direction $\normvec{\omega}_{\text{i}}$ from which the incident ray came. Let's also refer to the surface normal the \microfacet would need to have in order for a ray that comes from the direction $\normvec{\omega}_{\text{i}}$ to be reflected in the direction $\normvec{\omega}_{\text{o}}$ as $\normvec{h}$. This will give us

\begin{equation} \label{eq:householder_equation}
\normvec{\omega}_{\text{o}} \,=\, 2(\normvec{h}\cdot\normvec{\omega}_{\text{i}})\normvec{h} - \normvec{\omega}_{\text{i}}.
\end{equation}

Let's find what value $\normvec{h}$ must have in order for this equation to be satisfied. We can start by rewriting it as

\begin{equation}
\normvec{h} \,=\, \frac{1}{2(\normvec{h}\cdot\normvec{\omega}_{\text{i}})}\,(\normvec{\omega}_{\text{i}} + \normvec{\omega}_{\text{o}}),
\end{equation}

and by substituting $x$ for $(2(\normvec{h}\cdot\normvec{\omega}_{\text{i}}))^{-1}$, we get

\begin{equation} \label{eq:householder_equation_x_introduced}
\normvec{h} \,=\, x(\normvec{\omega}_{\text{i}} + \normvec{\omega}_{\text{o}}).
\end{equation}

Inserting this back into \eqref{eq:householder_equation} gives

\begin{equation}
\normvec{\omega}_{\text{o}} \,=\, 2(x(\normvec{\omega}_{\text{i}} + \normvec{\omega}_{\text{o}})\cdot\normvec{\omega}_{\text{i}})x(\normvec{\omega}_{\text{i}} + \normvec{\omega}_{\text{o}}) - \normvec{\omega}_{\text{i}}
\end{equation}

which can easily be rewritten as

\begin{equation} \label{eq:householder_x_second_order_with_vectors}
\vec{0} \,=\, (2(\normvec{\omega}_{\text{i}} + \normvec{\omega}_{\text{o}})\cdot\normvec{\omega}_{\text{i}}\,x^2 - 1)(\normvec{\omega}_{\text{i}} + \normvec{\omega}_{\text{o}}).
\end{equation}

Since both $\normvec{\omega}_{\text{i}}$ and $\normvec{\omega}_{\text{o}}$ are located in $\Omega_{\textbf{r}}$ which is on the positive side of the surface, we have

\begin{equation} \label{eq:reflection_direction_and_direction_against_light_source_on_positive_side_of_surface_separate}
\normvec{\omega}_{\text{i}}\cdot\normvec{h} > 0,\, \normvec{\omega}_{\text{o}}\cdot\normvec{h} > 0,
\end{equation}

which also leads to the inequality

\begin{equation} \label{eq:reflection_direction_and_direction_against_light_source_on_positive_side_of_surface_together}
(\normvec{\omega}_{\text{i}} + \normvec{\omega}_{\text{o}})\cdot\normvec{h} > 0.
\end{equation}

This means that $\normvec{\omega}_{\text{i}} + \normvec{\omega}_{\text{o}} \neq \vec{0}$, otherwise we would have $(\normvec{\omega}_{\text{i}} + \normvec{\omega}_{\text{o}})\cdot\normvec{h} = \vec{0}\cdot\normvec{h} \not > 0$. Since $\normvec{\omega}_{\text{i}} + \normvec{\omega}_{\text{o}} \neq \vec{0}$, we can rewrite \eqref{eq:householder_x_second_order_with_vectors} as

\begin{equation}
2(\normvec{\omega}_{\text{i}} + \normvec{\omega}_{\text{o}})\cdot\normvec{\omega}_{\text{i}}\,x^2 = 1,
\end{equation}

which means that

\begin{equation}
x = \pm\,\sqrt{\frac{1}{2(\normvec{\omega}_{\text{i}} + \normvec{\omega}_{\text{o}})\cdot\normvec{\omega}_{\text{i}}}}\,.
\end{equation}

Substituting this back into \eqref{eq:householder_equation_x_introduced} yields

\begin{equation} \label{eq:householder_equation_x_replaced}
\normvec{h} \,=\, \pm\frac{\normvec{\omega}_{\text{i}} + \normvec{\omega}_{\text{o}}}{\sqrt{2(\normvec{\omega}_{\text{i}} + \normvec{\omega}_{\text{o}})\cdot\normvec{\omega}_{\text{i}}}}\,.
\end{equation}

Now, since the distance is equally long both from the unreflected point and from the reflected point, to any point in the reflection plane, and since the \idxs{Householder}{transformation} assumes that the reflection plane contains the origin, we have

\begin{equation}
|\normvec{\omega}_{\text{i}}| \,=\, |\normvec{\omega}_{\text{o}}|,
\end{equation}

which means that

\begin{equation}
\renewcommand*{\arraystretch}{1.5}
\begin{array}{rl}
\displaystyle \sqrt{2(\normvec{\omega}_{\text{i}} + \normvec{\omega}_{\text{o}})\cdot\normvec{\omega}_{\text{i}}} & =\, \sqrt{2|\normvec{\omega}_{\text{i}}|^2 + 2\normvec{\omega}_{\text{i}}\cdot \normvec{\omega}_{\text{o}}} \\
\displaystyle & =\, \sqrt{|\normvec{\omega}_{\text{i}}|^2 + |\normvec{\omega}_{\text{0}}|^2 + 2\normvec{\omega}_{\text{i}}\cdot \normvec{\omega}_{\text{o}}} \\
\displaystyle & =\, \sqrt{\normvec{\omega}_{\text{i}}^2 + \normvec{\omega}_{\text{0}}^2 + 2\normvec{\omega}_{\text{i}}\cdot \normvec{\omega}_{\text{o}}} \\
\displaystyle & =\, \sqrt{(\normvec{\omega}_{\text{i}}+\normvec{\omega}_{\text{o}})^2} \\
\displaystyle & =\, |\normvec{\omega}_{\text{i}}+\normvec{\omega}_{\text{o}}|.
\end{array}
\end{equation}

So, we get

\begin{equation} \label{eq:halfway_vector_two_alternatives}
\normvec{h} \,=\, \pm\frac{\normvec{\omega}_{\text{i}} + \normvec{\omega}_{\text{o}}}{|\normvec{\omega}_{\text{i}}+\normvec{\omega}_{\text{o}}|}\,,
\end{equation}

which by using \eqref{eq:reflection_direction_and_direction_against_light_source_on_positive_side_of_surface_together} nicely reduces to

\begin{equation} \label{eq:halfway_vector_final}
\normvec{h} \,=\, \frac{\normvec{\omega}_{\text{i}} + \normvec{\omega}_{\text{o}}}{|\normvec{\omega}_{\text{i}}+\normvec{\omega}_{\text{o}}|}\,,
\end{equation}

which is just the normalization of $\normvec{\omega}_{\text{i}}+\normvec{\omega}_{\text{o}}$. In \idxs{computer}{graphics}, this vector is known as the \idxs{halfway}{vector}, since it is a \idxs{direction}{vector} that is "half way" between the direction vectors $\normvec{\omega}_{\text{i}}$ and $\normvec{\omega}_{\text{o}}$.

\eqref{eq:brdf_rewritten} can be rewritten to

\begin{equation} \label{eq:brdf_from_ratio_between_fluxes}
\rho'_{\text{r}}(\normvec{\omega}_{\text{i}},\, \normvec{\omega}_{\text{o}}) \,=\, (\cos\theta_{\text{o}})^{-1}\, \frac{\opd \anormvec{h}}{\opd \normvec{\omega}_{\text{o}}}\,\frac{\opd}{\opd \anormvec{h}}\left(\frac{\opd \Phi_{\text{o}}(\normvec{\omega}_{\text{o}})}{\opd \Phi_{\text{i}}(\normvec{\omega}_{\text{i}})}\right).
\end{equation}

Here, the term $\opd\,(\opd \Phi_{\text{o}}(\normvec{\omega}_{\text{o}})/\!\opd \Phi_{\text{i}}(\normvec{\omega}_{\text{i}}))$ can be identified as the ratio between the \idxs{reflected}{radient flux} that was reflected on a \microfacet with a surface \idx{anti-normal} within $\opd \anormvec{h}$, and the \idxs{incident}{radient flux}, for radiation coming from the direction $\normvec{\omega}_{\text{i}}$.

The anti-normal of the \microfacet is given by \eqref{eq:eta_anti_normal}, where the distribution of the first two elements is given by \eqref{eq:eta_grad_distribution_r_uniformly_distributed} for a \idxse{uniform}{distribution}{uniformly distributed} $\vec{r}$. However, here it is the incident rays that determine the distribution of $\vec{r}$, so $\vec{r}$ will not be uniformly distributed but somewhat \biased towards microfacets facing the light source.

If a large surface, whose area when projected down to the horizontal plane is $A_{\text{h}}$, is subdivided into many infinitesimal microfacets, the total project area of all microfacets whose gradient of the free surface elevation lies within $\opd\nabla\eta^*$ will be

\begin{equation}
\opd A'_{\text{h}} \,=\, A_{\text{h}}\,f(\nabla\eta^*(\vec{r}))\opd\nabla\eta^*,
\end{equation}

where an apostrophe denotes a restriction on the gradients of the free surface elevation on the included microfacets, and $f(\nabla\eta^*(\vec{r}))\opd \nabla\eta^*$ is the chance for a single microfacet that its free surface elevation gradient within $\opd\nabla\eta^*$. When projected onto the plane orthogonal to $\normvec{\omega}_{\text{i}}$, the total area of all microfacets will be

\begin{equation}
A_{\text{i}} \,=\, A_{\text{h}}\,\normvec{\omega}_{\text{i}}\cdot\normvec{n}\,(\normvec{n}\cdot\normvec{z})^{-1} \,=\, A_{\text{h}}\,\normvec{\omega}_{\text{i}}\cdot\anormvec{n}
\end{equation}

and the total area of all microfacets whose gradient of the free surface elevation lies within $\opd\nabla\eta^*$ will be

\begin{equation}
\opd A'_{\text{i}} \,=\, \opd A'_{\text{h}}\,\normvec{\omega}_{\text{i}}\cdot\normvec{h}\,(\normvec{h}\cdot\normvec{z})^{-1} \,=\, \opd A'_{\text{h}}\,\normvec{\omega}_{\text{i}}\cdot\anormvec{h},
\end{equation}

since all microfacets have the same area when projected to the horizontal plane. Here, $\normvec{h}$ is the normal the free surface elevation with the gradient around which $\opd\nabla\eta^*$ is centered. Therefore, the chance that a ray that comes from the direction $\normvec{\omega}_{\text{i}}$ and hits the surface will hit a \microfacet whose gradient of the free surface elevation lies within $\opd\nabla\eta^*$ will be

\begin{equation} \label{eq:gradient_restricted_reflection_probability}
\opd P'(\normvec{\omega}_{\text{i}}) \,=\, \frac{\opd A'_{\text{i}}}{A_{\text{i}}} \,=\, \frac{\opd A'_{\text{h}}\,\normvec{\omega}_{\text{i}}\cdot\anormvec{h}}{A_{\text{h}}\,\normvec{\omega}_{\text{i}}\cdot\anormvec{n}} \,=\, f(\nabla\eta^*(\vec{r}))\,\frac{\normvec{\omega}_{\text{i}}\cdot\anormvec{h}}{\normvec{\omega}_{\text{i}}\cdot\anormvec{n}}\opd \nabla\eta^*.
\end{equation}

For an increase in the radiation coming from the direction $\normvec{\omega}_{\text{i}}$, with the \idxs{incident}{radient flux} $\opd \Phi_{\text{i}}(\normvec{\omega}_{\text{i}})$, the increase in \idxs{reflected}{radient flux}, $\opd^2 \Phi_{\text{o}}(\normvec{\omega}_{\text{i}})$, that was reflected on a \microfacet with a free surface elevation whose gradient lies withing $\opd\nabla\eta^*$ will be

\begin{equation} \label{eq:gradient_restricted_reflected_flux}
\opd^2 \Phi_{\text{o}}(\normvec{\omega}_{\text{o}}) \,=\, \opd \Phi_{\text{i}}(\normvec{\omega}_{\text{i}})\,\opd P'(\normvec{\omega}_{\text{i}})\,R_{\normvec{n},x}(\normvec{\omega}_{\text{i}}),
\end{equation}

where $R_{\normvec{n},x}(\normvec{\omega}_{\text{i}})$ is the \idxs{reflection}{coefficient}, predicted by the \idxs{Fresnel}{equations}, for a ray coming from the direction $\normvec{\omega}_{\text{i}}$, for a surface with the surface normal $\normvec{n}$ and for the light polarization $x$.

In computer graphics, the reflection coefficient is often \approximated by using \idxs{Schlick's}{approximation}, which reduces the complexity of the expression as well as the time required to evaluate the expression. On the other hand, while the approximation works well as long as there is no filtering depending on the polarization direction involved, using the approximation when such a filtering \emph{is} involved will result in a poor end result since it doesn't take the polarization direction into account. In the case when filtering depending on the polarization direction is involved, the \idxs{Fresnel}{equations} can more accurately be \approximated by substituting low order polynomials for the square roots contained within the expressions.

Using \eqref{eq:gradient_restricted_reflection_probability} in \eqref{eq:gradient_restricted_reflected_flux} results in

\begin{equation} \label{eq:gradient_restricted_reflected_flux_pre_incident_flux_per_unit_gradient_squared}
\frac{\opd}{\opd \nabla\eta^*}\,\left(\frac{\opd \Phi_{\text{o}}(\normvec{\omega}_{\text{o}})}{\opd \Phi_{\text{i}}(\normvec{\omega}_{\text{i}})}\right) \,=\, f(\nabla\eta^*(\vec{r}))\,\frac{\normvec{\omega}_{\text{i}}\cdot\anormvec{h}}{\normvec{\omega}_{\text{i}}\cdot\anormvec{n}}\,R_{\normvec{n},x}(\normvec{\omega}_{\text{i}}).
\end{equation}

By realizing that the normal of the a \microfacet, $\normvec{n}^*$, needs to equal $\normvec{h}$ in order for a ray coming from the direction $\normvec{\omega}_{\text{i}}$ to be reflected in the direction $\normvec{\omega}_{\text{o}}$, we can use \eqref{eq:eta_anti_normal} to conclude that in this case, we can make the substitution

\begin{equation}
\opd \anormvec{h} \,=\, \opd\nabla\eta^*
\end{equation}

(note that most infinitesimal quantities used in this appendix are two-dimensional in their nature; in this case the minus sign in \eqref{eq:eta_anti_normal} is squared and vanishes). By applying this to \eqref{eq:gradient_restricted_reflected_flux_pre_incident_flux_per_unit_gradient_squared}, we can rewrite  \eqref{eq:brdf_from_ratio_between_fluxes}

\begin{equation}
\rho'_{\text{r}}(\normvec{\omega}_{\text{i}},\, \normvec{\omega}_{\text{o}}) \,=\, (\cos\theta_{\text{o}})^{-1}\, \frac{\opd\anormvec{h}}{\opd\normvec{\omega}_{\text{o}}}\,f(\nabla\eta^*(\vec{r}))\,\frac{\normvec{\omega}_{\text{i}}\cdot\anormvec{h}}{\normvec{\omega}_{\text{i}}\cdot\anormvec{n}}\,R_{\normvec{n},x}(\normvec{\omega}_{\text{i}}).
\end{equation}

Since we have the restriction $\anormvec{h}\cdot\normvec{z}\equiv 1$, $\opd\anormvec{h}$ is here the size of an infinitesimal "quasi area" element. Let's \assume this element is an axis aligned rectangle with the sides $\opd x_{\anormvec{h}}^{\,}\normvec{x}$ and $\opd y_{\anormvec{h}}^{\,}\normvec{y}$ respectively. We can then describe the infinitesimal solid angle element $\opd\normvec{\omega}_{\text{o}}$ as a parallelogram with the sides $\opd x_{\anormvec{h}}^{\,}J_{\normvec{\omega}_{\text{o}}}(\anormvec{h})^{\,}\normvec{x}$ and $\opd y_{\anormvec{h}}^{\,}J_{\normvec{\omega}_{\text{o}}}(\anormvec{h})^{\,}\normvec{y}$ respectively, where $J_{\normvec{\omega}_{\text{o}}}(\anormvec{h})$ is the \idxs{Jacobian}{matrix} for $\opd\normvec{\omega}_{\text{o}}$ as a function of $\anormvec{h}$, which contains all the partial derivatives of $\opd\normvec{\omega}_{\text{o}}$, that is

\begin{equation} \label{eq:jacobian}
J_{\normvec{\omega}_{\text{o}}}(\anormvec{h}) \,=\, \renewcommand*{\arraystretch}{2} \left( \begin{array}{ccc}
\displaystyle \frac{\partial(\normvec{\omega}_{\text{o}}\cdot\normvec{x})}{\partial(\anormvec{h}\cdot\normvec{x})} &
\displaystyle \frac{\partial(\normvec{\omega}_{\text{o}}\cdot\normvec{x})}{\partial(\anormvec{h}\cdot\normvec{y})} &
\displaystyle \frac{\partial(\normvec{\omega}_{\text{o}}\cdot\normvec{x})}{\partial(\anormvec{h}\cdot\normvec{z})} \\
\displaystyle \frac{\partial(\normvec{\omega}_{\text{o}}\cdot\normvec{y})}{\partial(\anormvec{h}\cdot\normvec{x})} &
\displaystyle \frac{\partial(\normvec{\omega}_{\text{o}}\cdot\normvec{y})}{\partial(\anormvec{h}\cdot\normvec{y})} &
\displaystyle \frac{\partial(\normvec{\omega}_{\text{o}}\cdot\normvec{y})}{\partial(\anormvec{h}\cdot\normvec{z})} \\
\displaystyle \frac{\partial(\normvec{\omega}_{\text{o}}\cdot\normvec{z})}{\partial(\anormvec{h}\cdot\normvec{x})} &
\displaystyle \frac{\partial(\normvec{\omega}_{\text{o}}\cdot\normvec{z})}{\partial(\anormvec{h}\cdot\normvec{y})} &
\displaystyle \frac{\partial(\normvec{\omega}_{\text{o}}\cdot\normvec{z})}{\partial(\anormvec{h}\cdot\normvec{z})}
\end{array} \right) \,=\, (\nabla_{\anormvec{h}}\otimes\normvec{\omega}_{\text{o}})^{\T},
\end{equation}

where in turn $\nabla_{\anormvec{h}}$ is the gradient operator containing all the partial derivatives with respect to the elements in $\anormvec{h}$, that is

\begin{equation}
\nabla_{\anormvec{h}} \,=\, (\partial/\partial(\anormvec{h}\cdot\normvec{x}),\, \partial/\partial(\anormvec{h}\cdot\normvec{y}),\, \partial/\partial(\anormvec{h}\cdot\normvec{z}))^{\T}.
\end{equation}

Using the fact that $\normvec{h} = \anormvec{h}\,\left|\anormvec{h}\right|^{-1}$, we can rewrite \eqref{eq:householder_equation} as

\begin{equation} 
\normvec{\omega}_{\text{o}} \,=\, \frac{2(\anormvec{h}\cdot\normvec{\omega}_{\text{i}})\anormvec{h}}{\anormvec{h}^2} - \normvec{\omega}_{\text{i}}
\end{equation}

and \eqref{eq:jacobian} can then be rewritten as

\begin{equation}
\renewcommand*{\arraystretch}{2}
\begin{array}{c}
\displaystyle (J_{\normvec{\omega}_{\text{o}}}(\anormvec{h}))^{\T} \,=\,  \nabla_{\anormvec{h}}\otimes\left(\frac{2(\anormvec{h}\cdot\normvec{\omega}_{\text{i}})\anormvec{h}}{\anormvec{h}^2} - \normvec{\omega}_{\text{i}}\right) \\
\displaystyle =\, \frac{2\anormvec{h}\cdot\normvec{\omega}_{\text{i}}}{\anormvec{h}^2}\,I + \frac{2\anormvec{h}\otimes\normvec{\omega}_{\text{i}}}{\anormvec{h}^2} + \dots
\end{array}
\end{equation}

\HRule

Using the product rule for the gradient operator,

\begin{math}
\nabla(\vec{u}\cdot\vec{v}) \,=\, (\vec{u} \cdot \nabla) \vec{v} + (\vec{v} \cdot \nabla) \vec{u} + \vec{u} \times (\nabla \times \vec{v}) + \vec{v} \times (\nabla \times \vec{u})
\end{math}

















\HRule

Since $\anormvec{h}$ has a constant z-component, and since the surface whose area is $\opd \normvec{\omega}_{\text{i}}$ has the normal $\normvec{\omega}_{\text{i}}$, we can conclude that

\begin{equation}
\frac{\opd \normvec{\omega}_{\text{o}}}{\opd \anormvec{h}} \,=\, \left(\left(J_{\normvec{\omega}_{\text{o}}}(\anormvec{h})\normvec{x}\right)\times\left(J_{\normvec{\omega}_{\text{o}}}(\anormvec{h})\normvec{y}\right)\right)\cdot\normvec{\omega}_{\text{o}},
\end{equation}

where $J_{\normvec{\omega}_{\text{o}}}(\anormvec{h})$ is the \idxs{Jacobian}{matrix}