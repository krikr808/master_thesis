\chapter{Discussion}

As mentioned in the last chapter, there were quite many issues that needed to be dealt with in order to make \thismethod, which was the method implemented in \thisprojectwork, work well. Although the method very well can simulate phenomenas such as turbulence, nonlinearity, breaking waves, splashes, sound waves, etc., just given enough computational power and enough time to improve it, these phenomenons do not need to be simulated in a flight simulator, and there are other methods, such as the \LPD method, that are simpler to implement, run faster and still have all the required properties. Some of the difficulties with the method includes:

\begin{itemize}
    \item Dynamical creation/destruction of surface cells and determination of properties in the new cells,
    \item Handling of high Courant numbers (described in \citep{Stam1999}),
    \item Solving \idxs{pressure Poisson}{equation} (described in a large amount of research papers, see \secref{sec:pressure_poison_equation_solution}),
    \item Handling of high speeds of sound if the fluids are assumed to be compressible (described in \citep{Kwatra2009}),
    \item Keeping the interface sharp (described in a large amount of research papers, see \secref{sec:advection_of_phase_fraction}), and
    \item Making it work in real-time.
\end{itemize}

\section{Other methods to use}

For real-time simulations of surface waves in large bodies of water, where \idxse{non-linear}{phenomenon}{non-linear phenomena} are not of any significant importance, the best method to use, considering both implementation time and simulation quality, is probably the \LPD method. If there would turn out to be an easy way to model \FSI while using the Fourier synthesis method, that method is probably to prefer, since is it very easy to implement and extend to make it simulate Gerstner waves, and because of the large number of sources describing it.

On the other hand, if the \FVM is going to be used anyway, maybe a better approach than using the \VOF method would be to use the \LS method, or perhaps the \CLSVOF method, since those methods don't suffer from \diffusion of the interface. Besides, since the property that is used for determining where the surface is located is stored in the corners of the cells, the \idxs{marching}{cubes} algorithm can be used more easily.

\section{Speed}

With its \idxs{linear}{running time}\index{time!running|see{running time}}, or $O(N)$ \idxs{time}{complexity}, where $N$ is the number of cells visible on the surface, the method implemented in \thisprojectwork places itself among the methods with the lowest time complexity, such as the \LPD method, even beating the Fourier synthesis method with its $O(N \log N)$ time complexity.

However, \thismethod has a very high time constant associated with the big O notation, making it very slow, and it probably will not surpass the Fourier synthesis method in speed until a ridiculously large number of cells are being simulated, which is very impractical. The Fourier synthesis method ais still a very fast method, much thanks to its simplicity.

\section{Already existing software}

There is a number of software dedicated for simulating fluids under various licenses. A few of the free software \idxsp{CFD}{package}{s} that exist are OpenFOAM (originally \idx{FOAM}, for \textit{\idxs{field}{operation and manipulation}}\index{field operation and manipulation|see{FOAM}}, outlined in \textit{\href{http://powerlab.fsb.hr/ped/kturbo/openfoam/docs/foam.pdf}{A tensorial approach to computational continuum mechanics using object-oriented techniques}} \citep{temp}), which is free and open source, and \Gerris \citep{temp}, published under the Free Software GPL license. There are several commercial CFD solvers as well, including \RealFlow and \idxs{ANNSYS}{Fluent}. %For a larger list of available software, see \textit{\href{http://www.cfd-online.com/Links/soft.html}{CFD Online: Links - Software}} \citep{temp}.

However, the majority of existing CFD packages are designed to produce general, high quality simulations to be used for analysis or by artists, and are not aimed towards \realtime simulations.