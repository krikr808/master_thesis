\chapter{Discussion}
\label{chap:discussion}

As mentioned in the last chapter, there were quite many issues that needed to be dealt with in order to make \thismethod, which was the method implemented in this thesis work, work well. Although the method very well can simulate phenomena such as turbulence, nonlinearity, breaking waves, splashes, sound waves, etc., just given enough computational power and enough time to improve it, these phenomena do not need to be simulated in a flight simulator, and there are other methods, such as the \LPD method, that are simpler to implement, run faster and still have all the required properties. Some of the difficulties with the method includes:

\begin{itemize}
    \item Making the simulation go in real-time, including having a high enough frame rate,
    \item Dynamical creation/destruction of surface cells and determination of properties in the new cells,
    \item Handling of large Courant numbers (described by among others \citet{Stam1999,Lentine2012}),
    \item Solving \idxs{pressure Poisson}{equation} (see \secref{sec:pressure_poisson_equation_solution}),
    \item Handling of high speeds of sound if the fluids are assumed to be compressible (described by \citet{Kwatra2009}), and
    \item Keeping the interface sharp (see \secref{sec:advection_of_phase_fraction}).
\end{itemize}

\section{Other methods to use}

For real-time simulations of surface waves in large bodies of water, where \idxse{non-linear}{phenomenon}{non-linear phenomena} are not of any significant importance, the best method to use, considering both implementation time and simulation quality, is probably the \LPD method. If it would turn out that there is an easy way to model \FSI while using the Fourier synthesis method (which is probable, considering what is discussed in \secref{sec:fluid_structure_interaction}), this method is probably to prefer, since it is very easy to implement and to extend in order to make it simulate Gerstner waves, and because of the large number of sources describing it.

On the other hand, if the \FVM is going to be used anyway, maybe a better approach than using the \VOF method would be to use the \LS method, or perhaps the \CLSVOF method, since those methods don't suffer from \diffusion of the interface. Besides, since the property that is used for determining where the surface is located is stored in the corners of the cells, the \idxs{marching}{cubes} algorithm can be used more easily.

\section{Speed}

With its \idxs{linear}{running time}\index{time!running|see{running time}}, or $O(N)$ \idxs{time}{complexity}, where $N$ is the number of cells visible on the surface, the method implemented in this thesis work places itself among the methods with the lowest time complexity, such as the \LPD method, even beating the Fourier synthesis method with its $O(N \log N)$ time complexity.

However, \thismethod has a very high time constant associated with the big O notation, making it very slow, and it probably will not surpass the Fourier synthesis method in speed until a ridiculously large number of cells are being simulated, which is very impractical. The Fourier synthesis method is still a very fast method, much thanks to its simplicity.

\section{Already existing software}

There is a number of software dedicated for simulating fluids under various licenses. A few of the free \CFD software packages that exist are \OpenFOAM (originally \idx{FOAM}, for \textit{\idxs{field}{operation and manipulation}}\index{field operation and manipulation|see{FOAM}}, outlined in an article by \citet{Weller1998}), which is free and open source, and \Gerris \citep{Popinet2003}, published under the Free Software GPL license. There are also several commercial CFD solvers, including \RealFlow and \idxs{ANSYS}{Fluent}. %For a larger list of available software, see http://www.cfd-online.com/Links/soft.html.

However, the majority of existing CFD packages are designed to produce generic, high quality simulations to be used for analysis or by artists, and are not aimed towards \realtime simulations.

\section{Improvements}
\label{chap:improvements}

There are a lot of improvements of the method that was implemented in this thesis work that can be performed, many of them which are necessary to make the method be practically useful in a real-time flight simulator.

%\subsection{Required improvements}

%In this section, some of the improvements that are required to make the method practically useful are presented.

\subsection{Adaptive Mesh Refinement}

The \LOD needs to be adapted to the \idxs{camera}{placement} and the \FOV/\idxs{viewing}{frustum} as well as the water depth in order to greatly reduce the number of cells used by the \FVM in the simulation, in order to only use the minimum amount of cells that are actually needed. This is referred to as \AMR.

Without \AMR it would be impossible to simulate larges bodies of water and fine surface details real-time in the same scene, as these would require heaps of orders of magnitude more cells to still be able to simulate waves with the same short \wavelength as with \AMR.

%\subsubsection{No simulation of air}

In this thesis work, not just the water was simulated, but also the air, which meant that the simulation domain needed to have a boundary in the air. This introduced the additional problem of choosing where the boundary should be located, and what boundary conditions it should have. Additionally, this meant extra computational work since more cells had to be simulated which reduced the simulation speed significantly, and it introduced the phenomenon that velocities could suddenly go to infinity, which froze the simulation.

A better solution would be to exclude the air region and treat that as vacuum, and move the boundary to the interface between the water and the vacuum/air region.

\subsection{Unconditionally stable flows}

In order to make a \FVM simulation become practically useful for \idxsp{real-time}{simulation}{s}, it is required that the flow is \idxse{unconditionally stable}{flow}{unconditionally stable}\index{stable flow!unconditionally|see{unconditionally stable flows}}. This implies that it is necessary that arbitrarily large \timesteps can be taken without making the simulation unstable.

When it comes to physical quantities with a non-damped \oscillating behavior, simulations which use any kind of normal, \idxs{explicit}{time-stepping}\index{stepping!in time|see{time-stepping}} often tend to make them become unstable and create a \idxs{numerical}{explosion}.

In this thesis work, a kind of \idxs{leapfrog}{integration} has been used to model the time evolution of the velocity and the pressure, but even \idxs{leapfrog}{integration} has to obey the \CFL condition, and it becomes numerically unstable when the \period of the oscillations becomes too small in relation to the \timestep. To solve this problem, many methods \approximate the fluid or the fluids as \incompressible, while other use a combination of standard advection methods and \idxse{implicit}{Euler integration}{backwards (or implicit) Euler integration}\index{integration!implicit Euler|see{implicit Euler integration}} when updating the pressure (see \secref{sec:pressure_calculation} for a more detailed discussion about this).

%In backwards Euler integration, the equations for ordinary Euler integration are set up as if the flow would have been solved backwards in time (that is, as if one would want to calculate the state for one point in time from the state for a later point in time). These equations are then used to solve the flow forward in time. This effectively reverses all instability and turns it into damping, and vice versa.

On the other hand, when it comes to advection, many advection schemes tend to become numerically unstable when the \idxs{Courant}{number} $C$ (defined in \eqref{eq:courant_number}) becomes too large (this normally means larger than one). Whereas most \idxsp{explicit}{Runge--Kutta method}{s}\index{method!explicit Runge--Kutta|see{explicit Runge--Kutta method}} for advecting any form of field tend to damp high frequent features in the field for $C < 1$, they do the opposite for $C > 1$ and instead enhance the high frequent features and become unstable. If a \idxs{backwards}{Runge--Kutta method}\index{method!backwards Runge--Kutta|see{backwards Runge--Kutta method}} is used (which is equivalent to using an ordinary, explicit Runge--Kutta method but with a negative time step) in an attempt to remedy this instability, the behaviors will just be reversed. High frequent features will then be damped for $C > 1$, hence making the simulation stable in those regions, but they will be enhanced for $C < 1$, hence still making the simulation unstable.

A remedy for this dilemma is to use semi-Lagrangian advection, which works by calculating which locations will end up on the \idxsp{discretization}{point}{s} in the end of the time step, and assigning interpolations of the field in those locations to the discretization points in the end of the time step.

All of these instability issues have remedies, at least for simulations on a regular grid \citep{Stam1999,Kwatra2009,Lentine2012}, but those remedies can probably also be extended for octree grids.

\subsection{Rendering}

In this thesis work, the water surface is not rendered. To render the surface properly, an isosurface needs to be extracted from the \idxs{phase fraction}{data}.

\subsubsection{Marching Cubes}
\label{sec:marchingcubesdescription}

In order to find where the \idxs{water}{surface} is located, the \idxs{marching}{cubes} \idxe{algorithm!marching cubes|see{marching cubes}}{algorithm} can be used. Marching cubes is an algorithm for finding \isosurfaces from a \idxs{scalar}{field}, and works by looping over a set of adjacent cubes tiled closely together which just covers the domain that is about to be rendered. For each cube, the algorithm looks at the value of the scalar field in all eight of the cube's corners. If the field values in the corners all lie at the same side of the value $c$ that the isosurface represents, it figures that the surface doesn't cross through that cube, otherwise it does and the algorithm adapts a surface consisting of a finite number of polygons through the cube depending on what the field values in the corners are.

In this thesis work, the scalar field that tells where the surface is located is the \idxs{phase fraction} {field}, $\alpha$, and since $\alpha$ crosses from 0 to 1 over the \interface, the isosurface can really be chosen for any level $0 < c < 1$ and the marching cubes algorithm will produce an isosurface lying within the interface, although it probably makes more sense to choose $c = 0.5$ which lies in the middle of the interval and makes the isosurface generation symmetric in that sense. One problem though is that the $\alpha$ field isn't discretized in the cell corners, but in the cell centers. This can be remedied by defining a new set of cubes which have their corners in the centers of the cells. This will become a bit problematic in \idxsp{LOD}{transition}{s}, for which the cubes if created in this way will stretch over two different LODs and thus not have a very well defined size. Another possible solution is to interpolate the $\alpha$ field to the cell corners so that the cells can be used directly as the cubes needed by the algorithm.

If the \VOF method for tracing the surface would be switched to using the \LS method instead, a signed distance function, $\phi$, would be used instead of $\alpha$, and $c = 0$ would then define where the surface is located. The discretization points for $\phi$ would be in the cell corners, so there would be no need for defining a new set of cubes in order for the marching cubes algorithm to work. A class of fast marching methods related to the \LS method was developed in a work by \citet{Sethian1995}.

However, the marching cubes algorithm would have apparent problems to generate the surface in \idxsp{LOD}{transition}{s}, since, in its original form, it requires that all cubes are of the same size, otherwise glitches may occur. This problem is well known, and remedies exist. For example, in a work by \citet{Lengyel2010} this is remedied for the case where the cubes used by the algorithm are modeled as an \octree.

\subsection{Parallelization}

In order to obtain a simulation speed-up that is almost certainly needed if the simulation is to run in real-time, it can be processed on multiple \CPUs simultaneously, as well as on the \GPU (if the \octree traversal is implemented to run without recursion, since \GPUs generally don't have any stack which means that they can't do recursion). This is generally referred to as \parallelization. For parallelization to be possible, the \idxs{simulation}{domain} first has to be \partitioned, so that a small part of the domain can be processed on each core and on each \idxs{GPU}{process}.

Since the process that takes the longest time to run will decide the speed of the simulation, the size of the partitions should be roughly the same. Besides, access to data that is located outside of the partition takes extra long time, so it is desired that the number of neighbors cells for each partition that lies outside of the partition itself is kept low, which is somewhat similar to the problem of minimizing the area between the partitions. For a structured grid, the partitioning can be effectively achieved by cutting the simulation domain into square or cubic pieces. However, for an \idxs{unstructured}{grid} like an \octree, the partitioning is more difficult, but still possible.

For octrees, there are algorithms that can be used that make use of \idxsp{space-filling}{curve}{s} and are fast, easy to implement and produce very acceptable results \citep{Valgaerts,Ji2012}. There also exists another kind of partitioning method that uses \idxs{domain}{decomposition} \citep{Fuster2011,Agbaglah2011}.

\subsection{Fluid--Structure Interaction}

In this thesis work, no interaction between ships and the surface waves was ever modeled, in other words, no \FSI was modeled. Modeling and simulating \FSI is necessary in order to make large waves affect ships so that they start to rock forth and back, as well as in order for ships to give rise to two \idxsp{wake}{line}{s} as they \sail on the water.

Some of the methods for modeling \FSI includes the \IB method \citep{Peskin2002,Ghias2007}, the \VOS method and the \idxs{rigid fluid}{method}\index{fluid!rigid|see{rigid fluid method}} \citep{Carlson2004}.
%See e.g.: \item http://physbam.stanford.edu/~fedkiw/papers/stanford2010-04.pdf (Numerically Stable Fluid-Structure Interactions Between Compressible Flow and Solid Structures) \item http://efdl.as.ntu.edu.tw/research/papers/JCP03GCIBM.pdf (A ghost-cell immersed boundary method for flow in complex geometry) \item http://www.cs.columbia.edu/~batty/papers/Batty07.pdf (A Fast Variational Framework for Accurate Solid-Fluid Coupling) (solid fraction, non-stick to walls)

The \idxs{rotation of a}{rigid body}, in this case the rotation of a \ship that is floating on the surface of the water when it is hit by large waves, depends both on the forces that act on the \idxs{rigid}{body} as well as the  \idxs{moment of}{inertia} of the rigid body. The dynamics of a rigid body is described by Euler's equations for rigid bodies (not to be confused with the Euler equations, which are discussed in \chapref{chap:thefinitevolumemethod}).

%\subsection{Desired improvements}

%In this \levelname, some improvements that are not required in order to make the method practically useful, but still desired in order to improve the quality of the simulation, are presented.

\subsection{Reduction of spurious reflections in LOD transitions}

When a wave hits a boundary to a region where it propagates with a different speed, or cannot propagate at all, a partial or total reflection of the wave will occur. In natural scenes, this typically happens when the depth of the water changes abruptly from deep water to shallow, or when the wave hits a cliff or similar.

In scenes simulated with a grid of finite size, this usually happens on the border of the grid, unless special care has been taken to this when deciding which \idxsp{boundary}{condition}{s} to use. These reflections are unwanted and referred to as \idxs{spurious}{reflections} since they don't occur in nature.

In scenes simulated on an \octree based \grid, this also happens in \LOD transitions. This is partly due to the fact that the set of wave frequencies that can be represented on the grid varies between different \idxsp{LOD}{level}{s}, and partly because the fact that a wavelength that can be represented on two different LOD levels often will propagate with different speeds on the two LOD levels, which will make part of the wave be reflected at the transition, and part of the wave be transmitted. The latter is especially apparent if the wavelength is close to the minimum wavelength that the highest (coarsest) of the two LOD levels can represent.

%In scenes simulated on an arbitrary grid though, the reflections corresponding to the reflections that takes place in LOD transitions are often less noticeable as the surface of reflection is less well-defined, which causes the waves to be reflected at multiple locations and become scattered in many directions, although the waves are still eventually totally reflected.

\subsubsection{Perfectly matched layers}

One method for greatly reducing spurious reflections at the border of a structured grid is to use \idxs{perfeclty matched}{layers}, which are described in among others a work by \citet{Soderstrom2010} that explains explicit dampening, implicit dampening, and wave absorbing boundaries --- the perfectly matched layer approach --- and evaluates the methods.

\subsubsection{Other absorbing boundary conditions}

Another method, that can be used when there is no wave dispersion, that is, waves of different wavelengths have the same speed, is to use open boundary conditions \citep{Muller-Fischer2008}. This method works ideally in the \idx{one-dimensional} case, but performs poorly in the \idxe{two-dimensional}{two-} and \idx{three-dimensional} cases when the waves come against the boundary at a glancing angle. Yet another method that can be used to prevent waves from being reflected at boundaries are the \AWWE \citep{Guddati2005}, in which the propagation of waves is only simulated in one direction, meaning that for every pair of wave vectors $\vec{k}$ and $-\vec{k}$, only one of them is considered in the simulation. Hence the waves are naturally prevented from propagating back from a boundary into the simulation domain. This is basically also what happens when the open boundary conditions are used, but since those are derived from the one-dimensional case, they only work well for waves hitting the boundary from an almost orthogonal direction, while the \AWWE assume an arbitrary angle for the incoming wave and hence can obtain a high accuracy for waves hitting the boundary from within a wide range of angles.

\subsection{Wind waves}

In order to obtain a realistic \idxs{sea}{state}\index{state of sea|see{sea state}} in a simulation, it is desirable that the \idxs{water}{surface} is affected by the \wind in order to give rise to waves; such waves are commonly referred to as \idxsp{wind}{wave}{s}.

\subsubsection{Spectral methods}

A very cheap and easy way to simulate wind waves is to use a wind dependent \idxs{wave}{spectrum} to initialize the sea state with wind waves already in it when the simulation starts. A wave spectrum tells how the wave amplitude is \distributed as a function of wavelength, and from that distribution the \idxs{Fourier}{transform} of the free surface elevation can easily be \generated simply by \sampling it. This is then used to \initialize the \idxs{sea}{state} in the simulation.

There are different models for describing the wave spectrum for a certain wind velocity, most of which are \empirical. Some of these spectra take into account the wind direction for increased alignment between the surface and the wind, for example the \idxs{Philips}{spectrum}, which is used and tweaked by \citet{Tessendorf2001}, or the \idxs{Pierson--Moskowitz}{spectrum}, which was supposedly developed in 1964 and is described by \citet{Premovze2001}. Another commonly used wave spectrum which builds up over time is the \JONSWAP spectrum\index{spectrum!JONSWAP|see{JONSWAP}} \citep{Journee2001b,WikiWaves2013}.

\subsubsection{Air--water interaction}

Another (somewhat dubious) alternative would be to simulate both the \water and the \air using the \FVM. This method would be slower than using a wave spectrum since it has to simulate more cells, and would probably give a quite poor result since it is difficult to catch the correct interaction between the water and the air for short \wavelengths.

\subsection{Improved rendering}

In order to create a realistic rendering of the water surface, there are a number of aspects that has to be considered. Below follows a few of those.

\subsubsection{Light reflectivity and transmissivity}

When \light hits the \idxs{water}{surface}, not all light is transmitted into the water. Some of the light is reflected, and the amount of light that is reflected depends on the angle of incidence as well as the polarization of the light, according to the \idxs{Fresnel}{equations}.

Normally however, when rendering a \scene with \idxs{computer}{graphics}, one doesn't usually start from the light source when calculating the brightness of an object, but from the observer. On the other hand, according to the \idxs{Helmholtz}{reciprocity principle}\index{principle!Helmholtz reciprocity|see{Helmholtz reciprocity principle}}, one can use the Fresnel equations not only to calculate how much of the light is being reflected, but also the other way around to calculate how much of the light that comes from the surface originally hit the surface from above and was reflected, and how much of it hit the surface from below and was transmitted. In that way it is possible to account also for the light that has been reflected under the water, which has been filtered by the objects it has been reflected on under the surface of the water.

\subsubsection{Statistical treatment of short wavelengths}

Typically, when creating a wire frame model for a sea surface, which is needed in rendering of the surface, only wavelengths that are long enough are included since wavelengths that are too short will give rise to folding. Besides, not all wavelengths are present in the simulation because of the limited resolution of the \grid, so it is not possible to include all wavelengths anyway, simply because much of them are unavailable. If much of the \idxs{wave}{energy} in a \idxs{sea}{state} is excluded from the wire frame model, which it will if for example the surface is viewed from a far distance, the surface will appear much smoother in the rendered image than what it really is, unless extra treatment is taken for the excluded wavelengths.

A naive shading model will use the surface normal of the wire frame model when calculating the light intensity of the surface, and completely ignore the part of the wave spectrum that is not in it. A more sophisticated shading model will estimate the probability density of the normal of the actual water surface, given the surface normal of the wire frame model in combination with the known statistics of the excluded wavelengths. The distribution of the normal vector is in turn used to model the distribution of the surface light intensity, from which the expectation value is calculated and used as the light intensity of the surface in the rendered image. This will effectively prevent the water surface from looking too smooth.

The derivation of a simple, physically based illumination model for sea surfaces is found in \appref{apdx:illumination_model_derivation}.
\subsection{Visual effects}

\subsubsection{Choppy waves}

A waves of a single frequency is not a perfect \sinusoidal, but a \trochoid, which makes the curvature greater at the \idxs{wave}{crest} and lesser at the \idxs{wave}{valley}, which makes the waves look more \idxse{choppy}{waves}{choppy}. By letting sinusoidals displace the water surface, not only vertically, but also horizontally, trochoids can effectively be created and a more realistic look of the water surface is obtained. These waves are called \idxsp{Gerstner}{wave}{s} and were first found as an approximate solution to the fluid dynamic equations almost 200 years ago, as noted by \citet{Tessendorf2001}, and is today extensively used in computer graphics for animating water surfaces. What is claimed to be an implementation of Gerstner waves can be seen in an animation on the web by \citet{ceribral2012}.

\subsubsection{Splash and foam}

In rough weather conditions, the waves will become higher and choppier, and their crests will break more easily, forming white foam where they break, as can be seen in \figref{fig:sea_storm}, which is known as oceanic whitecaps, or just whitecaps. When high waves hit the side of a ship, they will also shatter and form a spray of small particles in the air, usually just referred to as splashes.

In a work by \citet{Ihmsen2012}, "diffuse particles" are generated based on the water's potential to trap air, its likelihood to be at the crest of a wave and its kinetic energy. In a work by \citet{Takahashi2003}, water is turned into "splash particles" where the curvature of the water exceeds a certain \threshold. The results of the use of diffuse/splash particles can bee seen in videos on the web \citep{Chandel2009,RealFlowLabs2011,Ihmsen2012}.

In turn, in the work by \citet{Takahashi2003}, it is concluded that there seems to be "few papers on handling of effects of splashes and foam with fluid", even though it mentions a paper by \citet{Premovze2001} as one of them, which it claims "makes crude approximations". However, this paper includes an empirical formula for the "fractional area of the wind-blown water surface that is covered by foam" (the time-averaged area of the foam divided by the total area of the water surface), $f$, that looks like
%
\begin{equation} \label{eq:fractional_area_naiive}
f(U,\, T_{\text{a}},\, T_{\text{w}}) = 1.59 \times 10^{-5}U^{2.55}\exp[0.0861(T_{\text{w}} - T_{\text{a}})],
\end{equation}
%
where $U$ is the wind speed, $T_{\text{w}}$ is the water temperature and $T_{\text{a}}$ is the air temperature (this equation is in turn claimed to have been taken from a book by \citet{Monahan1986}). This kind of information has been obtained by doing \idxsp{satellite}{measurement}{s}. However, according to the \idxs{Beaufort}{scale}, crests don't begin to break until there is at least a gentle breeze, which starts at about \mbox{3.4 m/s}, while according to \eqref{eq:fractional_area_naiive}, there would always be some whitecaps area. As a remedy, one could subtract a small value
%
\begin{equation}
f_0 = f(U_0,\, T_{\text{a}},\, T_{\text{a}}+\Delta T_0)
\end{equation}
%
(where, tentatively, ${U_0 = 3.4\text{ m/s}}$ and ${\Delta T_0 = 0}$; note that $f_0$ is independent of the choice of $T_{\text{a}}$) and create a corrected estimate $f^*$ of the fractional area, defined as
%
\begin{equation}
f^* = max(f-f_0,\,0).
\end{equation}

Furthermore, \citeauthor{Premovze2001} suggest that "as a crude approximation to the true distribution, one can put whitecaps at positions on the surface where the amplitude of the waves is the largest". However, if following this method, the water surface would preferably first be high-pass filtered before determining on which parts of the surface the free surface elevation is the largest, to prevent that some (large) regions of the surface that would happen to be a bit higher elevated than other regions would get much more whitecaps area.

In a work by \citet{Monnier}, it is on the other hand the vertical downward acceleration that is used to determine where to put the whitecaps, instead of the free surface elevation. The \threshold for the acceleration is calculated dynamically to keep the total whitecaps area at the correct level. 

As it turns out, the larger the Lagrangian vertical downward acceleration is, the smaller the pressure gradient is. Without a pressure gradient, there will be nothing that keeps the water and the air separated, and the two fluids are much more prone to mix. So where the downward acceleration is large, the pressure gradient will be small, and water and air will naturally be more prone to mix, especially under large wind speeds where the \idxs{velocity}{shear} at the surface is large, which increases the chance of getting \idxs{Kelvin–Helmholtz}{instability} in the system.

Additionally to using a \threshold for some specific property of the surface to determine where the whitecaps are located, some methods also use a \idx{half-life} for the whitecaps that have transitioned to below the \threshold to model a more realistic destruction process for the whitecaps. This method is used in a few video on the web \citep{ozernik2009,cebasVT2010}.
