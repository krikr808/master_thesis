\chapter{Discussion}

Whenever a cell with $\alpha = 0$ doesn't border to any cell with $\alpha > 0$, it is not needed any longer.

The idea was to remove them, although this feature was never implemented. Although the advection scheme that is used is ideal in \idxs{one}{dimension}\index{dimensions!one|see{one dimension}}, it proved to smear the \idxs{$\alpha$}{field} out somewhat in \idxse{two}{dimensions}{two} and \idxs{three}{dimensions}, making the interface between water and vacuum become thicker and thicker, which affects both the \idxs{simulation}{quality}, since the surface becomes less and less well-defined as the simulation goes on, as well as the \preformance of the \idxs{numerical}{method} since it will have to consider more and more cells as the interface gets thicker and thicker. Another advection scheme that maybe could remedy this problem is \MULES, described in \citep{Berberovi2009} and further developed to for better handling of more than two phases in \citep{Kissling2010}, although it would have to be modified to cope with compressible flow.

\section{Difficulties and drawbacks with the method}

Difficulties:
\begin{itemize}
    \item Dynamical creation/termination of surface cells and determination of properties in new cells
    \item High Courant numbers
    \item High speeds of sound (remedied in \textit{\href{http://physbam.stanford.edu/~kwatra/papers/compressible_semi_implicit/compressible_semi_implicit.pdf}{A method for avoiding the acoustic time step restriction in compressible flow}})
    \item Keeping a sharp interface
    \item Making it work in realtime
\end{itemize}

\section{Conservation laws}

\begin{itemize}
    \item Mass
    \item Energy
    \item (d/d$t$)\,Momentum $-$ Force = 0
    \item Angular momentum
    \item (d/d$t$)\,(Center of mass) $-$ Momentum = 0
\end{itemize}

\section{Already existing software}

\begin{itemize}
    \item OpenFOAM (\red{FOAM, see} \textit{\href{http://powerlab.fsb.hr/ped/kturbo/openfoam/docs/foam.pdf}{A tensorial approach to computational continuum mechanics using object-oriented techniques}})
\end{itemize}

Reasons to use them: They have been tried out and they work, maybe not exactly for these purtposes, though. It saves time to use them instead of having to develop your own software. Reasons for developing you own software: You get full control of what you are doing, so you can adapt the software after your needs. You can approximate and simplify the things that are usually important but not for this purpose, so you can optimize the code for your purpose. You don't have to pay for any license; this otherwise have a tendency to become really expensive for a large company.

\section{Other methods to use}

For realtime simulation of surface waves in large bodies of water, where non-linear phenomena are not of any significant importance, the best method to use, considering both implementation time and simulation quality, is probably spectral methods.

\chapter{Improvements}

Except from these problems with the implementation that was made, it would need a lot of improvements in order to be applicable in a real-time flight simulator.

%\section{Adapdive Meshing}
\section{Adapdive Grid Refinement}

LODs that adapt to the camera placement and the Field of View (FOV)/Viewing frustum in order to limit the number of cells needed in the FVM

View dependent adaptive grid refinement

\section{Fluid--Structure Interactin}

see e.g.:

\begin{itemize}
    \item \textit{\href{http://physbam.stanford.edu/~fedkiw/papers/stanford2010-04.pdf}{Numerically Stable Fluid-Structure Interactions Between Compressible Flow and Solid Structures}}
    \item \textit{\href{http://efdl.as.ntu.edu.tw/research/papers/JCP03GCIBM.pdf}{A ghost-cell immersed boundary method for flow in complex geometry}}
    \item \textit{\href{http://www.cs.columbia.edu/~batty/papers/Batty07.pdf}{A Fast Variational Framework for Accurate Solid-Fluid Coupling}} (solid fraction, non-stick to walls)
\end{itemize}

\subsection{Immersed Boundary Method}

\begin{itemize}
    \item Reference: \textit{\href{http://www4.ncsu.edu/~zhilin/TEACHING/MA798Z/Peskin1.pdf}{The immersed boundary method}}
    \item For compressible flow: \textit{\href{http://www.cecs.wright.edu/~haibo.dong/wp-content/themes/publications/IBM_JCP_2007.pdf}{A sharp interface immersed boundary method for compressible viscous flows}}
\end{itemize}

\subsection{Volume of Solid Method (VOS)}

\begin{itemize}
    \item Reference: \textit{The simulation of fluid-rigid body interaction}
    \item Described in \textit{Numerical Modeling of Deforming Bubble Transport Related to Cavitating Hydraulic Turbines}
\end{itemize}

\subsection{Rigid Fluid method}

\begin{itemize}
    \item Reference: \textit{\href{http://www.amath.unc.edu/Faculty/mucha/Reprints/siggraph04.pdf}{Rigid Fluid: Animating the Interplay Between Rigid Bodies and Fluid}}
\end{itemize}

\subsection{Rotation of rigid bodies}

See \textit{\href{http://en.wikipedia.org/wiki/Euler\%27s_equations_\%28rigid_body_dynamics\%29}{Euler's equations (rigid body dynamics)}}

\section{Parallellization}

See e.g. \textit{\href{http://gfs.sourceforge.net/papers/agbaglah2011.pdf}{Parallel simulation of multiphase flows using octree adaptivity and the volume-of-fluid method}}

\subsection{Space filling curves}

See e.g. \textit{\href{http://j.teresco.org/research/publications/octpart02/octpart02.pdf}{Dynamic Octree Load Balancing Using Space-Filling Curves}}

and \textit{\href{http://downloads.isrn.com/journals/appmath/2012/246491.pdf}{Parallel Adaptive Mesh Refinement Combined with Additive Multigrid for the Efficient Solution of the Poisson Equation}}

\section{Unconditionaly Stable Flows}

\textit{\href{http://www.dgp.toronto.edu/people/stam/reality/Research/pdf/ns.pdf}{Stable Fluids}} (is it simple to implement with te \VOF method and is it still mass conservative after that?)

\SIMPLE (although not completely implicit (?))

Use implicit time-stepping

Switch from explicit to implicit methods + use semi-Lagrangian advection to allow arbitrarily large time steps

\section{Visualization}

\subsection{Marching Cubes}

\subsection{Surface shininess}

Shading/reflection (increased reflection spreading for increased height = shininess as a function of the part of the spectrum that contains the supressed high-frequency waves)

%\section{Local-time stepping}

%Very dubious technique as you can make the method allow large time steps. (see below)

\section{Perfectly matched layers}

Avoiding spurious reflections at LOD transitions (Absorbing boundary conditions, Perfectly Matched Layers)

See e.g. \textit{\href{http://liu.diva-portal.org/smash/get/diva2:359805/FULLTEXT01}{Memory Efficient Methods for Eulerian Free Surface Fluid Animation}}, which explains explicit dampening, implicit dampening, and wave absorbing boundaries -- the perfectly matched layer approach, and evaluates methods.

\section{Wind waves}

\subsection{Spectral methods}

See \textit{\href{http://graphics.ucsd.edu/courses/rendering/2005/jdewall/tessendorf.pdf}{Simulating Ocean Water}} for a modified Phillis spectrum (increased directional dependence)

\subsection{Air-water interaction}

\section{Visual effects}

\subsection{Splash and foam}

(See \textit{\href{http://en.wikipedia.org/wiki/Sea_foam}{Wikipedia -- Sea foam}} or search for \textit{protein skimming} or \textit{foam fractionation})

Implemented in \textit{\href{http://nguyendangbinh.org/Proceedings/Eurographics/2003/cgf/volume22/issue3/paper127/paper127.pdf}{Realistic Animation of Fluid with Splash and Foam}}

which references \textit{\href{http://citeseerx.ist.psu.edu/viewdoc/download?doi=10.1.1.4.6262&rep=rep1&type=pdf}{Rendering Natura lWaters}}

which in turn is based on work presented in the book \textit{\href{http://books.google.se/books?id=xuwFz1bPTHgC}{Oceanic Whitecaps: Their Role in Air-Sea Exchange Processes}} by E. C. Monahan and G. MacNiocaill.

See also \textit{\href{http://www.ias.ac.in/jess/sep2002/Ps18.pdf}{Oceanic whitecaps: Sea surface features detectable via satellite that are indicators of the magnitude of the air-sea gas transfer coefficient}} (also by  E. C. Monahan)

See also \textit{\href{http://cg.informatik.uni-freiburg.de/publications/2012_CGI_sprayFoamBubbles.pdf}{Unified Spray, Foam and Bubbles for Particle-Based Fluids}}

\subsection{Choppy waves}

horizontal offset

See \textit{\href{http://graphics.ucsd.edu/courses/rendering/2005/jdewall/tessendorf.pdf}{Simulating Ocean Water}}

\subsection{Turbulence}

See \textit{\href{http://publications.dice.se/attachments/water\%20interaction\%20ottosson_bjorn.pdf}{Real-time Interactive Water Waves}} (2.1.3) for a discussion of why that is important.

\section{Sharpening of various advected fields}

\subsection{Backward Error Compensation and Forward Error Correction}

\begin{itemize}
    \item Reference: \textit{\href{http://smartech.gatech.edu/xmlui/bitstream/handle/1853/29473/2002-389.pdf}{Back and forth error compensation and correction methods for removing errors induced by uneven gradients of the level set function}}
    \item Applied to the velocity field and images: \textit{\href{http://www.gvu.gatech.edu/~jarek/papers/FlowFixer.pdf}{FlowFixer: Using BFECC for Fluid Simulation}}
\end{itemize}

\section{Various optimizations}

* Pure code optimizations
* Local time stepping (dibious improvement)