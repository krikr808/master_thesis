\chapter{Free-Surface Modeling}
\label{chap:freesurfacemodeling}

A fluid flow with a free surface in the form of a \idxs{sharp}{interface} between two \idxs{immiscible}{fluids}, such as the flow of \water and \air, is called a \idxs{two-phase}{flow}. When simulating a two-phase flow, one of the most challenging tasks is \FSM, in which the surface in between the two phases is represented. There are several ways to model this surface --- all with their respective strengths and weaknesses --- of which a few of the most common are discussed in this chapter.

In this thesis work, a method known as the \VOF method has been used, together with a variant of an \idxs{advection}{scheme} known as \idxse{Hyper-C}{advection scheme}{Hyper-C}. The \VOF method and a few other methods for modeling the free surface will be covered briefly in this \levelname.

%http://physbam.stanford.edu/~fedkiw/papers/stanford2002-07.pdf (Robust Treatment of Interfaces for Fluid Flows and Computer Graphics)

%Lecture: http://www.ims.nus.edu.sg/Programs/fluiddynamic/files/Lecture1-basics.pdf (Moving Interface Problems: Methods \& Applications Tutorial Lecture I)

\section{Mesh based surface tracking methods}

In \idxsp{mesh based surface tracking}{method}{s}\index{surface tracking method!mesh based|see{mesh based surface tracking method}}\index{tracking method!surface!mesh based|see{mesh based surface tracking method}}, the surface is modeled as a grid that moves along with the flow through \idxs{Lagrangian}{advection}.

These methods results in a very precise way of representing the surface. Besides, \idxs{adapive}{mesh refinement}\index{refinement!mesh!adaptive|see{adaptive mesh refinement}} can be used to adjust the \idxs{level of}{detail} as needed, to provide a higher resolution for regions closer to the \camera or regions with more \turbulence, although as a rule of thumb, the size of the grid that models the surface should be of the same order of magnitude of the grid the fluids are represented on in the visible regions.

On the other hand, the constantly require remeshing, just like most methods involving meshes that are exposed to Lagrangian advection. Besides, it needs some way to determine when the mesh should split, some way to detect mesh intersections, as well as some way to sew the different parts of the mesh together after they have intersected \citep{Wojtan2009}.

\section{Level Set method}

The \LS method, which was first described by \citet{Osher1988}, is a general method for tracking interfaces and shapes and is one of the most common ways to track interfaces between two immiscible fluids. In the \LS method, a \idxs{signed distance}{function}\indexs{signed}{distance function}, also knows as \idxs{level set}{function}, $\phi$ is used to keep track of the distance to the surface. The absolute value of $\phi$ represents the distance to the interface, whereas the sign of $\phi$ designates the \phase. If the two phases are air and water, $\phi \leq 0$ usually designates water and $\phi > 0$ designates air, so that $\phi$ can be interpreted as the "height" when the system is in its \equilibrium position. The surface is then defined as the \isosurface, or \idxs{level}{set}, where $\phi = 0$, that is, as
%
\begin{equation} \label{eq:level_set}
L_c(\phi) \,=\, \{\vec{r} \,\mid\, \phi(\vec{r})=c\}
\end{equation}
%
where $c$ is a distinct level of $\phi$ and $L_c(\phi)$ is the set of all locations $\vec{r}$ where $\phi(\vec{r}) = c$. In the \LS method, $c = 0$ is used to represent the surface.

In each \idxs{time}{step}, $\phi$ is transported according to the equation
%
\begin{equation} \label{eq:level_set_function_transport}
\frac{\partial\phi}{\partial t} + \vec{u}\cdot\nabla\phi \,=\, 0.
\end{equation}

One of the greatest strengths with this method, besides from being simplie, is its handiness when it comes to \idx{post-processing}. It is fairly easy to visualize an \isosurface of a field that is \discretized into the \nodes of \idxsp{cell}{corner}{s}, and for that reason the \LS method has become very popular in \FVM simulations where the possibility to quickly \render a realistic image of the surface is essential \citep{Losasso2004,Chentanez2011}. The \idxs{marching}{cubes}\index{algorithm!marching cubes|see{marching cubes}} algorithm, which was first described by \citet{Lorensen1987}, is used to create triangle models of isosurfaces from a scalar field in a very straightforward manner, by marching through a set of adjacent cubes covering the \idxs{rendering}{domain}, and by looking at the value of the function in the \idxe{cell corner}{corners} of the cubes. In a work by \citet{Sethian1995}, a class of fast marching methods related to the \LS method was developed. The marching cubes algorithm itself is described in greater detail in \secref{sec:marchingcubesdescription}.

When $\phi$ has become discretized, \eqref{eq:level_set} cannot be used directly to define the interface since it is not known exactly where the isosurface defined by $\phi = 0$ is located. However, the marching cubes algorithm produces a good approximation from the data that is available for where the interface is located. It has therefore, together with the \LS method, been used in numerous works that uses the \FVM, in order to visualize free surfaces \citep[e.g.][]{Losasso2004}.

On the other hand, probably the greatest weakness with this method is that, after $\phi$ and \eqref{eq:level_set_function_transport} have been \discretized, it doesn't conserve mass perfectly. Advection will cause $\phi$ to be smeared out, and most of the surface details are washed away \citep{Wojtan2009}. This problem is more apparent when the simulation contains thin features which have a tendency to disappear or start to flicker \citep{nthuerey2009}. However, this problem can be remedied somewhat (although not completely) by using \BFECC, which was first described by \citet{Dupont2003}; this method has been tested both in advection of the velocity field as well as on images for illustration purposes \citep{Kim2005}.

\section{Volume of Fluid method}

\label{sec:vof}

Another very commonly used way of tracking interfaces between two immiscible fluids is the \VOF method, first described by \citet{Hirt1981}. For a non-discretized \PDE, the \VOF method uses a \idxs{phase}{fraction} $\alpha^*$ that is either $0$ or $1$ to designate the \phase. If the two phases are air and water, $\alpha^* = 0$ usually designates air and $\alpha^* = 1$ designates water, so that for a non-discretized \PDE, $\alpha^*$ can be expressed as
%
\begin{equation} \label{eq:phase_fraction_continuous}
\alpha^*(\vec{r}) \,=\, \lim_{\epsilon\,\rightarrow\,0} \frac{V_{\epsilon,\text{w}}}{V_{\epsilon,\text{t}}}\,(\vec{r}),
\end{equation}
%
where $V_{\epsilon,\text{t}}$ is the volume of the sphere with radius $\epsilon$, centered in $\vec{r}$, and $V_{\epsilon,\text{w}}$ is the volume of the water contained within $V_{\epsilon,\text{t}}$. This fraction can only result in 0 or 1, since $\vec{r}$ cannot be partially emerged in water, except from in the case where $\vec{r}$ is located on the interface where $\alpha^*$ \idxse{discontinuous}{jump}{jumps discontinuously} from 0 to 1 or from 1 to 0 and hence doesn't have a well defined value. The interface using the non-discretized phase fraction, $I^*(\alpha^*)$, can therefore be defined as all locations where a transition takes place such that $\alpha^*$ makes this kind of jump, or in other words
%
\begin{equation} \label{eq:vof_interface_continuous}
I^*(\alpha^*) \,=\, \{\vec{r} \,\mid\, \cancel{\exists}\,\alpha(\vec{r})\in\reals\},
\end{equation}
%
where $\reals$ is the set of \idxsp{real}{number}{s}. However, in this thesis work, $\alpha^*$ has been \discretized into $\alpha$ which is a sequence of \idxse{cell}{property}{cell properties} rather than a continuous function of location and which, because of its discrete nature, is unable to perfectly describe how the two phases are distributed. The value of $\alpha$ in a cell takes the whole cell into account and provides a value of how big ratio of the cell that is filled with water, rather than only looking at one point as is the case with other discretized properties. The definition of $\alpha$ is therefore the sequence of
%
\begin{equation} \label{eq:phase_fraction_discretized}
\alpha_i(C_i) \,=\, \frac{V_{i,\text{w}}}{V_{i,\text{t}}}\,,
\end{equation}
%
where $\alpha_i$ is the value of $\alpha$ in $C_i$ which is the cell with index $i$, $V_{i,\text{t}}$ is the total volume of $C_i$ and $V_{i,\text{w}}$ is the volume of the water contained within $V_{i,\text{t}}$. In contrast to in the non-discretized case, this fraction will never be undefined, so substituting $\alpha$ for $\alpha^*$ in \eqref{eq:vof_interface_continuous} and use that for the definition of the surface will not work. On the other hand, since the cells the interface cross through will be partially filled with water, $\alpha$ will be somewhere in between 0 and 1. We can therefore define the interface using the discretized phase fraction as
%
\begin{equation} \label{eq:vof_interface_discrete}
I(\alpha) \,=\, \{C_i \,\mid\, 0 < \alpha_i < 1\}.
\end{equation}

Note that this definition of the interface will not result in a surface but in a set of cells, which is a volume. The interface will therefore get a thickness, and we can conclude that it will never be perfectly sharp as long as there are cells containing a mix of both phases.

In contrast to the \LS method, for which it is almost given that the interface is going to be constructed from the discretized signed distance function by using the marching cubes algorithm, when using the \VOF method it is a non-trivial task to construct the interface from $\alpha$. However, in many cases when the method is used, that is not necessary. Besides, \VOF methods are rarely used in computer graphics applications because of \idxse{flotsam}{artifact}{flotsam} and \idxse{jetsam}{artifact}{jetsam artifacts} \citep{Wojtan2009}. However, there do exist various methods for constructing an interface from $\alpha$, of which most are used mainly to assist in the advection of $\alpha$ in order to keep the interface sharp (see \secref{sec:geometric_advection_schemes}), but those are not very useful when visualizing the surface.

%\begin{itemize}
%    \item Comparsion: http://capfluidicslit.mme.pdx.edu/reference/Numerics/Gopala_ChemEngJ2008_VOFMethodsFreeSurfaceFlow.pdf (Volume of fluid methods for immiscible-fluid and free-surface flows)
%\end{itemize}

%\subsection{VOF vs. Pseudo VOF}

%\begin{itemize}
%    \item Explanation: http://www.flow3d.com/cfd-101/cfd-101-VOF.html (VOF (Volume of Fluid) - What's in a Name?)
%\end{itemize}

%\subsection{Interface reconstruction}

\section{Coupled Level Set/Volume of Fluid method}

The \LS method has problems to conserve the mass, and the \VOF method has difficulties with keeping the interface sharp without the use of advanced \idxsp{advection}{scheme}{s}, as well as being inconvenient when \visualizing the surface. Therefore, hybrid methods called \CLSVOF methods have been developed, which aim to combine the strengths of both the \LS method and the \VOF method and overcome their weaknesses by using both a \idxs{phase}{fraction} field and a \idxs{signed distance}{function}. The \CLSVOF method was first described by \citet{Puckett1998} and has later been noted to generally be superior to either the \LS or the \VOF method alone \citep{Sussman2000}.