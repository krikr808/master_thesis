\chapter{Free Surface Modelling (FSM)}

A fluid flow with a free surface in the form of a \idxs{sharp}{interface} between two \idxs{immiscible}{fluids}, such as the flow of \water and \air, is called a \idxs{two-phase}{flow}. When simulating a two-phase flow, one of the most challenging things is to represent the surface in a satisfying way. There are several ways to model the surface, all with their respective strengths and weaknesses.

In \thiswork, a method called the volume of fluid method has been used, together with a variant of an advection scheme called CICSAM. These and a few other methods for modelling the free surface will be covered briefly in this \levelname.


\HRule

There are both surface trackibng methods and surface capturing methods.

See also \textit{\href{http://physbam.stanford.edu/~fedkiw/papers/cam1998-17.pdf}{A Non-Oscillatory Eulerian Approach to Interfaces in Multimaterial Flows (The Ghost Fluid Method)}}

and \textit{\href{http://physbam.stanford.edu/~fedkiw/papers/stanford2002-07.pdf}{Robust Treatment of Interfaces for Fluid Flows and Computer Graphics}}

\begin{itemize}
    \item Lecture: \textit{\href{http://www.ims.nus.edu.sg/Programs/fluiddynamic/files/Lecture1-basics.pdf}{Moving Interface Problems: Methods \& Applications Tutorial Lecture I}}
\end{itemize}

\section{Mesh based surface tracking methods}

\begin{itemize}
    \item See e.g. \textit{\href{http://www.cc.gatech.edu/~turk/my_papers/thin_fluid_features.pdf}{Physics-Inspired Topology Changes for Thin Fluid Features}}
\end{itemize}

\section{Level Set method}

One of the most common ways to track interfaces between two immiscible fluids is the \LS method, which is first presented in \citep{Osher1988}. In the \LS method, a \idxs{signed distance}{function}\indexs{signed}{distance function}, also called a \idxs{level set}{function}, $\phi$ is used to keep track of the distance to the surface. The absolute value of $\phi$ represents the distance to the interface, whereas the sign of $\phi$ designates the \phase. If the two phases are air and water, $\phi \leq 0$ usually designates water and $\phi > 0$ designates air, so that $\phi$ can be interpreted as the "height" when the system is in its \equilibrium position. The surface is then defined as the \isosurface, or \idxs{level}{set}, where $\phi = 0$, that is, as

\begin{equation} \label{eq:level_set}
L_c(\phi) \,=\, \{\vec{r} \,\mid\, \phi(\vec{r})=c\}
\end{equation}

where $L_c(\phi)$ is the set of all locations $\vec{r}$ where $\phi(\vec{r}) = c$. In the \LS method, $c = 0$ is used to represent the surface.

In each \idxs{time}{step}, $\phi$ is transported according to the equation

\begin{equation} \label{eq:level_set_function_transport}
\frac{\partial\phi}{\partial t} + \vec{u}\cdot\nabla\phi \,=\, 0.
\end{equation}

One weakness with this method is that it doesn't conserve mass perfectly. This is more apparent when the simulation contains thin features, which, as noted e.g. in \citep{Wojtan2009}, have a tendency to disappear.

\section{Volume of fluid method}

Another, very commonly used way of tracking interfaces between two immiscible fluids is the \VOF method, first presented in \citep{Hirt1981}. For a non-discretized \PDE, the \VOF method uses a \idxs{phase}{fraction} $\alpha$ that is either $0$ or $1$ to designate the \phase. If the two phases are air and water, $\alpha = 0$ usually designates air and $\alpha = 1$ designates water, so that $\alpha$ can be expressed as

\begin{equation} \label{eq:phase_fraction_}
\alpha(\vec{r}) \,=\, \frac{\opd V_{\text{w}}}{\opd V_{\text{t}}}(\vec{r}),
\end{equation}

where $\opd V_{\text{t}}$ is an infinitesimal volume element located at $\vec{r}$ and $\opd V_{\text{w}}$ is the volume of the water contained in $\opd V_{\text{t}}$. The interface $I(\alpha)$ is then defined as all locations where a transition takes place such that $\alpha$ \jumps (discontinuously) from 0 to 1 or from 1 to 0. This means that the divergence of $\alpha$ is undefined, that is 

\begin{equation} \label{eq:vof_interface}
I(\alpha) \,=\, \{\vec{r} \,\mid\, \cancel{\exists}\nabla\alpha(\vec{r})\},
\end{equation}

which is in fact an equivalent condition.

However, for a  

In contrast to with the \LS method, there are also phases

\begin{itemize}
    \item Reference: \textit{\href{http://pages.csam.montclair.edu/~yecko/icodes/HirtNichols_Surfer_JCP1981.pdf}{Volume of Fluid (VOF) Method for the Dynamics of Free Boundaries}}
    \item Comparsion: \textit{\href{http://capfluidicslit.mme.pdx.edu/reference/Numerics/Gopala_ChemEngJ2008_VOFMethodsFreeSurfaceFlow.pdf}{Volume of fluid methods for immiscible-fluid and free-surface flows}}
\end{itemize}

\subsection{VOF vs. Pseudo VOF}

\begin{itemize}
    \item Explanation: \textit{\href{http://www.flow3d.com/cfd-101/cfd-101-VOF.html}{VOF (Volume of Fluid) - What's in a Name?}}
\end{itemize}

\subsection{Interface reconstruction}
%\section{Internal alpha distribution}

\subsection{Smearing during advection}

\subsection{Geometric advection schemes}

\begin{itemize}
    \item A simple (at least so it seems) scheme: \textit{\href{http://www.lmm.jussieu.fr/~zaleski/nota02.pdf}{A geometrical area-preserving Volume-of-Fluid advection method}}
\end{itemize}

\subsection{Algebraic advection schemes}

\subsubsection{Convection Boundedness Criterion (CBC)}

\sloppy
\begin{itemize}
    \item Reference: \textit{Curvature-compensated Convective Transport: SMART a New Boundedness- Preserving Transport Algorithm}
    \item Extended Convective Boundedness Criterion (ECBC): \textit{Discussion on Numerical Stability and Boundedness of Convective Discretized Scheme}
    \item General Convective Boundedness Criterion (GCBC): \textit{\href{http://gr.xjtu.edu.cn:8080/upload/PUB.1673.4/Wei_NHT.pdf}{A New General Convective Boundedness Criterion}}
    \item \red{Convection Boundedness Criterion for arbitrarily unstructured meshes:} \textit{\href{http://powerlab.fsb.hr/ped/kturbo/openfoam/papers/GammaPaper.pdf}{High resolution NVD differencing scheme for arbitrarily unstructured meshes}}
\end{itemize}
\fussy

More:
\begin{itemize}
    \item Normalised Variable Diagram (NVD)
    \item \textit{\href{http://warminski.pollub.plwww.ptmts.org.pl/Waclaw-Koron-2-08.pdf}{Comparison of CICSAM and HRIC High-resolution Schemes for Interface Capturing}}
    \item \textit{\href{http://proceedings.fyper.com/eccomascfd2006/documents/85.pdf}{MODELING OF THE WAVE BREAKING WITH CICSAM AND HRIC HIGH-RESOLUTION SCHEMES}}
\end{itemize}

\subsubsection{Multidimensional Universal Limiter with Explicit Solution (MULES)}

References: \citep{Berberovi2009,Kissling2010}

See \textit{OpenFOAM-1.5.x/src/finiteVolume/fvMatrices/solvers/MULES/MULES.H} for details

See also \href{http://fds.duke.edu/db/aas/Physics/weller}{Henry R Weller, Professor Emeritus}

% Escape characters
%\& \% \$ \# \_ \{ \}
%\textasciitilde  = ~
%\textasciicircum = ^
%\textbackslash   = \

\begin{itemize}
    \item Described here: \textit{\href{http://link.libris.kb.se/sfxliub?sid=?url_ver=Z39.88-2004&rfr_id=info:sid/bibl.liu.se\%3Axerxes+\%28+PubMed+LiU\%29&rft.genre=article&rft_val_fmt=info\%3Aofi\%2Ffmt\%3Akev\%3Amtx\%3Ajournal&rft.issn=15393755&rft.date=2009&rft.jtitle=Phys+Rev+E+Stat+Nonlin+Soft+Matter+Phys&rft.volume=79&rft.issue=3+Pt+2&rft.spage=036306&rft.atitle=Drop+impact+onto+a+liquid+layer+of+finite+thickness+\%3A+dynamics+of+the+cavity+evolution+&rft.aulast=Berberovi\%C4\%87&rft.aufirst=Edin}{Drop impact onto a liquid layer of finite thickness: Dynamics of the cavity evolution}}
    \item An improvement for more than two phases: \textit{\href{http://www.mathematik.uni-ulm.de/numerik/staff/urban/reports/ECCOMASCFD2010paperfinal.pdf}{A Coupled Pressure Based Solution Algorithm Based on the Volume-Of-Fluid Approach for Two or More Immiscible Fluids}}
\end{itemize}

\subsubsection{SOLA-VOF}

\begin{itemize}
    \item Reference: \textit{\href{http://www.ewp.rpi.edu/hartford/~ernesto/Su2012/CFD/Readings/SOLA-VOF-1980-P1.pdf}{SOLA-VOF: A Solution Algorithm for Transient Fluid Flow with Multiple Free Boundaries}}
\end{itemize}

\subsubsection{Hyper-C flux limiter}

\begin{itemize}
    \item Reference: \textit{\href{http://www.water.tkk.fi/wr/kurssit/Yhd-12.112/TVD1.pdf}{The Ultimate Conservative Difference Scheme Applied to Unsteady One-Dimensional Advection}}
\end{itemize}

\paragraph{Floating mixed cells}

\begin{itemize}
    \item Remedy: \textit{\href{https://e-reports-ext.llnl.gov/pdf/245038.pdf}{A Simple Advection Scheme for Material Interface}}
\end{itemize}

\subsubsection{Compressive Interface Capturing Scheme for Arbitrary Meshes (CICSAM)}

\begin{itemize}
    \item Reference: \textit{\href{http://ac.els-cdn.com/S0021999199962769/1-s2.0-S0021999199962769-main.pdf?_tid=85161b57da5f4401e55c9d07495e24ea&acdnat=1336167249_a59e4f578adbacf3bff69936c48cdd57}{A Method for Capturing Sharp Fluid Interfaces on Arbitrary Meshes}}
    \item Also described in (by the same author): \textit{\href{http://powerlab.fsb.hr/ped/kturbo/OpenFOAM/docs/OnnoUbbinkPhD.pdf}{Numerical prediction of two fluid systems with sharp interfaces}}
    \item Test with different Courant numbers: \textit{\href{http://www.marin.nl/upload_mm/8/2/c/1807524470_1999999096_2007-ECCOMAS_HoekstraVazAbeilBunnik.pdf}{Free Surface Flow Modelling with Interface Capturing Techniques}}
    \item Improvement 1: \textit{\href{http://powerlab.fsb.hr/ped/kturbo/openfoam/docs/HenrikRuschePhD2002.pdf}{Computational Fluid Dynamics of Dispersed Two-Phase Flows at High Phase Fractions}}
\end{itemize}

\subsubsection{High Resolution Interface Capturing (HRIC) scheme}

\begin{itemize}
    \item Described here: \textit{\href{http://warminski.pollub.plwww.ptmts.org.pl/Waclaw-Koron-2-08.pdf}{Comparison of CICSAM and HRIC High-resolution Sche\-mes for Interface Capturing}}
\end{itemize}

\subsubsection{Switching Technique for Advection and Capturing of Surfaces scheme (STACS)}

\begin{itemize}
    \item Reference: \textit{\href{http://webfea-lb.fea.aub.edu.lb/cfd/pdfs/publications2/STACS-Complete.pdf}{Convective Schemes for Capturing Interfaces of Free-Surface Flows on Unstructured Grids}}
\end{itemize}

\subsubsection{Inter-Gamma Scheme}

\begin{itemize}
    \item Reference: \textit{\href{http://powerlab.fsb.hr/ped/kturbo/openfoam/docs/InterTrack.pdf}{Interface Tracking Capabilities of the Inter-Gamma Differencing Scheme}}
\end{itemize}

\subsubsection{Constrained Interpolation Profile (CIP) method}

%TODO: Used for advecting fluid interfaces?? At least apparently very good for simple advection.

\begin{itemize}
    \item Reference: \textit{\href{http://www.mech.titech.ac.jp/~ryuutai/paper/JCP2001CIPReviewYabe.pdf}{The Constrained Interpolation Profile Method for Multiphase Analysis}}
\end{itemize}

\subsection{Advection schemes for compressible water}

\begin{itemize}
    \item Remedy: Advect both water volume and total volume and then define alpha as the ration between them
\end{itemize}

\subsubsection{Fast Compressive Surface Capturing Formulation (FCSCF)}

\begin{itemize}
    \item Reference: \textit{\href{http://researchspace.csir.co.za/dspace/bitstream/10204/5282/1/Heyns_2011.pdf}{Free-Surface Modelling Technology for Compressible and Violent Flows}}
\end{itemize}

\section{Coupled Level Set/Volume of Fluid  method}

\begin{itemize}
    \item Reference: \textit{\href{http://pages.csam.montclair.edu/~yecko/icodes/SussmanPuckett_LevelSetVOF.pdf}{A Coupled Level Set and Volume-of-Fluid Method for Computing 3D and Axisymmetric Incompressible Two-Phase Flows}}
\end{itemize}