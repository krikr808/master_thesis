\chapter{Conclusions}

Training in flight simulators would benefit from employing a couled simulation of water waves and ship movement. The surface wave simulation is required to run in real-time, have a resolution that is high enough for the surface to look realistic from the pilots point of view, and manage wave dispersion in at least an approximate way as well as \FSI. Probably the only method of those that have been studied in \thismasterthesiswork, that today easily can be constructed to meet all these requirements, is the two-dimensional method using \LPD. Although Fourier synthesis is often used in computer graphics animations to quickly create realistic surface waves, the method is not suitable for modelling \FSI.

Alternativelly, a two-dimensional simulation using \LPD or Fourier synthesis can be coupled with a local three-dimensional fluid simulation around each ship to allow for strong non-linear phenomenas such as splashes and a more natural \FSI, as well as a fast simulation of \idxs{ambient}{waves}.

The \FVM on an octree with \FSM is an advanced method and takes long time to implement, probably a few years (or at least much longer than \masterthesisworktime), ans to make it ideal for the purposes of this master thesis work. However, when it has been fully implemented, basically its only bottleneck is the low speed.

The \FVM on an octree with \FSM runs in linear time to the number of grid points visible on the surface, which places the method among those with the lowest time complexity, but has a very high time constant associated with the big O notation which as of today makes it too slow to be used in real-time applications. However, this constant can be leviated by parallelizing the code and running it on multiple cores. And with the continuous increase in processor power it is only a matter of time before the the method can run in real-time, and it will come to compete with \LPD and eventually even surpass it as it has more to offer.