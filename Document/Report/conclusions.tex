\chapter{Conclusions}
\label{chap:conclusions}

The training in a flight simulator would become more valuable if a realistic, coupled simulation of water waves and ship movement was integrated into the flight simulator. It is therefore required that the wave simulation runs in real-time, has a resolution high enough to represent waves that normally roam the water surfaces and affect the ships as well as the wake lines that sailing ships give rise to. It should also manages to simulate wave dispersion, in at least an approximate way, as well as some form of \FSI to get the coupling between ship and waves.

The method that was implemented in \thisprojectwork was the \FVM on an octree with \FSM, which is an advanced method that takes a long time to implement and to make ideal for the purposes of \thismasterthesiswork \comment{this master thesis work}\nspace, and the implementation would still need a lot of improvement if it was to be applied in a real-time simulator.

The \FVM on an octree with \FSM runs in $O(N)$ time, where $N$ is the number of grid points visible on the surface, which places it among the methods with the lowest time complexity. However, it has a very high time constant associated with the big O notation which, as of today, makes it too slow to be used in real-time applications.

Probably the only one of the methods that were studied in \thismasterthesiswork, that today easily can be constructed to meet all the required properties, is the two-dimensional method using \LPD. Although Fourier synthesis often is used in real-time computer graphics applications for the creation of surface waves, the method is not suitable for modeling \FSI.

Alternatively, a two-dimensional simulation, using \LPD or Fourier synthesis, can be coupled with a local three-dimensional fluid simulation around each ship to allow for strong non-linear phenomenas such as splashes and more natural \FSI, as well as a fast simulation of \idxs{ambient}{waves}, proveded the three-dimensional method is fas enough.

However, it is believed that, after some improvement of the method that was implemented in \thisprojectwork, basically its only bottleneck would be its low speed, which to its defense can be alleviated by parallelizing the code and spread the computational load on many \CPUs and possibly even on \GPUs. And with the continuous increase in processor power it is only a matter of time before the method can run in real-time with an acceptable quality. Hence, it will in time come to compete with methods such as Fourier synthesis and the \LPD method, and will eventually even surpass them even for real-time simulations, as it models the behavior of water in a more realistic way which, intrinsically,  means that its potential of what it can achieve is higher.