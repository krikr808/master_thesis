\nonumchapter{Outline of thesis}

In \partref{part:introduction}, the background of the work in this thesis work will be presented. In 
\chapref{chap:motivation}, the problem is formulated and the motivation for the work done in the thesis is presented. In \chapref{chap:requirementsanddifficulties}, some of the difficulties that are faced when solving the problem are explained. And in \chapref{chap:relatedwork}, some of the extensive amount of works that have been done to solve related problems are briefly discussed.

In \partref{part:theoreticalbackground}, the theoretical foundation the work presented in this report builds on is described. In \chapref{chap:thefinitevolumemethod}, the finite volume method, which is the core of the method used in this thesis work, is described; for the Poisson equation, which arises and has to be solved in order to obtain the pressure when flow is incompressible or when the speed of sound is high, a few different solution methods are discussed; and a method that allows this equation to be only approximately solved, while still preserving mass, is presented. In \chapref{chap:octrees}, the octree, which is the framework in which the finite volume method operates in this thesis work, is described, along with the closely connected level of detail concept. In \chapref{chap:freesurfacemodeling}, a few methods used for free-surface modeling, which is necessary if the simulation is supposed to contain two or more immiscible fluids, are discussed. In \chapref{chap:advectionofproperties}, different advection schemes, which describe how scalar or vector fields are transported as the fluids move, are discussed. Finally, in \chapref{chap:methodsummary} there is a brief summary of all methods used in the thesis work, in case that was not obvious from the previous chapters.

In \partref{part:analysis} the thesis work is analyzed; \chapref{chap:results} contains some of its notable results, \chapref{chap:discussion} contains a discussion about these results and the work in general, \chapref{chap:improvements} discusses a number of improvements that would be necessary or at least highly desired if the method used in the thesis work was actually to be used in a flight simulator, and \chapref{chap:conclusions} contains some final conclusions and the most important things to remember from this report.

\partref{part:appendices} contains the attached appendices. In \appref{apdx:algorithmsanddatastructures}, a couple of the data structures used in this thesis work is described in greater detail, and an algorithm for transporting the velocity field, that preserves momentum and doesn't introduce unnecessary smearing is presented.

In \appref{apdx:illumination_model_derivation}, a simple, physically based illumination model for the rendering of water surfaces is derived and compared to the Blinn--Phong shading model. While the Blinn--Phong shading model is at least empirically based, while the illumination model presented here is purely theoretical, it is concluded that the shape of the specular highlights are very similar between the two models.

In \appref{apdx:pde_derivation}, a two-dimensional partial differential equation in the spatial domain, describing the time evolution of surface waves for intermediate, mildly varying water depths is presented. Although the method would be capable of running with the time complexity $O(N)$ per frame, it is still deemed to be too slow to be used in real-time simulations of water waves, and its behavior is unknown.