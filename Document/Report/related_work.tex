\chapter{Related work}

%TODO: Why is CFD needed for this thesis work? (maybe this should go under motivation)

\CFD is a well established area of research, and a large number of widely differing methods have been developed during the years. This \levelname will present and briefly discuss some of the most common methods for modeling and simulating fluids, and how suitable they are for \thisprojectwork.

\section{Two-dimensional methods}

\idxse{two-dimensional}{method}{Two-dimensional methods} for simulating water waves, sometimes also called 2.5 methods, are the easiest to understand and to implement and are often the fastest when it comes to simulation speed, but are not completely realistic models and therefore can't simulate all effects that can be simulated with a \idxs{three-dimensional}{method}, such as splashes or wave breaking, although attempts to extend such methods to cope with wave breaking have been made, such as in \citep{Miklos2009}. They can also sometimes be quite unintuitive when it comes to modeling \FSI.

\subsection{Two-dimensional PDEs for shallow water}

There exist a number of different two-dimensional \PDEs which describe the evolution of the \idx{free surface elevation}, $\eta$. The ordinary wave equation, \eqref{eq:wave_equation}, will as concluded work badly for simulating surface waves in \idxs{deep}{water}, since it doesn't handle wave dispersion at all, which quickly becomes obvious when wave patters such as the \idxs{Kelvin}{wave pattern} has to be simulated.

On the other hand, the wave equation does work for simulating very weak, low-frequent waves in \idxs{shallow}{water} with constant or very \idxs{mildly varying}{depth}, since those are close to linear, and are almost not affected by wave dispersion at all thanks to the low water depth. For waves with higher amplitude, there exists the \idxs{Boussinesq}{equations} and the \idxs{shallow water}{equations}, which both can be used to simulate large, non-breaking waves in shallow waters. Therefore, these equations are commonly used in \SWS. Shallow water equations are usually used to simulate waves whose \wavelength is similar to the overall \idxs{water}{height}, as noted in \citep{Thurey2006}. These equations are often easily extendable to handle \FSI.

Most of these method ensure the time complexity $O(N)$ per time step, where $N$ is the number of \idxs{surface}{grid points}.

In \appref{chap:pde_derivation}, a set of new two-dimensional \PDEs intended for simulation of water waves at varying, arbitrary water depths are derived and discussed.

\subsection{Fourier Synthesis}
%\subsection{Spectral methods}

This method, which was first presented in \citep{Mastin1987}, builds on \idxsp{Fourier}{transform}{ing} a representation of the surface in the frequency domain and has, according to \citep{Monnier}, been extensively used and described in the world of computer graphics. This method operates in the \idxs{frequency}{domain}, and uses \FFT, which here is the most time consuming process, to calculate the free surface elevation. It is characterized by high speed and doesn't (in contrast to commonly used two-dimensional \PDEs) have any problem with \idxs{wave}{dispersion}.

However, this method has been noted to be incompatible with \FSI, as in \citep{Chentanez2011a}. Besides, it requires a constant water depth, and hence cannot simulate \idxs{wave}{shoaling}. This issue becomes noticeable when animating the surface water close to the \idxs{shore}{line}, where the water is shallow and waves naturally behave differently than what they do on deep water.

This method ensures the time complexity $O(N\,\log(N))$ per time step for trivial grid setups, where $N$ is the number of \idxs{surface}{grid points}.

\subsection{Laplacian Pyramid Decomposition}

A method that also handles \idxs{wave}{dispersion} well, and operates in the \idxs{spatial}{domain}, was presented in \citep{Ottosson2011} and uses \LPD to decomposes the free surface elevation. This means that different parts of the wave spectrum is stored one different \grids, where each next grid has half the \resolution of the previous grid. Each grid therefore contains a unique idxsp{band-pass}{filter}{ing} of the free surface elevation, or in other words, waves with different speeds are simulated on different grids. It then becomes possible to \approximate the dispersion very well by concentrating on the part of the spectrum associated with each grid and by expressing the second time derivative of the band-pass filtered free surface elevation, $\eta_i$, beloning to the $i$th grid, as a low order \polynomial of the \idxs{Laplace}{operator}, $\nabla^2$, operating on $\eta_i$. The non-filtered free surface elevation is then composed according to

\begin{equation} \label{eq:laplacian_pyramid_composition}
\eta \,=\, \sum_i \eta_i
\end{equation}

which also serves to define $\eta$.

This method is easily extendable to handle \FSI, and ensures the time complexity $O(N)$ per time step, where $N$ is the number of \idxs{surface}{grid points} in the grid with the highest resolution. If the \idxs{level of}{detail} depends on the location on the surface, why the grid with the highest resolution would not stretch over the entire simulated region, $N$ is the number of grid points located on a grid and on a location for which there are no grid with higher resolution, stretching over the same region.

\section{Three-dimensional methods}

\idxse{three-dimensional}{method}{Three-dimensional methods} are often highly realistic in the sense that they will be able to simulate all different kind of phenomenas that can be described with the \idxs{Navier--Stokes}{equations} (see \secref{sec:ns_equations}). There are a few exceptions though.

\subsection{Smoothed-Particle Hydrodynamics}

The \SPH method is a highly realistic model that simulates a flow by simulating a large number of small particles. Between each pair of particles that are within a certain \idxs{cut-off}{distance} from each other, there is a repelling or attractive force, described by an \ODE. The interaction between two particles in the simulation is usually modeled by a potential like those used in \MD, for example a \LJ potential that has been softened to limit the maximum acceleration that particles can be exposed to. The cut-off distance is used in order to ensure that the number of interactions is $O(N)$, an not $O(N^2)$ as for a system where all pair of particles interact with each other, where $N$ is the number of particles in the system.

There are a few major advantages with using this method. When the \idxs{Eulerian}{specification of the flow field} is used to describe the fluid motion, the equations tend to become more complicated as they contain \idxsp{advection}{term}{s}. In \SPH on the other hand, the \idxs{Lagrangian}{specification of the flow field} is used and no advection terms are therefore present in the equations which makes the model relatively simple, and it is easily implemented. Besides, no advection of fields with an Eulerian representation is simulated, which prevents additional problems that can arise during the advection, and conservation of various properties, like \momentum and \energy is usually automatically well preserved as a result of that. In fluid simulations, there is no need to model the air, and there is no need to keep track of where the surface of the fluid is located since this is information that can be extracted during the \idx{post-processing} phase.

On the other hand, the \SPH method requires that the entire simulation domain is filled with small particles, which often means that an \emph{extremely} large number of particles, proportionally to the volume of the fluid, have to be simulated. This implies a very heavy workload on the computer, and as a result of that, \SPH is very seldom used in \idxsp{real-time}{simulation}{s}. However, adaptive particle sizes have been used in order to reduce the amount of particles needed in the less important parts of the fluid, like in the bulk, in order to remedy this problem, a technique that was first used in \citep{Desbrun1999} and later improved in a number of reports, for example in \citep{Yan2009}.

For a grid with a random access time complexity of $O(1)$ and for particles that all have the same size, this method ensures the time complexity $O(N)$ per time step, where $N$ is the number of particles.

\subsection{Finite Volume Method on a restricted tall cell grid}

The \FVM is a highly realistic model that solves a set of \PDEs by dividing the region of interest into small \idxsp{volume}{element}{s}, and by discretizing the fields that are described by the \PDEs into points in the volume elements or on the border of the volume elements, usually a fixed number of points per volume element, as well as discretizing the \PDEs into a number of \ODEs describing the evolution of the discretized fields. The volume elements are commonly referred to as \cells. The \FVM and its usage in \CFD is described in greater detail in \chapref{chap:ns_equations}.

The \FVM ensures the time complexity $O(N)$ per time step, where $N$ is the number of cells.

The \FVM was first used on a restricted tall cell grid in \citep{Chentanez2011}. This approach uses an orthogonal grid, where the water closest to the water surface are modeled with small \idxsp{cubic}{cell}{s}, and the water deeper down is modeled with tall cells that has the same horizontal size as the cubic cells close to the surface but stretch vertically all the way from where the surface cells end down to the bottom on which the water rests.

The advantages with this approach is that it significantly reduces the number of cells that have to be processed by \approximating a large number of small cells as a much smaller number of tall cells, and it will still catch surface disturbances and simulate waves with short wavelengths with a high accuracy, and it also simulates waves with really long wave lengths relative to the water depth and an overall motion of the water with a high accuracy. On the other hand, it cannot simulate waves with intermediate wavelength with a very high accuracy.

This method is ideal for simulating flowing water when the main focus does not lie on simulating surface wave properly, but for simulating an ocean, the tall cells are not in their right element.

\subsection{Finite Volume Method on an octree grid}

The aim of this method, is the same as the aim of the restricted tall cell grid approach, to reduce the number of cells that are needed in the simulation. It does so by modeling the grid with an \octree, which allows for easy \idxs{adaptive}{mesh refinement}. The fluid in important regions of the simulation, such as that closest to a surface, is therefore modeled with a fine grid to capture the small scale visual detail, while liquid further away from such regions are modeled with increasingly large cell sizes.

This method was probably used with a restricted \idxs{octree}{grid} for the first time in \citep{Popinet2003}, and was later used in \citealp{Losasso2004} to simulate water and smoke on an unrestricted octree grid in an implementation which had been adapted for \idxs{computer}{graphics} purposes. 

Using this method to simulate surface waves propagating on a large body of water allows an arbitrarily high accuracy, depending on how quickly the cells grow in size when the get farther away from regions of high importance to the simulation, no matter how large the wavelength is. However, it may, just like any other method that discretizes a set of \PDEs, have some problems with getting the correct wave speed for waves with a wavelength close to the \idxs{Nyquist}{frequency}.

It also guarantees that the number of cells that are used in the simulation is $O(N_{\text{s}})$, where $N_{\text{s}}$ is the number of cells visible on the surface, so it ensures the time complexity $O(N_{\text{s}})$ per time step.

\section{Hybrid two-dimensional and three-dimensional methods}

\idxse{two-dimensional}{method}{Two-dimensional methods} is usually much faster than \idxsp{three-dimensional}{method}{s}. On the other hand, three-dimensional methods often have a level of realism that you can't find among two-dimensional methods, and are able to simulate phenomena such as splashes or wave breaking, and the mathematical model describing \FSI often follows naturally. For that reason, \idxe{hybrid method}{hybrid two-dimensional and three-dimensional methods} have been developed, which aim to combine the strengths of two-dimensional and three-dimensional methods and overcome their weaknesses by simulating regions with more complex water motion using with three-dimensional method and regions with less complex water motion with a two-dimensional method, and then couple these simulations with each other. Regions with complex water motion as close to a moving structure such as a ship, and close to the \shoreline if \idxs{wave}{breaking} or \idxs{wave}{shoaling} is important.

In \citep{Thurey2006}, a method known as the \idxs{Lattice-Boltzmann}{method}, which is similar to the \FVM, was simulated in a small box and coupled to a \SWS taking place outside of the box. It showed that with a two-dimensional region covering 35 times as large area as the three-dimensional region, the three-dimensional region still required more than $2/3$ of the entire simulation time, and the update of a three-dimensional cell took in average three times as long time as a two-dimensional cell. Although the simulation didn't run in realtime, it further concluded that given enough computational resources, and in combination with adaptive grids, parallelization and low grid-resolutions, this could be used for interactive, real-time simulations of large water surfaces.

\section{Miscellaneous other methods}

Except from the methods already covered in this chapter, there are a few other methods commonly used in \CFD which for selected reasons are not suitable for simulating oceans.

We have the \LBM which has connections to cellular automata. The \LBM solved the \idxs{discrete}{Boltzmann equation}\index{equation!discrete Boltzmann|see{discrete Boltzmann equation}} and is basically a less powerful \CFD method than the \FVM.

We have the \MAC method, which is a \FVM simulation in which many small, massless marker particles initially are homogeneously distributed in the fluid and then carried with the flow to mark the presence of fluid, but this is basically a less powerful method than the \indexs{volume of fluid}{method}.

We have the \BEM in which the equations of motion are converted to \idxsp{integral}{equation}{s} and solved solely from the \boundaries, and has some interesting potential for the simulation of surface waves. However, it belongs to the category of computationally expensive methods and is for that reason not suitable for simulating the surface of a large body of water in real-time.

We also have the \FEM which is the Lagrangian correspondence to the \FVM, where the simulated region is divided into many small elements which move along with the flow instead of letting the flow go through the cell walls. The \FEM have some advantages over the \FVM, like using the Lagrangian versions of the equations of motion, which don't contain advection terms and hence are unaffected by smearing, and it naturally provides a representation of fluid interfaces. However, it is more complicated to implement than the \FVM, and comes with some nasty surprises like cells being stretched out which makes remeshing necessary, and is still not guaranteed to give results wich have better or even the same quality as results gained with the \FVM. In the field of \CFD, the \FEM is also not that well established, while the \FVM is somewhat of an industry standard.