\chapter{Related work}

% Why is CFD needed for this thesis work?

\CFD is a well established area of researche, and a large number of widely differing methods have been developed during the years. 

\section{Two-dimensional methods}

\idxse{two-dimensional}{method}{Two-dimensional methods} for simulating water waves are the easiest to understand and implement, and often the fastest when it comes to simulation speed, but are not completely realistic models and therefore can't simulate all effects that can be simulated with a \idxs{three-dimensional}{method}. Besides, they tend to become very unintuitive when it comes to modelling \FSI. That doesn't mean that it's necessarily difficult to model \FSI when a two-dimensional method is used, but there are no natural model to describe this interaction. Instead, different models that the programmer think might work have to be tried out, in order to find a good \idxse{empirical}{model}{empirical} one.

\subsection{Two-dimensional PDEs for shallow water}

There exist a number of different two-dimensional \PDEs which describe the evolution of the \idx{free surface elevation}, $\eta$. The ordinary wave equation, \eqref{eq:wave_equation}, will as concluded work badly for simulating surface waves in \idxs{deep}{water}, since it doesn't handle wave dispersion at all, which quickly becomes obvious when wave patters such as the \idxs{Kelvin}{wave pattern} has to be simulated.

On the other hand, the wave equation does work for simulating very weak, low-frequent waves in \idxs{shallow}{water} with constant or very \idxs{mildly varying}{depth}, since those are close to linear, and are almost not affected by wave dispersion at all thanks to the low water depth. For waves with higher amplitude, there exists the \idxs{Boussinesq}{equations} and the \idxs{shallow water}{equations}, which both can be used to simulate large, non-breaking waves in shallow waters.

In \appref{chap:pde_derivation}, a set of new two-dimensional \PDEs intended for simulation of water waves at varying, arbitrary water depths are derived and discussed.

\subsection{Spectral methods}

These methods build on \idxsp{Fourier}{transform}{ing} a representation of the surface in the frequency domain and have, according to \citep{Monnier}, been extensively used and described in the world of computer graphics. This method is characterized by speed and, in contrast to any two-dimensional \PDE that is normally used and that describes the evolution of surface waves, handles wave dispersion very well for surface waves on \idxs{deep}{water}.

The main drawback with this method, except from being inconvenient when modelling \FSI, as any other two-dimensional \CFD method, is that it requires a constant water depth, and hence cannot simulate \idxs{wave}{shoaling}. This issue becomes noticable when the surface water close to the \idxs{shore}{line} is observed, where waves normally behave differently than waves far off shore, but do not when using this method.

\section{Three-dimensional methods}

\idxse{three-dimensional}{method}{Three-dimensional methods} are often highly realistic in the sence that they will be able to simulate all different kind of phenomenas that can be described with the \idxs{Navier--Stokes}{equations} (see \secref{sec:ns_equations}). There are a few exceptions though.

\subsection{Smoothed-Particle Hydrodynamics}

The \SPH method is a highly realistic model that simulates a flow by simulating a large number of small particles. Between each pair of particles that are within a certain \idxs{cut-off}{distance} from each other, there is a repelling or atractive force, described by an \ODE. The interaction between two particles in the simulation is usually modeled by a potential like those used in \MD, for example the \LJ potential. The cut-off distance is used in order to ensure that the number of interactions is $O(N)$, an not $O(N^2)$ as for a system where all pair of particles interact with each other, whene $N$ is the number of particles in the system.

There are a few major advantages with using this method. When the \idxs{Eulerian}{specification of the flow field} is used to describe the fluid motion, the equations tend to become more complicated as they contain \idxsp{advection}{term}{s}. In \SPH on the other hand, the \idxs{Lagrangian}{specification of the flow field} is used and no advection terms are therefore present in the equations which makes the model relatively simple, and it is easily implemented. Besides, no advection of fields with an Eulerian representation is simulated, which prevents additional problems that can arrise during the advection, and conservation of various properties, like \momentum and \energy is usually automatically well preserved as a result of that. In fluid simulations, there is no need to model the air, and there is no need to keep track of where the surface of the fluid is located since this is information that can be extracted during the \idx{post-processing} phase.

On the other hand, the \SPH method requires that the entire simulation domain is filled with small particles, which often means that an \emph{extremely} large number of particles, proportionally to the volume of the fluid, have to be simulated. This implies a very heavy workload on the computer, and as a result of that, \SPH is very seldom used in \idxsp{real-time}{simulation}{s}. Howwever, adaptive particle sizes have been used in order to reduce the amount of particles needed in the less important parts of the fluid, like in the bulk, in order to remedy this problem, a technique that was first used in \citep{Desbrun1999} and later improved in a number of reports, for example in \citep{Yan2009}.

%\subsubsection{Screen Space Meshes}

%\begin{itemize}
%    \item Reference: \textit{\href{http://www.matthiasmueller.info/publications/screenSpaceMeshes.pdf}{Screen Space Meshes}}
%\end{itemize}

%\subsection{Marker-and-Cell method (MAC)}

%\begin{itemize}
%    \item Reference: \textit{Numerical calculation of time-dependent viscous incompressible flow of fluid with a free surface}
%    \item Described in: \textit{\href{http://people.sc.fsu.edu/~jburkardt/pdf/fluid_flow_for_the_rest_of_us.pdf}{Fluid Flow for the Rest of Us: Tutorial of the Marker and Cell Method in Computer Graphics}}
%\end{itemize}

%\subsection{Boundary element method}

%\subsection{Finite element method}

\subsection{Finite Volume Method on a restricted tall cell grid}

The \FVM is a highly realistic model that solves a set of \PDEs by dividing the region of interest into small \idxsp{volume}{element}{s}, and by discretising the fields that are described by the \PDEs into points in the volume elements or on the border of the volume elements, usually a fixed number of points per volume element, as well as discretizing the \PDEs into a number of \ODEs describing the evolution of the discretized fields. The volume elements are commonly referred to as \cells. The \FVM and its usage in \CFD is described in greater detail in \chapref{chap:ns_equations}.

The \FVM was first used on a restricted tall cell grid in \citep{Chentanez2011}. This approach uses an orthogonal grid, where the water closest to the water surface are modeled with small \idxsp{cubic}{cell}{s}, and the water deeper down is modelled with tall cells that has the same horizontal size as the cubic cells close to the surface but stretch vertically all the way from where the surface cells end down to the bottom on which the water rests.

The advantages with this approach is that it significantly reduces the number of cells that have to be processed by \approximating a large number of small cells as a much smaller number of tall cells, and it will still catch surface disturbances and simulate waves with short wavelengths with a high accuracy, and it also simulates waves with really long wave lengths relative to the water depth and an overall motion of the water with a high accuracy. On the other hand, it cannot simulate waves with intermediate wavelength with a very high accuracy.

This method is ideal for sumulating flowing water when the main focus does not lie on simulating surface wave properly, but for simulating an ocean, the tall cells are not in their right elment.

\subsection{Finite Volume Method on an octree grid}

The aim of this method, is the same as the aim of the restricted tall cell grid approach, to reduce the number of cells that are needed in the simulation. It does so by modelling the grid with an \octree, which allows for easy \idxs{adaptive}{mesh refinement}. The fluid in importand regions of the simulation, such as that closest to a surface, is therefore modelled with a fine grid to capture the small scale visual detail, while liquid further away from such regions are modelled with increasingly large cell sizes.

This method was probably used with a restricted \idxs{octree}{grid} for the first time in \citep{Popinet2003}, and was later used in \citealp{Losasso2004} to simulate water and smoke on an unrestructed octree grid in an implementation which had been adapted for \idxs{computer}{graphics} purposes. 

Using this method to simulate surface waves on a large body of water allowes the propagation of surface waves to be simulated with arbitrarily high accuracy, depending on how quickly the cells grow in size when the get farther away from regions of high importance to the simulation, no matter how large the wavelength is, although it may, just like any other method that discretizes a set of \PDEs have some problems with getting the correct wave speed for the smallest waves. It also guarantees that the number of cells that are used in the simulation is $O(N_{\text{s}})$, where N_{\text{s}} is the number of cells visible on the surface.

\section{Twodimensional and threedimensional hybrid methods}

See the work of Andreas Söderström, a LiU student, who has researched about coupling between the spatial domain and the frequency domain. Look at electronic press. (\textit{\href{http://liu.diva-portal.org/smash/get/diva2:359805/FULLTEXT01}{Memory Efficient Methods for Eulerian Free Surface Fluid Animation}}?)