\chapter{Advection of properties}
\label{chap:advectionofproperties}

The choice of \idxs{advection}{scheme} depends entirely on the type of filed that is being exposed to advection and the necessary requirements that field puts on the advection scheme.

\section{Advection of smooth fields}

For a field with smooth changes and no sharp edges, which type the advection scheme is of is usually not that important, as long as it is stable and advects the field properly. Because of the low requirements, \idxsp{linear}{advection scheme}{s} are often used because of their simplicity and for the fact that linearity often brings a lot of benefits with it, for example the possibility to use \idxs{Fourier}{analysis} to study each frequency of the field that is exposed to advection independently of the other frequencies.

%\subsection{Stability of linear advection schemes}
\subsection{Stability and energy preservation}

For an advection scheme to be stable, no frequency, or eigenvector for an irregular grid, must be amplified, that is, any eigenvalue $\lambda_i$ of the $i$th eigenvector must have an absolute value less than or equal to one, i.e.\ $|\lambda_i| \leq 1$. When this requirement is not fulfilled, instability will occur, which is characterized by escalating oscillations and is usually the most evident for short wavelengths.

%\subsection{Smearing caused by linear advection schemes}

It is also often desired that the advection scheme damps the frequencies of the field as little as possible in order to preserve the energy in the field. Just like the amplifying that occurs for an instable advection scheme, damping is usually strongest for short wavelengths. The damping that occurs during advection is caused by the interpolation that is performed when moving the discretized field, and is often referred to as \smearing because of its low-pass filtering character. This causes an energy loss which is often undesired, even though it in many cases is considered a minor issue since it only affects the fine level details. Damping occurs whenever an eigenvalue $\lambda_i$ has an absolute value less than one, i.e.\ $|\lambda_i| < 1$.

\subsection{Error reduction for linear advection schemes}

A method that was originally intended to improve the quality of the \LS method, known as \BFECC, was first described in a publication by \citet{Dupont2003}. However, \BFECC can also be used to improve the energy preservation in any non-ideal, linear operator $A$ acting on a smooth field $\Phi$ (or a vector or any other object on which it makes sense to apply a linear operator), where A is an \approximation of an ideal (at least theoretically existing) operator $A^*$. This ideal operator is both stable and doesn't introduce damping. Let's assume that our goal is to improve the operator $A$ in such a way that the damping is reduced. \BFECC works under the circumstances that $A$ has a corresponding reverse operator $A'$ which is an \approximation of $(A^*)^{-1}$ and has an equal damping effect on the frequencies in the field as $A$, or more specifically, $A'$ has the same \eigenfunctions as $A$, with the corresponding eigenvalues
%
\begin{equation}
\lambda'_i \,=\, \overline{\lambda_i}, \,\forall\,i,
\end{equation}
%
where $\lambda'_i$ is the eigenvalue of the $i$th \eigenfunction of $A'$, and an overline denotes a complex conjugation. For a linear advection scheme, if $A$ is an operator that advects the field in a specific way, $A'$ is simply the operator that uses the same advection scheme as $A$ but that takes a field that has been advected by $A$ back to the position it came from.

In \BFECC, $\Phi$ is first updated forward in time, and then backwards to get another copy of the field, $\Phi_1$, such that $\Phi_1 = A'A\,\Phi$. If there would have been no numerical error, $A'$ would have been the inverse of $A$ and $\Phi$ and $\Phi_1$ would have been equal. The idea is to use the difference $\Phi_1-\Phi$ as information about the error, and use this information to compensate $\Phi$ before applying $A$. Since the error is introduced twice when calculating $\Phi_1$, a compensated variable $\Phi_2$ is calculated using only half of $\Phi_1-\Phi$, or
%
\begin{equation}
\Phi_2 \,=\, \Phi - \tfrac{1}{2}(\Phi_1 \,-\, \Phi).
\end{equation}
%
$A^*\Phi$ can then more accurately be \approximated as $A\,\Phi_2$, which in turn can be expanded to
%
\begin{equation}
\begin{array}{c}
A\,\Phi_2 \,=\, A\,\left(\Phi \,-\, \tfrac{1}{2}(\Phi_1 \,-\, \Phi)\right) \,=\, A\,\Phi \,-\, \tfrac{1}{2}(A\,A'A\,\Phi \,-\, A\,\Phi) \\
=\, A\,\left(\tfrac{3}{2} \,-\, \tfrac{1}{2}A'A\right)\,\Phi \,=\, \tilde{A}\,\Phi,
\end{array}
\end{equation}
%
where $\tilde{A}$ is defined as
%
\begin{equation} \label{eq:compensated_advection_operator}
\tilde{A} \,=\, A\,\left(\tfrac{3}{2} \,-\, \tfrac{1}{2}A'A\right)
\end{equation}
 %
and is an improved approximation of $A^*$. As can be seen, $\tilde{A}$ is the product of $A$ and a \polynomial of $A'A$, and the polynomial could if desired easily be extended with more terms as discussed in the end of this \levelname. Now, if we let this operator act on an \eigenfunction $\phi$, whose \eigenvalue is $\lambda$ when $A$ is acting on it and $\lambda' = \overline{\lambda}$ when $A'$ is acting on it, we see that 
%
\begin{equation}
\renewcommand*{\arraystretch}{1.5}
\begin{array}{c}
\displaystyle
\tilde{A}\,\phi \,=\, A\,\left(\tfrac{3}{2} \,-\, \tfrac{1}{2}A'A\right)\,\phi \,=\, \lambda\,\left(\tfrac{3}{2} \,-\, \tfrac{1}{2}\overline{\lambda}\lambda\right)\,\phi \\
\displaystyle =\, \frac{\lambda}{|\lambda|}\,\left(\tfrac{3}{2}\,|\lambda| \,-\, \tfrac{1}{2}|\lambda|^3\right)\,\phi \,=\, \tilde{\lambda}\,\phi,
\end{array}
\end{equation}
%
where $\tilde{\lambda} = \lambda/|\lambda|\,\left(\tfrac{3}{2}\,|\lambda| \,-\, \tfrac{1}{2}|\lambda|^3\right)$ is the eigenvalue of $\phi$ when $\tilde{A}$ is acting on it. If $|\lambda|$ is written as
%
\begin{equation}
|\lambda| \,=\, 1 - h, \quad h \leq 1
\end{equation}
%
where $h$ is a measure of the amount of damping introduced by $A$, we see that
%
\begin{equation}
\renewcommand*{\arraystretch}{1.5}
\begin{array}{c}
|\tilde{\lambda}| \,=\, \left|\tfrac{3}{2}\,|\lambda| \,-\, \tfrac{1}{2}|\lambda|^3\right| \,=\, \left|\tfrac{3}{2}\,(1-h) \,-\, \tfrac{1}{2}(1-h)^3\right| \\
=\, \left|\tfrac{3}{2}-\tfrac{3}{2}\,h \,-\, \tfrac{1}{2} + \tfrac{3}{2}\,h - \tfrac{3}{2}\,h^2 + \tfrac{1}{2}\,h^3\right| \,=\, \left|1 - \displaystyle\frac{3\,h^2 - 1\,h^3}{2}\right|.
\end{array}
\end{equation}

The damping introduced by $\tilde{A}$ is given by $\tilde{h} = 1 - |\tilde{\lambda}|$. If we assume that possible instabilities in $A$ are small, which means that $-h \ll 1$ if $h < 0$, and consider the fact that $h \leq 1$, $\tilde{h}$ can be expanded to
%
\begin{equation}
\tilde{h} \,=\, 1 - |\tilde{\lambda}| \,=\, 1 - \frac{|\lambda|}{|\lambda|}\left|1 - \displaystyle\frac{3\,h^2 - 1\,h^3}{2}\right| \,=\, \displaystyle\frac{3\,h^2 - 1\,h^3}{2} \,=\, O(h^2),
\end{equation}
%
and we see that the damping introduced by $\tilde{A}$ is of second order if we consider the damping introduced by $A$ to be of first order. We can also notice that if $A$ is unstable, which means that $h < 0$ (although still very small), $\tilde{h}$ will still be positive, which means that $\tilde{A}$ is stable. \BFECC will therefore both heavily reduce damping as well as fix small instabilities in $A$.

Similar schemes of even higher order can be obtained by extending the \polynomial that $\tilde{A}$ is partially composed of with even more terms of type $(A'A)^n$. Since we desire that $\tilde{\lambda} = \lambda/|\lambda|$, we require that the polynomial will result in a division by $|\lambda|$ when acting on $\phi$. This result is \approximated if we let the polynomial be an approximation of $1/\sqrt{A'A}$. The polynomial can therefore be expressed as a \idxs{Taylor}{polynomial} of $1/\sqrt{A'A}$ centered around the point $A'A = 1$. The Taylor polynomial in turn is given by the first terms of the \idxs{binomial}{series} of $(1+x)^{-\frac{1}{2}}$, where $x = A'A - 1$. All these schemes will be stable as long as $A$ is stable, and all schemes with even order will also stabilize the advection process if $A$ contains instabilities that are small enough.

Another approach to increase the order of the advection scheme is to apply \BFECC multiple times to improve the already improved approximation of $A^*$, each time doubling the order of the scheme, but also tripling the number of times $A$ and $A'$ has to be applied. This method is more bulletproof than the method using a Taylor polynomial as it theoretically can stabilize advection schemes with eigenvalues whose absolute values are up to (but less than) $\sqrt{5}$, while the stabilizing abilities of the method using a Taylor polynomial decreases the higher the order of the method becomes. On the other hand, it faster becomes slow when the order of the method increases, as it needs to use $A$ and $A'$\ \ $3^n$ times to obtain the order $2^n$, where $n$ is an integer, while the method using a Taylor polynomial only needs to use $A$ and $A'$\ \ $2\cdot 2^n - 1$ times to obtain the same order of the value $\tilde{h}$ used to measure the damping introduced by $\tilde{A}$ on $\phi$.

\section{Advection of the phase fraction field}

\label{sec:advection_of_phase_fraction}

In contrast to smooth fields, the phase fraction field $\alpha$ is supposed to be either 0 or 1 and have just a very thin region in which the transition between 0 and 1 takes place, i.e. a region in which $0 < \alpha < 1$. When advecting this field, three things are important:

\begin{enumerate}
\item That the mass is conserved, which corresponds to the conservation of $\alpha$,
\item That the field stays bound between 0 and 1, and
\item That the interface stays sharp and doesn't become thicker and thicker.
\end{enumerate}

\subsection{Geometric advection schemes}
\label{sec:geometric_advection_schemes}

\idxse{geometric}{advection scheme}{Geometric advection schemes} work by constructing a geometry for the surface from the discretized \idxs{phase fraction}{field}, and by using the generated geometry to calculate the fluxes between each adjacent pair of cells.

One of the most well known and common geometric advection schemes is \PLIC, which is also one of the advection schemes that introduces the least error. However, it comes at the cost of algorithmic complexity, especially in three-dimensional situations, where the tracking and reconstruction of free surfaces remains complex \citep{Ingram2009}.

Tracking of the free surface tends to be somewhat complex when using a geometric advection scheme. For example, in the two-dimensional case, a naive approach will lead to that volumes in cell corners may be fluxed twice --- once to each adjacent cell --- since there are two cells adjacent to each corner volume. In the three-dimensional case, a naive approach will lead to that edge volumes may be fluxed twice. Corner volumes may even be fluxed thrice since there are three cells adjacent to each corner volume, which makes the three-dimensional case even more complex than the two-dimensional.

This easily leads to unboundedness, as the amount of one phase that is fluxed from a cell during one time step may be larger than what is currently in the cell. Special concern must therefore be taken to the edge and corner volumes in order to prevent fluid from being fluxed to more than one other cell \citep{Rider1997}.

An alternative approach that avoids the complexity mentioned above is described by \citet{Aulisa2003}, where the phase fraction field is fluxed first in the x-direction and then in the y-direction, thus only requiring that each part of the cell that is fluxed, is fluxed to one cell at a time. However, this method is non-symmetric, as it is dependent on in which order the phase fraction field is fluxed in the different dimensions; this fact is discussed further by \citet{Ubbink1999}, although this time the issue is discussed for an \idxs{algebraic}{advection scheme} --- \CICSAM.

\subsection{Algebraic advection schemes}

As opposed to geometric advection schemes, a geometric representation of the surface is never created in an \idxs{algebraic}{advection scheme}. This usually makes these methods a lot simpler than most geometric advection schemes, and the reduced complexity may lead to a performance boost. In this thesis work, an algebraic advection scheme known as \idxs{Hyper-C}{flux limiter}, which was described by \citet{Leonard1988,Leonard1991} and which is optimal for \idx{one-dimensional}, \idxs{incompressible}{flow}, has been used in a form that has been modified to cope with \idxs{compressible}{flow}.

There exist other algebraic advection schemes that are more suitable for multidimensional, incompressible flow, like \CICSAM, which was first described by \citet{Ubbink1999}, and \STACS, which was probably first described at earliest in 2003 \citep{Darwish}. The Hyper-C flux limiter, \CICSAM and \STACS are all derived from something known as the \CBC which is a criterion that --- if followed --- guarantees that the advected field stays bounded.

Furthermore, the \MULES method is also designed for multidimensional incompressible flow, but doesn't build on the \CBC. \MULES was designed for the \idxs{open}{source} software \idx{OpenFOAM} and has after its implementation been described by others than the original authors of the algorithm, for example by \citet{Berberovi2009}. It has also been improved to cope with more than two phases \citep{Kissling2010}, since that wasn't handled very well by the original formulation.

%\subsubsection{Algebraic advection schemes for compressible flows}

However, although \CICSAM, \STACS and \MULES are also designed for and work well for incompressible flow, they would have to be modified to cope with compressible flow. This can often be achieved by advecting \idxsp{volume}{fraction}{s} of both phases separately, and by defining $\alpha$ as the quotient of the volume fraction for one of the phases and the sum of the volume fractions for both phases, instead of simply as the volume fraction for one of the phases. This guarantees that $\alpha$ is equal to either 0 or 1 for a cell that contains only one phase. This problem is also discussed by \citet{Heyns2011}.
