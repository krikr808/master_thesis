\chapter{Advection of properties}

\section{Basic theory of advection}



\section{Geometric advection schemes}

\begin{itemize}
    \item A simple (at least so it seems) scheme: \textit{\href{http://www.lmm.jussieu.fr/~zaleski/nota02.pdf}{A geometrical area-preserving Volume-of-Fluid advection method}}
\end{itemize}

\section{Algebraic advection schemes}

When choosing an \idxs{advection}{scheme}, there are a few things that are important to note. One of those things is that many advection schemes tend to smear out the field that is being transported, which is devastating if the field contains sharp features, such as interfaces between two phases of a flow, that it is important to keep sharp.

A \LUDS, like the \UPWIND scheme, the \CD scheme or the \QUICK scheme, tends to increase the region of the interface. A simple \UPWIND scheme will \idxs{low-pass}{filter} the field that is transported, eventually removing all sharp features present in the field.

Other \LUDSs, like the \CD or \QUICK schemes, will presenve sharp features, but they are on the other hand affected by \idxs{frequency}{dispersion}, thry are \unbounded (there is no limit for how large the field that is being transported can become), and they may even be unstable, especially for large \idxp{time step}{s}.

\subsection{Stability}

When advecting advecting a \idxse{scalar}{field}{scalar} or \idxsp{vector}{field}, it is important that the advection scheme that is used is stable. Unstable advection schemes tend to gradually increase the total \idxe{energy!of a field}{energy}\index{field!energy|see{energy}} of the field, making it go to infinity as the time goes to infinity which causes an explosion.

The \UPWIND scheme is both very simple to implement and unconditionally stable as long as the \CFL conditin is satisfied, and is therefore suitable to use when it is important that the energy of the field that is being transported stays bounded. 

\subsection{Convection Boundedness Criterion (CBC)}

\sloppy
\begin{itemize}
    \item Reference: \textit{Curvature-compensated Convective Transport: SMART a New Boundedness- Preserving Transport Algorithm}
    \item Extended Convective Boundedness Criterion (ECBC): \textit{Discussion on Numerical Stability and Boundedness of Convective Discretized Scheme}
    \item General Convective Boundedness Criterion (GCBC): \textit{\href{http://gr.xjtu.edu.cn:8080/upload/PUB.1673.4/Wei_NHT.pdf}{A New General Convective Boundedness Criterion}}
    \item \red{Convection Boundedness Criterion for arbitrarily unstructured meshes:} \textit{\href{http://powerlab.fsb.hr/ped/kturbo/openfoam/papers/GammaPaper.pdf}{High resolution NVD differencing scheme for arbitrarily unstructured meshes}}
\end{itemize}
\fussy

More:
\begin{itemize}
    \item Normalised Variable Diagram (NVD)
    \item \textit{\href{http://warminski.pollub.plwww.ptmts.org.pl/Waclaw-Koron-2-08.pdf}{Comparison of CICSAM and HRIC High-resolution Schemes for Interface Capturing}}
    \item \textit{\href{http://proceedings.fyper.com/eccomascfd2006/documents/85.pdf}{MODELING OF THE WAVE BREAKING WITH CICSAM AND HRIC HIGH-RESOLUTION SCHEMES}}
\end{itemize}

\subsection{Multidimensional Universal Limiter with Explicit Solution (MULES)}

References: \citep{Berberovi2009,Kissling2010}

See \textit{OpenFOAM-1.5.x/src/finiteVolume/fvMatrices/solvers/MULES/MULES.H} for details

See also \href{http://fds.duke.edu/db/aas/Physics/weller}{Henry R Weller, Professor Emeritus}

% Escape characters
%\& \% \$ \# \_ \{ \}
%\textasciitilde  = ~
%\textasciicircum = ^
%\textbackslash   = \

\begin{itemize}
    \item Described here: \textit{\href{http://link.libris.kb.se/sfxliub?sid=?url_ver=Z39.88-2004&rfr_id=info:sid/bibl.liu.se\%3Axerxes+\%28+PubMed+LiU\%29&rft.genre=article&rft_val_fmt=info\%3Aofi\%2Ffmt\%3Akev\%3Amtx\%3Ajournal&rft.issn=15393755&rft.date=2009&rft.jtitle=Phys+Rev+E+Stat+Nonlin+Soft+Matter+Phys&rft.volume=79&rft.issue=3+Pt+2&rft.spage=036306&rft.atitle=Drop+impact+onto+a+liquid+layer+of+finite+thickness+\%3A+dynamics+of+the+cavity+evolution+&rft.aulast=Berberovi\%C4\%87&rft.aufirst=Edin}{Drop impact onto a liquid layer of finite thickness: Dynamics of the cavity evolution}}
    \item An improvement for more than two phases: \textit{\href{http://www.mathematik.uni-ulm.de/numerik/staff/urban/reports/ECCOMASCFD2010paperfinal.pdf}{A Coupled Pressure Based Solution Algorithm Based on the Volume-Of-Fluid Approach for Two or More Immiscible Fluids}}
\end{itemize}

\subsection{SOLA-VOF}

\begin{itemize}
    \item Reference: \textit{\href{http://www.ewp.rpi.edu/hartford/~ernesto/Su2012/CFD/Readings/SOLA-VOF-1980-P1.pdf}{SOLA-VOF: A Solution Algorithm for Transient Fluid Flow with Multiple Free Boundaries}}
\end{itemize}

\subsection{Hyper-C flux limiter}

\begin{itemize}
    \item Reference: \textit{\href{http://www.water.tkk.fi/wr/kurssit/Yhd-12.112/TVD1.pdf}{The Ultimate Conservative Difference Scheme Applied to Unsteady One-Dimensional Advection}}
\end{itemize}

\subsubsection{Floating mixed cells}

\begin{itemize}
    \item Remedy: \textit{\href{https://e-reports-ext.llnl.gov/pdf/245038.pdf}{A Simple Advection Scheme for Material Interface}}
\end{itemize}

\subsection{Compressive Interface Capturing Scheme for Arbitrary Meshes (CICSAM)}

\begin{itemize}
    \item Reference: \textit{\href{http://ac.els-cdn.com/S0021999199962769/1-s2.0-S0021999199962769-main.pdf?_tid=85161b57da5f4401e55c9d07495e24ea&acdnat=1336167249_a59e4f578adbacf3bff69936c48cdd57}{A Method for Capturing Sharp Fluid Interfaces on Arbitrary Meshes}}
    \item Also described in (by the same author): \textit{\href{http://powerlab.fsb.hr/ped/kturbo/OpenFOAM/docs/OnnoUbbinkPhD.pdf}{Numerical prediction of two fluid systems with sharp interfaces}}
    \item Test with different Courant numbers: \textit{\href{http://www.marin.nl/upload_mm/8/2/c/1807524470_1999999096_2007-ECCOMAS_HoekstraVazAbeilBunnik.pdf}{Free Surface Flow Modelling with Interface Capturing Techniques}}
    \item Improvement 1: \textit{\href{http://powerlab.fsb.hr/ped/kturbo/openfoam/docs/HenrikRuschePhD2002.pdf}{Computational Fluid Dynamics of Dispersed Two-Phase Flows at High Phase Fractions}}
\end{itemize}

\subsection{High Resolution Interface Capturing (HRIC) scheme}

\begin{itemize}
    \item Described here: \textit{\href{http://warminski.pollub.plwww.ptmts.org.pl/Waclaw-Koron-2-08.pdf}{Comparison of CICSAM and HRIC High-resolution Sche\-mes for Interface Capturing}}
\end{itemize}

\subsection{Switching Technique for Advection and Capturing of Surfaces scheme (STACS)}

\begin{itemize}
    \item Reference: \textit{\href{http://webfea-lb.fea.aub.edu.lb/cfd/pdfs/publications2/STACS-Complete.pdf}{Convective Schemes for Capturing Interfaces of Free-Surface Flows on Unstructured Grids}}
\end{itemize}

\subsection{Inter-Gamma Scheme}

\begin{itemize}
    \item Reference: \textit{\href{http://powerlab.fsb.hr/ped/kturbo/openfoam/docs/InterTrack.pdf}{Interface Tracking Capabilities of the Inter-Gamma Differencing Scheme}}
\end{itemize}

\subsection{Constrained Interpolation Profile (CIP) method}

%TODO: Used for advecting fluid interfaces?? At least apparently very good for simple advection.

\begin{itemize}
    \item Reference: \textit{\href{http://www.mech.titech.ac.jp/~ryuutai/paper/JCP2001CIPReviewYabe.pdf}{The Constrained Interpolation Profile Method for Multiphase Analysis}}
\end{itemize}

\section{Advection schemes for compressible flows}

\begin{itemize}
    \item Remedy: Advect both water volume and total volume and then define alpha as the ration between them
\end{itemize}

\subsection{Fast Compressive Surface Capturing Formulation (FCSCF)}

\begin{itemize}
    \item Reference: \textit{\href{http://researchspace.csir.co.za/dspace/bitstream/10204/5282/1/Heyns_2011.pdf}{Free-Surface Modelling Technology for Compressible and Violent Flows}}
\end{itemize}

\section{Smearing during advection}