\chapter{Improvements}

There are a lot of improvements of the method that was implemented in \thisprojectwork that can be done, many of them which are necessary  to make the method be practically useful in a real-time flight simulator.

%\section{Adapdive Meshing}
\section{Adaptive Grid Refinement}

LODs that adapt to the camera placement and the Field of View (FOV)/Viewing frustum in order to limit the number of cells needed in the FVM

View dependent adaptive grid refinement

\section{Fluid--Structure Interaction}

see e.g.:

\begin{itemize}
    \item \textit{\href{http://physbam.stanford.edu/~fedkiw/papers/stanford2010-04.pdf}{Numerically Stable Fluid-Structure Interactions Between Compressible Flow and Solid Structures}}
    \item \textit{\href{http://efdl.as.ntu.edu.tw/research/papers/JCP03GCIBM.pdf}{A ghost-cell immersed boundary method for flow in complex geometry}}
    \item \textit{\href{http://www.cs.columbia.edu/~batty/papers/Batty07.pdf}{A Fast Variational Framework for Accurate Solid-Fluid Coupling}} (solid fraction, non-stick to walls)
\end{itemize}

\subsection{Immersed Boundary Method}

\begin{itemize}
    \item Reference: \textit{\href{http://www4.ncsu.edu/~zhilin/TEACHING/MA798Z/Peskin1.pdf}{The immersed boundary method}}
    \item For compressible flow: \textit{\href{http://www.cecs.wright.edu/~haibo.dong/wp-content/themes/publications/IBM_JCP_2007.pdf}{A sharp interface immersed boundary method for compressible viscous flows}}
\end{itemize}

\subsection{Volume of Solid Method (VOS)}

\begin{itemize}
    \item Reference: \textit{The simulation of fluid-rigid body interaction}
    \item Described in \textit{Numerical Modeling of Deforming Bubble Transport Related to Cavitating Hydraulic Turbines}
\end{itemize}

\subsection{Rigid Fluid method}

\begin{itemize}
    \item Reference: \textit{\href{http://www.amath.unc.edu/Faculty/mucha/Reprints/siggraph04.pdf}{Rigid Fluid: Animating the Interplay Between Rigid Bodies and Fluid}}
\end{itemize}

\subsection{Rotation of rigid bodies}

See \textit{\href{http://en.wikipedia.org/wiki/Euler\%27s_equations_\%28rigid_body_dynamics\%29}{Euler's equations (rigid body dynamics)}}

The six ship motions in the steadily translating system are defined by (see \textit{\href{http://www.shipmotions.nl/DUT/LectureNotes/OffshoreHydromechanics.pdf}{OFFSHORE HYDROMECHANICS}}, p.~18):

Translation:

\begin{itemize}
    \item x-axes (back--front): Surge
    \item y-axes (right--left): Sway
    \item z-axes (bottom--top): Heave
\end{itemize}

Rotation:

\begin{itemize}
    \item x-axes (back--front): Roll
    \item y-axes (right--left): Pitch
    \item z-axes (bottom--top): Yaw
\end{itemize}

\section{Parallellization}

See e.g. \textit{\href{http://gfs.sourceforge.net/papers/agbaglah2011.pdf}{Parallel simulation of multiphase flows using octree adaptivity and the volume-of-fluid method}}

\subsection{Space filling curves}

See e.g. \textit{\href{http://j.teresco.org/research/publications/octpart02/octpart02.pdf}{Dynamic Octree Load Balancing Using Space-Filling Curves}}

and \textit{\href{http://downloads.isrn.com/journals/appmath/2012/246491.pdf}{Parallel Adaptive Mesh Refinement Combined with Additive Multigrid for the Efficient Solution of the Poisson Equation}}

\section{Unconditionally Stable Flows}

\textit{\href{http://www.dgp.toronto.edu/people/stam/reality/Research/pdf/ns.pdf}{Stable Fluids}} (is it simple to implement with te \VOF method and is it still mass conservative after that?)

\SIMPLE (although not completely implicit (?))

Use implicit time-stepping

Switch from explicit to implicit methods + use semi-Lagrangian advection to allow arbitrarily large time steps

\section{Visualization}

\subsection{Marching Cubes}

\subsection{Surface shininess}

Shading/reflection (increased reflection spreading for increased height = shininess as a function of the part of the spectrum that contains the suppressed high-frequency waves)

%\section{Local-time stepping}

%Very dubious technique as you can make the method allow large time steps. (see below)

\section{Perfectly matched layers}

Avoiding spurious reflections at LOD transitions (Absorbing boundary conditions, Perfectly Matched Layers)

See e.g. \textit{\href{http://liu.diva-portal.org/smash/get/diva2:359805/FULLTEXT01}{Memory Efficient Methods for Eulerian Free Surface Fluid Animation}}, which explains explicit dampening, implicit dampening, and wave absorbing boundaries -- the perfectly matched layer approach, and evaluates methods.

\section{Wind waves}

In order to get a realistic \idxs{sea}{state}\index{state of sea|see{sea state}}, it is desirable that the \idxs{water}{surface} \interacts with the \wind in order to give rise to \idxsp{wind}{wave}{s}, so that the sea state is a reflection of the wind speed and direction.

\subsection{Spectral methods}

A very cheap and easy way to simulate wind waves is to use a wind dependent \idxs{wave}{spectrum}, which basically tells how much energy that is stored in each part of the wave spectrum in \average, depending on the \idxs{wind}{velocity}. The wave spectrum $P_h(\vec{k})$ is defined as

\begin{equation} \label{eq:wave_spectrum}
P_{\eta}(\vec{k}) = \left\langle \left|\fdfunc{\eta}(\vec{k})\right|^2 \right\rangle,
\end{equation}

where $\fdfunc{\eta}(\vec{k})$ is the \idxs{Fourier}{transform} of the \idxs{free surface}{elevation} $\eta(\vec{r})$ for the wave vector $\vec{k}$ and the brackets $\langle\rangle$ denotes an \idxs{ensemble}{average}. From that spectrum, a \random sea state is \generated in the \idxs{frequency}{domain}, which is then used to \initialize the simulation. There are different models for describing the wave spectrum for a certain wind velocity, most of which are \empirical. Some of these spectra take into account the wind direction for increased alignment between the surface and the wind, for example the \idxs{Philips}{spectrum} which is used and tweaked in \citep{temp} or the \idxs{Pierson--Moskowitz}{spectrum}.

See \textit{\href{http://graphics.ucsd.edu/courses/rendering/2005/jdewall/tessendorf.pdf}{Simulating Ocean Water}} for a modified Phillis spectrum (increased directional dependence)

\subsubsection{Choppy waves}

Gerstner waves (horizontal offset), almost 200 years ago (see \textit{\href{http://graphics.ucsd.edu/courses/rendering/2005/jdewall/tessendorf.pdf}{Simulating Ocean Water}})

\subsection{Air-water interaction}

\section{Visual effects}

\subsection{Splash and foam}

In rough weather conditions, 

(See \textit{\href{http://en.wikipedia.org/wiki/Sea_foam}{Wikipedia -- Sea foam}} or search for \textit{protein skimming} or \textit{foam fractionation})

Implemented in \textit{\href{http://nguyendangbinh.org/Proceedings/Eurographics/2003/cgf/volume22/issue3/paper127/paper127.pdf}{Realistic Animation of Fluid with Splash and Foam}}

which references \textit{\href{http://citeseerx.ist.psu.edu/viewdoc/download?doi=10.1.1.4.6262&rep=rep1&type=pdf}{Rendering Natural Waters}}

which in turn is based on work presented in the book \textit{\href{http://books.google.se/books?id=xuwFz1bPTHgC}{Oceanic Whitecaps: Their Role in Air-Sea Exchange Processes}} by E. C. Monahan and G. MacNiocaill.

See also \textit{\href{http://www.ias.ac.in/jess/sep2002/Ps18.pdf}{Oceanic whitecaps: Sea surface features detectable via satellite that are indicators of the magnitude of the air-sea gas transfer coefficient}} (also by  E. C. Monahan)

See also \textit{\href{http://cg.informatik.uni-freiburg.de/publications/2012_CGI_sprayFoamBubbles.pdf}{Unified Spray, Foam and Bubbles for Particle-Based Fluids}}

\subsection{Turbulence}

See \textit{\href{http://publications.dice.se/attachments/water\%20interaction\%20ottosson_bjorn.pdf}{Real-time Interactive Water Waves}} (2.1.3) for a discussion of why that is important.

\section{Sharpening of various advected fields}

\subsection{Backward Error Compensation and Forward Error Correction}

\begin{itemize}
    \item Reference: \textit{\href{http://smartech.gatech.edu/xmlui/bitstream/handle/1853/29473/2002-389.pdf}{Back and forth error compensation and correction methods for removing errors induced by uneven gradients of the level set function}}
    \item Applied to the velocity field and images: \textit{\href{http://www.gvu.gatech.edu/~jarek/papers/FlowFixer.pdf}{FlowFixer: Using BFECC for Fluid Simulation}}
\end{itemize}

\section{Various optimizations}

* Pure code optimizations
* Local time stepping (dubious improvement)